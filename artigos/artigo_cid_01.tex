%\documentclass[journal, onecolumn, letterpaper]{IEEEtran}
%\documentclass[journal,onecolumn]{IEEEtran}
% \documentclass[conference]{IEEEtran}
\documentclass[a4paper, 12pt, onecolumn,singlespacing]{article}

% The preceding line is only needed to identify funding in the first footnote. If that is unneeded, please comment it out.
\usepackage[level]{fmtcount} % equivalent to \usepackage{nth}
% \include{util}
\usepackage[portuguese, brazil, english]{babel}
\usepackage{multirow}
\usepackage{array} % for defining a new column type
\usepackage{varwidth} %for the varwidth minipage environment
\usepackage[super]{nth}
\usepackage{authblk}
\usepackage{cite}
\usepackage{amsmath,amssymb,amsfonts}
\usepackage{ulem}
\usepackage{graphicx}
% \usepackage{subfig}
\usepackage{textcomp}
\usepackage{xcolor}
\usepackage{mathptmx}
\usepackage[T1]{fontenc}
\usepackage{textcomp}
\usepackage{titlesec}
\usepackage{helvet}
\usepackage{gensymb}
\usepackage{setspace} % espacamento entre linhas
\usepackage{pgfplots}
\usepackage{tikz}
\usepackage{subcaption}
\usepackage{minted}
\usepackage[left=2cm, right=2cm, bottom=2cm, top=2cm]{geometry} 
\usepackage{makecell}
\usepackage{pdfpages}
\usepackage[ISO]{diffcoeff}
\usepackage{lscape}
\usepackage{tabularx}
\usepackage{booktabs}

\renewcommand\theadalign{bc}
\renewcommand\theadfont{\bfseries}
\renewcommand\theadgape{\Gape[4pt]}
\renewcommand\cellgape{\Gape[4pt]}

%dashed line
\usepackage{booktabs, makecell}
\renewcommand\theadfont{\bfseries}
\renewcommand\theadgape{}
\usepackage{arydshln}
\setlength\dashlinedash{0.2pt}
\setlength\dashlinegap{1.5pt}
\setlength\arrayrulewidth{0.3pt}

% padrao 1.5 de espacamento entre linhas
\setstretch{1.5}

\title{Tradução de Artigo - FRACTAL MODEL OF A COMPACT INTRACLOUD DISCHARGE. I. FEATURES OF THE STRUCTURE AND EVOLUTION - D. I. Iudin and S. S. Davydenko}

\author[1]{Augusto Mathias Adams}
\affil[1]{augusto.adams@ufpr.br}
\setcounter{Maxaffil}{0}
\renewcommand\Affilfont{\itshape\small}

\begin{document}
	% Seleciona o idioma do documento
	\selectlanguage{brazil}
	
	% título
	\maketitle

	\section{RESUMO}
	
	Nós examinamos as características da emissão eletromagnética de uma descarga compacta intra-nuvem (CID) dentro do contexto da abordagem fractal [1] descrita na primeira parte do artigo. A descarga compacta intra-nuvem é considerada como resultado da interação elétrica de duas estruturas do tipo streamer bipolar previamente desenvolvidas nas regiões de campo elétrico forte dentro da nuvem de tempestade. Para estimar a emissão eletromagnética da descarga, a estrutura complexa em forma de árvore das correntes elétricas nas fases preliminar e principal da CID foi representada como a soma de um componente médio linear de grande escala que varia relativamente lentamente e constituintes pequenos e rápidos correspondentes à formação inicial de canais condutores elementares da árvore de descarga. A corrente linear média da descarga é considerada como uma fonte efetiva de emissão VLF/LF tanto nas fases preliminar quanto principal de uma CID. Componentes eletrostáticos, de indução e de radiação do campo elétrico em diferentes distâncias da corrente média são calculados levando em consideração características específicas de ambas as fases da descarga dentro do modelo de linha de transmissão. É mostrado que na fase preliminar apenas o componente eletrostático pode ser principalmente detectado, enquanto na fase principal todos os componentes acima mencionados do campo elétrico podem ser medidos de forma confiável. A dependência do campo elétrico de radiação na fase principal em relação ao comprimento do canal de descarga e à velocidade de propagação da frente de corrente é analisada. Verifica-se que, devido à expansão bidirecional da corrente na fase principal de uma CID, o pulso do campo de radiação permanece estreito em uma ampla gama de parâmetros de descarga. As correntes de pequena escala correspondentes à quebra inicial entre as células vizinhas do domínio da descarga são consideradas como as fontes de radiação HF/VHF. É mostrado que a emissão HF/VHF na fase preliminar é negligenciável em comparação com a emissão na fase principal. Também é estabelecido que na fase principal, em primeiro lugar, a explosão de emissão HF/VHF correlaciona-se bem com o pico inicial do pulso do campo elétrico VLF/LF e, em segundo lugar, seu espectro corresponde à lei de potência com um expoente entre -2 e -1.
	
	\section{HISTÓRIA DOS ESTUDOS E MODELOS FÍSICOS DE DESCARGAS COMPACTAS INTRANUVEM}
	
	Os resultados de observações de descargas intra-nuvem incomuns, cujo campo elétrico na zona distante tinha a forma de um único pulso bipolar com duração de 10 a 30 $\mu s$ e acompanhado por uma explosão curta de alta potência de radiação de alta frequência, foram publicados pela primeira vez em [1]. Mais tarde, essas fontes elétricas foram classificadas como uma classe separada e denominadas descargas compactas intra-nuvem (CIDs). Apesar dos estudos plurianuais das CIDs e do significativo volume de dados experimentais acumulados até o momento, a natureza desse fenômeno é, em muitos aspectos, pouco clara. Neste artigo, propomos um novo modelo de descargas compactas intra-nuvem, que se baseia na abordagem fractal para a descrição de sua estrutura elétrica.
	
	\subsection{HISTÓRICO DOS ESTUDOS DE DESCARGAS COMPACTAS INTRANUVEM}
	
	Os resultados da detecção de radiação eletromagnética de descargas elétricas nas nuvens com propriedades incomuns foram publicados no início dos anos 80 do século passado [1]. A característica principal dessas descargas era uma explosão de alta potência de radiação de alta frequência nas frequências de 3 a 300 MHz, cujo nível excedia consideravelmente os valores das descargas intra-nuvem típicas e das descargas entre nuvem e solo. Sincronamente com a explosão de radiação de alta frequência, sensores terrestres não calibrados registraram uma variação característica no campo elétrico de baixa frequência na forma de um pulso bipolar com duração total de 10 a 20 $\mu s$ (no experimento [1], o critério para o início do pulso de campo elétrico era o excedente da intensidade de radiação do nível dado, relativamente alto, em uma frequência de 3, 139 ou 295 MHz). De acordo com [1], a duração de um pulso bipolar excedeu consideravelmente a duração do pulso de alta frequência registrado nas proximidades do máximo do pulso do campo elétrico, e a amplitude do pulso bipolar era cerca de 1/3 do valor de pico do campo elétrico para um retorno típico. Em tal caso, a direção do campo elétrico no primeiro semiperíodo de um pulso bipolar era oposta à direção do campo de tempo bom, ou seja, o pulso bipolar tinha uma polaridade oposta em comparação com as rajadas de campo para a descarga negativa da nuvem para o solo.
	
	Nos anos seguintes, o estudo experimental e teórico dessas descargas tem sido objeto de um número significativo de trabalhos, que proporcionaram uma visão bastante completa da radiação eletromagnética. Em particular, os autores de [2] realizaram medições de banda larga do componente de radiação do campo elétrico $E$ e sua derivada $\frac{dE}{dt}$ para dezenas de pulsos bipolares curtos e, utilizando os dados de medição de $\frac{dE}{dt}$, calcularam as densidades espectrais de energia e potência do pulso do campo elétrico. Normalizando e fazendo uma média dos pulsos bipolares de polaridade positiva (quando o campo elétrico no primeiro semiciclo do pulso é direcionado para cima para o observador em terra), os autores de [2] obtiveram os seguintes parâmetros do pulso: duração da parte inicial (positiva) do pulso em HPPW de 2,4 $\mu s$, intervalo entre os máximos do campo positivo e negativo de 11 $\mu s$, razão entre esses máximos de 8,9 e duração total de 20 a 30 $\mu s$. O valor de pico médio do campo elétrico do pulso calculado para uma distância de 100 km da descarga foi (8 $\pm$ 5,3) V/m (0,72 do valor correspondente para o pulso médio do campo de retorno), e a potência de pico média da derivada do campo foi igual a (20 $\pm$ 15) V/($m \dot \mu s$). Nenhuma variação no campo elétrico de fundo foi observada em um intervalo de medição de 400 $\mu s$, e o perfil da derivada do campo elétrico do pulso continha um componente considerável de ruído em comparação com o perfil correspondente para outras descargas nas nuvens. Em [2], foram detectados pulsos curtos de ambas as polaridades (os pulsos positivos predominaram), mas não foi observada uma dependência significativa da forma do pulso em relação à polaridade do pulso. A análise quantitativa do espectro de energia do campo elétrico de pulsos bipolares mostrou que, em frequências de centenas de quilohertz a 8 MHz, esse espectro é comparável ao espectro de radiação de um típico pulso de retorno, mas já em uma frequência de 18 MHz, na qual o espectro de radiação do pulso de retorno diminui para o nível de ruído, o espectro dos pulsos bipolares curtos o excede em 16 dB, estendendo-se acima de 50 MHz. Assim, em [2], a conclusão de [1] de que essas descargas são a fonte mais poderosa de radiação de alta frequência nas nuvens de tempestade foi confirmada. Medições de banda larga dos pulsos bipolares do campo elétrico em frequências de 3 a 50 MHz também foram apresentadas em [3], onde foi constatado que os pulsos de polaridades positiva e negativa têm durações diferentes e não há correlação temporal dos pulsos de polaridade positiva e negativa com qualquer atividade de relâmpago conhecida na nuvem.
	
	Logo após a publicação dos dados de medição realizados em solo mencionados em [4-6], mais de 500 pares de pulsos incomuns de radiação de alta frequência foram registrados a bordo do satélite ALEXIS. A dispersão dos pares de explosões, claramente visível em seus espectros dinâmicos, indicou que sua fonte está localizada sob a ionosfera. Portanto, em [4, 5], tais explosões foram chamadas de pares de pulsos transionosféricos (\textit{TIPPs}, na sigla em inglês). De acordo com [4, 5], na faixa de 28 a 95 $MHz$, os pares de pulsos transionosféricos são explosões de radiação com duração de 1 a 20 $\mu s$ (com duração média de 2 a 4 $\mu$s), separadas por um intervalo de tempo de 10 a 100 $\mu s$ (com intervalo médio de 50 $\mu$s). A intensidade do pulso excedeu em 20 a 40 dB o nível de fundo, e a potência do pulso nessa faixa foi pelo menos uma ordem de magnitude maior do que a potência irradiada de uma descarga típica de raios. Na faixa de 117 a 166 \textit{MHz}, as características estatísticas dos pares de pulsos transionosféricos diferiram apenas ligeiramente em relação às características na parte inferior da faixa de frequência muito alta (VHF, de 30 a 300 MHz), exceto pela diminuição na dispersão e atraso médio entre os pulsos, que foi de 37 $\mu$s [6]. Tanto na parte de baixa frequência quanto na parte de alta frequência da faixa VLF, durante o registro de um par de pulsos (com duração de registro de 7 a 100 ms), a radiação de descargas de raios não foi observada como regra geral. Os autores de [4-6] fizeram duas suposições sobre a origem dos pares de pulsos. A primeira suposição foi que o segundo pulso é a reflexão da radiação de uma fonte pulsada de alta altitude a partir da superfície da Terra. A segunda suposição é que ambos os pulsos são irradiados por fontes diferentes, mas correlacionadas, cuja conexão não está clara.
	
	A gravação de pares de pulsos subionosféricos (\textit{SIPPs}) na faixa de alta frequência (HF, de 3 a 30 $MHz$) durante observações em solo não esclareceu a origem e a natureza do segundo pulso [7]. Uma vantagem das observações em solo foi a determinação mais precisa da posição da fonte: o raio da região de sua possível localização variou de 300 a 520 $km$, à medida que a altitude da fonte variou de 5 a 15 $km$, sendo muito menor que o raio da área de vigilância do ALEXIS, que era de 3000 $km$. Isso levou à conclusão de que cerca de 350 pares de pulsos subionosféricos dos 500 observados correspondiam a regiões de atividade de tempestades próximas, e em outros casos a atividade de tempestades foi observada a uma distância de até alguns milhares de quilômetros. Além disso, todos os pares de pulsos subionosféricos foram registrados no período de 20:00 às 04:00 LT, o que também indica sua relação com as tempestades regionais. As características dos pares de pulsos subionosféricos e transionosféricos eram semelhantes, exceto pela ausência de dispersão e divisão de modos nas observações em solo da maioria dos eventos. A comparação das características de dispersão da radiação refletida na ionosfera de uma descarga de raio remota e sete pares de pulsos subionosféricos registrados com dispersão permitiu aos autores concluir sobre a localização próxima (não mais do que algumas dezenas de quilômetros) de suas fontes e da nuvem de tempestade. Em geral, de acordo com [7], a gravação de pares de pulsos subionosféricos é evidência da existência de uma fonte separada para o segundo pulso, uma vez que nas observações em solo a presença do segundo pulso não pode ser explicada pela reflexão do pulso inicial da superfície terrestre.
	
	Características importantes de pulsos bipolares curtos foram obtidas em [8] com base nos dados de dois sistemas multiponto de registro do campo elétrico e radiação de alta frequência na faixa de 3 a 30 $MHz$. Nesse caso, as características de ajuste do sistema de disparo dos sensores permitiram o registro apenas dos pulsos de polaridade positiva. Além das características médias (ao longo de 24 eventos) dos pulsos bipolares curtos do campo elétrico e das respectivas rajadas de radiação de HF para três tempestades no Novo México e Texas (EUA; ver Tabela 1), a localização da fonte de radiação foi determinada pela primeira vez nesse artigo. A distância até a fonte foi determinada pelo tempo de atraso do sinal nas estações espacialmente separadas, e a altitude da fonte, pelo atraso na chegada dos sinais refletidos pela ionosfera e pela Terra e ionosfera em relação ao pulso que se propaga sem reflexões, pela distância mais curta. Como resultado, verificou-se que as fontes de pulsos bipolares estão localizadas em nuvens de tempestade a altitudes de 8 a 11 km acima do nível do mar, próximas a regiões com nível de reflexão do sinal de radar acima de 40 dBZ. Assim como nos estudos anteriores, os autores de [8] observaram nenhuma correlação entre pulsos bipolares curtos e qualquer atividade conhecida de raios na nuvem e relataram que os pulsos bipolares eram quase sempre o único evento no intervalo de 5 a 50 $ms$. O mesmo é verdade para a rajada de radiação de HF.
	
	\begin{table}[htbp]
		\centering
		\caption{Average characteristics of short bipolar pulses of the electric field and synchronous high-frequency radiation bursts in the range 3--30 MHz according to [8].}
		\begin{tabularx}{\textwidth}{@{}p{10cm}c@{}}
			\toprule
			\multicolumn{2}{c}{\textbf{Characteristics of Short Bipolar Pulses of the Electric Field}} \\
			\midrule
			Field rise time (from 10\% to 90\% of the maximum), $\mu$s & $2.3 \pm 0.8$ \\
			Duration of the initial burst of the electric field (at the half-peak level), $\mu$s & $4.7 \pm 1.3$ \\
			\midrule
			Total pulse duration, $\mu$s & $25.8 \pm 4.9$ \\
			\midrule
			Peak value of the electric field at the initial (first) burst of a bipolar pulse (at a distance of 100 km from the source), V/m & $9.5 \pm 3.6$ \\
			\midrule
			Peak value of the electric field at the second burst of a bipolar pulse (at a distance of 100 km from the source), V/m & $-3.9 \pm 1.6$ \\
			\midrule
			Ratio of the peak amplitudes of the electric field at the first and second bursts of a bipolar pulse & $2.7$ \\
			\midrule
			Ratio of the peak amplitudes of the electric field of a bipolar pulse and the pulse due to the return stroke & $0.71$ \\
			\midrule
			Ratio of the peak amplitudes of the electric field of a bipolar pulse and the pulse due to the intracloud discharge & $2.6$ \\
			\midrule
			\multicolumn{2}{c}{\textbf{Characteristics of High-Frequency Radiation}} \\
			\midrule
			Duration, $\mu$s & $2.8 \pm 0.8$ \\
			Peak value of the field (at a distance of 10 km in the 1-kHz band), mV/m & $2.4 \pm 1.1$ \\
			\midrule
			Ratio of the peak amplitudes of the HF field of a compact intracloud discharge and return stroke & $9.9$ \\
			\midrule
			Ratio of the peak amplitude of the HF field of compact and conventional intracloud discharges & $29$ \\
			\bottomrule
		\end{tabularx}
	\end{table}

	Como resultado da comparação dos dados acima sobre a rajada de radiação de HF que é síncrona com um pulso bipolar curto do campo elétrico e os parâmetros dos pares de pulsos transionosféricos, os autores de [8] chegaram à conclusão de que o fenômeno físico é o mesmo. O segundo pulso transionosférico é a reflexão do pulso inicial da superfície da Terra, o que proporciona uma interpretação natural dos diferentes atrasos entre os pulsos nas diferentes faixas de frequência [4-6]. O ponto é que as observações dos pares de pulsos na parte inferior da faixa VLF foram realizadas sobre regiões equatoriais (ver [4, 5]), e na parte superior da faixa VLF, sobre o território dos EUA (ver [5]), onde a altitude das nuvens de tempestade é menor do que perto do equador. Se a fonte de pulsos VLF estiver localizada nas proximidades do topo de uma nuvem de tempestade, então o atraso entre os pulsos é diferente nesses casos. Essa suposição também explica a situação comum em que a intensidade do segundo pulso no par é maior do que a do primeiro pulso (ver [4, 5]), o que pode ser devido ao padrão de diretividade não homogêneo da fonte.
	
	Uma vez que os pulsos bipolares curtos do campo elétrico em [8] foram registrados exclusivamente na zona distante, seus parâmetros podem ser convenientemente usados para a estimativa das características da fonte correspondente. De acordo com [8], a variação média do momento dipolar da fonte de pulsos bipolares foi de 0,38 C·km, com valores mínimos e máximos de 0,26 e 0,8 C·km, respectivamente. A corrente média da fonte variou entre algumas dezenas e algumas centenas de quiloampères, à medida que a escala espacial da fonte variou de 300 a 1000 m, e a velocidade de propagação do pulso da corrente da fonte pode ser próxima à velocidade da luz. Levando em consideração as características únicas da emissão de rádio e os tamanhos relativamente pequenos dessas fontes, os autores de [9] sugeriram classificá-las como uma classe separada, denominada descargas compactas intra-nuvem (CIDs).
	
	Os resultados importantes sobre a natureza das CIDs foram obtidos utilizando o LMA (Lightning Mapping Array), um sistema terrestre de vários pontos para registro da radiação VHF de descargas de raios, localizado no Novo México (EUA). O sistema não conseguia determinar a forma e a duração de um pulso de radiação HF, mas localizava a fonte com uma precisão de até algumas centenas de metros pelo atraso da chegada do pulso nos pontos de medição [10]. Com base nos dados do LMA, descobriu-se que as descargas compactas intra-nuvem positivas estão localizadas entre as regiões das principais cargas positivas e negativas e, ao contrário das observações anteriores, estão sempre associadas à descarga inicial de uma descarga intra-nuvem convencional. Nesse caso, a maioria das descargas intra-nuvem se desenvolveu sem a ocorrência de CIDs na fase inicial. De acordo com [10], as fontes de radiação VLF de alta potência das CIDs estavam distribuídas na região com uma escala típica de algumas centenas de metros a um quilômetro e, ao contrário das descargas de raios convencionais, não eram descritas por um conjunto de fontes pontuais. Estimativas [10, 11] mostraram que a potência da radiação CID na faixa de 60-66 MHz poderia atingir 300 kW e é mais de uma ordem de grandeza maior do que a potência radiada da descarga intra-nuvem subsequente, que não ultrapassa 10 kW.
	
	As possibilidades de registrar a radiação das descargas compactas intra-nuvem se tornaram muito mais amplas com o lançamento do satélite FORTE em 1997 [12]. Os receptores do satélite registravam a radiação nas frequências de 26 a 48 MHz simultaneamente para duas polarizações diferentes ou a radiação de uma polarização simultaneamente nas faixas de 26-48 e 118-140 MHz. Além disso, sensores ópticos de alta sensibilidade foram instalados a bordo do satélite. O registro por satélite de pares de pulsos transionosféricos no caso em que a localização da descarga não é conhecida permitiu determinar a relação entre as altitudes das fontes de diferentes eventos [12], e a altitude da descarga se os dados de coordenadas geográficas da descarga estiverem disponíveis [13], assumindo que o segundo pulso é o reflexo do primeiro pulso na superfície da Terra. As coordenadas geográficas da descarga foram determinadas, em particular, a partir dos dados do registro do campo elétrico na faixa VLF/HF (nas frequências de 3 a 300 kHz) pelos sistemas de vários pontos LASA (Los Alamos Sferic Array) [14] e NLDN (National Lightning Detection Network) [15].
	
	A análise dos dados de medição terrestre e por satélite da radiação de descargas compactas permitiu determinar sua localização, altitude e brilho óptico. Em primeiro lugar, foi demonstrado, com precisão na determinação da localização do satélite FORTE, que as fontes de um pulso bipolar curto do campo elétrico e pares de pulsos transionosféricos coincidem no espaço e no tempo nos casos de registro combinado dos eventos nas faixas VLF/LF e HF/VHF [16, 17]. Verificou-se que os pares de pulsos transionosféricos de alta frequência, com potência efetiva superior a 40 kW na faixa de 26 a 48 MHz e duração de 3 a 5 $mu s$, são fracamente polarizados, incoerentes, observados separadamente de outros fenômenos ou iniciam uma descarga intra-nuvem, cuja radiação na faixa de VHF é muito mais fraca do que a radiação CID [16, 18]. Quando a radiação CID é registrada simultaneamente nas faixas de baixa frequência (LASA) e alta frequência (FORTE), constatou-se que o número de pares de pulsos transionosféricos excede o número de pulsos de campo elétrico bipolar curto. O pulso de radiação CID de uma descarga compacta, regra geral, não é acompanhado do componente de baixa frequência no caso em que o CID inicia uma descarga intra-nuvem e o desenvolvimento do líder. Se o pulso de alta frequência é acompanhado pela formação de um pulso de campo elétrico bipolar curto, então, inversamente, a descarga intra-nuvem não se desenvolve [18]. Independentemente da formação do pulso de campo elétrico bipolar, as descargas compactas intra-nuvem irradiam de forma relativamente fraca na faixa óptica em comparação com as descargas de raios convencionais [19, 20]. Entre os resultados recentes das observações por satélite da radiação CID, mencionamos o artigo [21], no qual foram observadas diferenças específicas dos pulsos transionosféricos diretos e refletidos pela Terra em uma frequência de 130 MHz, e o efeito ionosférico pôde ser negligenciado.
	
	O padrão de radiação de alta frequência das CIDs obtido por meio de experimentos realizados em satélite é complementado de forma natural com os resultados do registro da radiação CID por sistemas terrestres. Isso ocorre porque algumas características das CIDs, em particular a polaridade e a estrutura espacial detalhada, só podem ser determinadas pelo registro do campo elétrico terrestre. Como principais resultados das observações terrestres de CIDs obtidas nos últimos anos, mencionamos o seguinte. Em [17], com base na análise de quase 100.000 pulsos bipolares curtos registrados pelo sistema LASA, verificou-se que as descargas compactas de polaridade diferente aparecem em altitudes diferentes. A altitude média acima da superfície da Terra para as CIDs positivas e negativas foi de 13 km (para a faixa de altitudes de descarga de 7 a 15 km) e 18 km (para a faixa de altitudes de descarga de 15 a 20 km), respectivamente, e a proporção entre o número de descargas positivas e negativas foi de 58:42. As diferentes altitudes de início das CIDs de polaridade positiva e negativa correspondem às regiões com diferentes direções do campo elétrico intra-nuvem. Mais especificamente, a faixa inferior de altitudes corresponde ao limite das regiões das principais cargas negativas e positivas, e a faixa superior corresponde ao limite das regiões da principal carga positiva e da camada de blindagem negativa.
	
	Os autores do artigo [22] apresentaram as primeiras (e únicas até agora) estimativas do momento de carga das CIDs, com base nos registros do campo elétrico de sete descargas compactas na zona próxima. De acordo com [22], o momento de carga das CIDs variou entre 0,15 e 2 C·km, com um valor médio de cerca de 1 C·km. Os autores do artigo [23] analisaram a correlação entre a taxa de ocorrência de CIDs e as descargas convencionais de raios, bem como a taxa de convecção nas nuvens de tempestade sobre as Grandes Vales (EUA) de maio a julho de 2005. A comparação dos dados do LASA e do NLDN com os resultados do radar mostrou o seguinte. Em primeiro lugar, verificou-se que a taxa de ocorrência de CIDs apresenta uma correlação fraca com a taxa de convecção na nuvem de tempestade e a taxa de ocorrência das descargas convencionais de raios. Por um lado, nas nuvens com convecção fraca, as descargas compactas não foram detectadas, enquanto nas nuvens com CIDs, ao contrário, sempre houve uma convecção intensa (em particular, para essas nuvens, a região com nível de reflexão do sinal de radar superior a 30 dBZ estava localizada 4-5 km acima, e o máximo do nível de reflexão era de 10-13 dB mais alto do que para as descargas convencionais de raios). Por outro lado, as CIDs não puderam ocorrer nas células de tempestade com a convecção mais intensa. Uma característica importante das descargas compactas foi o agrupamento espaço-temporal. Ou seja, a maioria das CIDs foi registrada em várias tempestades intensas (em particular, 34\% de todas as CIDs foram detectadas na frente de nuvens de tempestade intensas durante 11 horas em 24-25 de maio de 2005). Também foi constatado que o número de CIDs não excede 0,5\% do número total de descargas de raios. As CIDs de ambas as polaridades puderam ser registradas em uma única nuvem, sendo que o número de CIDs positivas correspondia a aproximadamente 77\% de todas as descargas compactas registradas.
	
	As conclusões sobre o efeito da taxa de convecção no desenvolvimento das CIDs e na predominância das descargas compactas positivas, obtidas em [23], geralmente estão em conformidade com os resultados de medições em outras regiões e condições climáticas (consulte, em particular, [14, 24–26]). No entanto, a taxa de ocorrência de CIDs em relação ao número total de descargas de raios varia drasticamente dependendo da latitude geográfica das observações, variando de mais de 3\% na Flórida (EUA) [14, 24] a 0,034\% no norte da China [25], chegando à ausência total na Suécia [26]. A análise das observações das CIDs positivas na Flórida (EUA) confirmou sua natureza predominantemente isolada, com aproximadamente 73\% dos eventos não estando relacionados às descargas convencionais de raios. Pares de descargas compactas (cerca de 4\% dos eventos), separados por um intervalo de tempo de 43 a 181 ms [27], foram registrados pela primeira vez lá. Em [28], com base nos registros de 244 pulsos bipolares positivos curtos do campo elétrico, foi sugerido categorizá-los por forma (o "clássico" pulso bipolar suave foi observado apenas em 5\% dos casos). Também foi observado em [28] que as CIDs são geralmente isoladas no espaço e no tempo em relação a outras descargas na nuvem, enquanto as altitudes das CIDs correspondem bem aos resultados de [17]. Uma análise comparativa das propriedades das CIDs positivas e negativas com base nos dados de medição do campo elétrico VLF/LF é apresentada em [29], onde, além das diferentes altitudes de localização das CIDs de polaridade diferente, foi observado que as CIDs negativas, em geral, não apresentam uma maior amplitude do pulso bipolar do campo elétrico e estão mais separadas no tempo. Também foi mencionado que, apesar do maior número médio de CIDs positivas, a taxa de ocorrência das descargas compactas negativas foi maior em alguns intervalos de tempo do que a das positivas.
	
	\subsection{MODELOS FÍSICOS DE DESCARGAS COMPACTAS INTRA-NUVEM (CID)}
	
	Uma das primeiras suposições sobre a natureza das CIDs foi feita em [1], onde foi sugerido que os surtos de radiação observados podem ser interpretados como consequência do processo K muito rápido. No entanto, os tempos característicos de variação do campo elétrico nos processos K (cerca de 1 ms) excedem em muito as escalas de tempo das CIDs, e essas variações geralmente têm a forma de uma sequência de saltos separados por 10 a 30 ms [30], em contradição com a natureza predominantemente singular dos surtos de campo elétrico de baixa frequência para CIDs. Além disso, as descargas de campo elétrico de milissegundos que são típicas dos processos K não são observadas para as CIDs, e a potência da radiação VHF excede em ordens de grandeza o valor correspondente para os processos K [31, 32]. Nesse sentido, a interpretação proposta em [1] para as CIDs não é satisfatória.
	
	Outro modelo para a ocorrência da descarga compacta é baseado no mecanismo de quebra por elétrons em fuga [33, 34]. Esse mecanismo baseia-se no fato de que a força de desaceleração de um elétron no meio é inversamente proporcional à sua energia. Portanto, na presença de um campo elétrico externo que excede algum valor limite, um elétron semente energético pode ser constantemente acelerado e adquirir energia suficiente para ionizar moléculas neutras. Como resultado, surgem os chamados elétrons secundários no meio. Parte deles possui alta energia e também pode se tornar elétrons em fuga. A avalanche resultante de elétrons em fuga e um grande número de elétrons secundários lentos pode alterar drasticamente a condutividade do meio e levar a uma quebra elétrica [35]. A fonte de elétrons semente são os raios cósmicos, cuja energia determina, em muitos aspectos, a forma do pulso de corrente formado e a emissão eletromagnética de banda larga da avalanche de elétrons.
	
	
	
\end{document}