%\documentclass[journal, onecolumn, letterpaper]{IEEEtran}
%\documentclass[journal,onecolumn]{IEEEtran}
% \documentclass[conference]{IEEEtran}
\documentclass[a4paper, 12pt, onecolumn,singlespacing]{article}

% The preceding line is only needed to identify funding in the first footnote. If that is unneeded, please comment it out.
\usepackage[level]{fmtcount} % equivalent to \usepackage{nth}
% \include{util}
\usepackage[portuguese, brazil, english]{babel}
\usepackage{multirow}
\usepackage{array} % for defining a new column type
\usepackage{varwidth} %for the varwidth minipage environment
\usepackage[super]{nth}
\usepackage{authblk}
\usepackage{cite}
\usepackage{amsmath,amssymb,amsfonts}
\usepackage{ulem}
\usepackage{graphicx}
% \usepackage{subfig}
\usepackage{textcomp}
\usepackage{xcolor}
\usepackage{mathptmx}
\usepackage[T1]{fontenc}
\usepackage{textcomp}
\usepackage{titlesec}
\usepackage{helvet}
\usepackage{gensymb}
\usepackage{setspace} % espacamento entre linhas
\usepackage{pgfplots}
\usepackage{tikz}
\usepackage{subcaption}
\usepackage{minted}
\usepackage[left=2cm, right=2cm, bottom=2cm, top=2cm]{geometry} 
\usepackage{makecell}
\usepackage{pdfpages}

\usepackage{hyperref}
\usepackage{fancyhdr}
\renewcommand{\headrulewidth}{1pt}
\renewcommand{\footrulewidth}{0.5pt}
\fancyhf{} % limpa os cabecalhos e rodapés
\fancyhead[C]{\textit{CURSO DE ENGENHARIA DOS RAIOS - TE981} } % define o cabeçalho personalizado
\fancyfoot[C]{\textit{AUGUSTO MATHIAS ADAMS}}
\pagestyle{fancy} % sem definir esse comando, o cabeçalho personalizado não é exibido

\hypersetup{
	colorlinks=true,
	linkcolor=blue,
	filecolor=magenta,      
	urlcolor=blue,
	pdftitle={ENGENHARIA DOS RAIOS - TE981 - ONE MINUTE PAPER}
}
\renewcommand\theadalign{bc}
\renewcommand\theadfont{\bfseries}
\renewcommand\theadgape{\Gape[4pt]}
\renewcommand\cellgape{\Gape[4pt]}

%dashed line
\usepackage{booktabs, makecell}
\renewcommand\theadfont{\bfseries}
\renewcommand\theadgape{}
\usepackage{arydshln}
\setlength\dashlinedash{0.2pt}
\setlength\dashlinegap{1.5pt}
\setlength\arrayrulewidth{0.3pt}

% padrao 1.5 de espacamento entre linhas
\setstretch{1.5}
\makeatletter
\def\@maketitle{%
	\newpage
	\null
	\vskip 2em%
	\begin{center}%
		\let \footnote \thanks
		{\LARGE \@title \par}%
		\vskip 1.5em%
		{\large
			\lineskip .5em%
			\begin{tabular}[t]{c}%
				\@author
			\end{tabular}\par}%
		%\vskip 1em%
		%{\large \@date}%
	\end{center}%
	\par
	\vskip 1.5em}
\makeatother

\title{\normalsize{ENGENHARIA DOS RAIOS - TE981}\\ \huge{\textbf\textit{{AULA 3 - FORMAÇÃO DE NUVENS DE TEMPESTADE}}\\}}
\author{\small{AUGUSTO MATHIAS ADAMS}}
\setcounter{Maxaffil}{0}
\renewcommand\Affilfont{\itshape\small}

\begin{document}
	% Seleciona o idioma do documento
	\selectlanguage{brazil}
	
	% título
	\maketitle
	
	\section{Aprendizado da Aula}
	
	\begin{itemize}
		\item \textbf{\textit{Formação de nuvens de tempestade $\Rightarrow$ }} As nuvens de tempestade se formam quando uma parcela de ar quente e úmido sobe na atmosfera e se resfria, causando a condensação do vapor de água presente no ar. Existem vários fatores que podem causar essa elevação do ar, como os efeitos orográficos (quando o ar úmido é forçado a subir ao encontrar uma montanha, por exemplo), as correntes de advecção (quando o ar quente é transportado horizontalmente e se encontra com uma massa de ar frio), a convergência de ventos e a convecção (quando o Sol aquece a superfície terrestre e gera correntes ascendentes de ar quente). Quando essas parcelas de ar quente e úmido se elevam, formam as nuvens de tempestade, que podem resultar em raios, trovões, chuvas fortes e outros fenômenos meteorológicos intensos.
		\item \textbf{\textit{Conceitos base $\Rightarrow$ Processo de Formação das Nuvens}}
		\begin{itemize}
			\item \textbf{\textit{Processo: }}Formação da Nuvem
			\item \textbf{\textit{Combustível: }}parcela de ar quente e umido
			\item \textbf{\textit{Catalizador: }}advecção, efeitos orográficos, convergência de ventos, convecção.
		\end{itemize}
		\item \textbf{\textit{Tipos de Nuvem:}}
		\begin{itemize}
			\item  \textbf{\textit{Cirrus:}} são nuvens finas, brancas e com aparência fibrosa. Geralmente são encontradas em altitudes elevadas e indicam tempo bom, mas podem indicar mudanças no clima em algumas situações.
			
			\item \textbf{\textit{Cumulus:}} são nuvens brancas e fofas, com uma aparência de "algodão". Elas podem se formar em altitudes diferentes, mas geralmente indicam tempo bom.
			
			\item \textbf{\textit{Stratus:}} são nuvens cinzentas e uniformes, com uma aparência plana e baixa. Elas geralmente se formam em altitudes baixas e indicam tempo nublado ou chuvoso.
			
			\item \textbf{\textit{Nimbostratus:}} são nuvens densas e escuras que geralmente se formam em altitudes baixas e médias. Elas indicam chuva ou neve.
			
			\item \textbf{\textit{Cumulonimbus:}} são nuvens grandes e volumosas, com uma aparência em forma de bigorna. Elas se formam em altitudes elevadas e são frequentemente associadas a tempestades, raios e ventos fortes.
			
			\item \textbf{\textit{Stratocumulus:}} são nuvens baixas e espessas, com uma aparência de "rolos" ou "bolas". Elas geralmente indicam tempo nublado, mas podem se dissipar rapidamente.
			
			\item \textbf{\textit{Altocumulus:}} são nuvens brancas ou cinzentas que aparecem em camadas. Elas geralmente indicam tempo bom, mas também podem indicar mudanças no clima.
			
			\item \textbf{\textit{Cirrostratus:}} são nuvens finas e transparentes que parecem um véu branco. Elas geralmente indicam tempo bom, mas também podem indicar a chegada de uma frente fria.
			
			\item \textbf{\textit{Cirrocumulus:}} são nuvens pequenas e redondas que parecem bolinhas brancas. Elas geralmente indicam tempo bom, mas também podem indicar a chegada de uma frente fria.
		\end{itemize}
		\item \textbf{\textit{Estágios de Evolução de uma Nuvem de Tempestade $\Rightarrow$ }} A nuvem Cumulonimbus (Cb) é uma nuvem de tempestade que pode se formar a partir de um Cumulus quando há calor e umidade suficientes na atmosfera para alimentar sua formação. A evolução de uma nuvem Cb pode ser dividida em quatro estágios principais:
		\begin{itemize}
			\item \textbf{\textit{Estágio de Desenvolvimento:}} Neste estágio, a nuvem Cb é caracterizada por um grande volume de ar ascendente, que pode ser visto como uma torre em forma de cogumelo. A base da nuvem está a uma altitude relativamente baixa, e o topo da nuvem pode se estender a grandes altitudes. Neste estágio, a nuvem está ganhando energia e crescendo rapidamente.
			
			\item \textbf{\textit{Estágio de Maturidade}}: Neste estágio, a nuvem Cb atinge seu tamanho máximo e é caracterizada por uma grande área de precipitação. O ar ascendente continua a alimentar a nuvem, mas a área de precipitação começa a se espalhar para fora da nuvem. Neste estágio, a nuvem pode produzir trovões, relâmpagos e ventos fortes.
			
			\item \textbf{\textit{Estágio de Dissipação:}} Neste estágio, a nuvem Cb começa a perder sua energia e a se dissipar. A precipitação se torna menos intensa e a base da nuvem começa a se elevar. Neste estágio, ainda podem ocorrer trovões e ventos fortes, mas a intensidade geral da tempestade está diminuindo.
			
			\item \textbf{\textit{Estágio de Dissipação Completa:}} Neste estágio, a nuvem Cb se dissipou completamente e não há mais energia disponível para sustentá-la. A tempestade está completamente acabada, e a área afetada começa a se recuperar.
		\end{itemize}
	
	\end{itemize}
	\section{Temas impactantes, dúvidas e questionamentos}
	
	Até aqui tudo bem, o processo de desenvolvimento de uma nuvem de tempestade está claro e compreensível, mas tenho uma dúvida cruel: em um programa antigo do \textit{Discovery Channel}, chamado de \textit{Mortes Estranhas}, há um episódio que narra a morte de um vendedor de bíblias que morreu tostado por um raio em um dia de sol, na porta de um provável cliente (Agradou a Deus, Ele levou pra casa). Além dos raios laterais (que saem de uma nuvem de tempestade e caem em qualquer lugar), existe a possibilidade de o raio ter se formado em condições de tempo bom? Se sim, gostaria de conhecer o mecanismo.
	
\end{document}