%\documentclass[journal, onecolumn, letterpaper]{IEEEtran}
%\documentclass[journal,onecolumn]{IEEEtran}
% \documentclass[conference]{IEEEtran}
\documentclass[a4paper, 12pt, onecolumn,singlespacing]{article}

% The preceding line is only needed to identify funding in the first footnote. If that is unneeded, please comment it out.
\usepackage[level]{fmtcount} % equivalent to \usepackage{nth}
% \include{util}
\usepackage[portuguese, brazil, english]{babel}
\usepackage{multirow}
\usepackage{array} % for defining a new column type
\usepackage{varwidth} %for the varwidth minipage environment
\usepackage[super]{nth}
\usepackage{authblk}
\usepackage{cite}
\usepackage{amsmath,amssymb,amsfonts}
\usepackage{ulem}
\usepackage{graphicx}
% \usepackage{subfig}
\usepackage{textcomp}
\usepackage{xcolor}
\usepackage{mathptmx}
\usepackage[T1]{fontenc}
\usepackage{textcomp}
\usepackage{titlesec}
\usepackage{helvet}
\usepackage{gensymb}
\usepackage{setspace} % espacamento entre linhas
\usepackage{pgfplots}
\usepackage{tikz}
\usepackage{subcaption}
\usepackage{minted}
\usepackage[left=2cm, right=2cm, bottom=2cm, top=2cm]{geometry} 
\usepackage{makecell}
\usepackage{pdfpages}

\renewcommand\theadalign{bc}
\renewcommand\theadfont{\bfseries}
\renewcommand\theadgape{\Gape[4pt]}
\renewcommand\cellgape{\Gape[4pt]}

%dashed line
\usepackage{booktabs, makecell}
\renewcommand\theadfont{\bfseries}
\renewcommand\theadgape{}
\usepackage{arydshln}
\setlength\dashlinedash{0.2pt}
\setlength\dashlinegap{1.5pt}
\setlength\arrayrulewidth{0.3pt}

% padrao 1.5 de espacamento entre linhas
\setstretch{1.5}

\title{Atividade Remota (Video 03)}

\author[1]{Augusto Mathias Adams}
\affil[1]{augusto.adams@ufpr.br}
\setcounter{Maxaffil}{0}
\renewcommand\Affilfont{\itshape\small}

\begin{document}
	% Seleciona o idioma do documento
	\selectlanguage{brazil}
	
	% título
	\maketitle
	
	\section{Aprendizado da Aula}
	
	Atividade remota sobre o \textbf{\textit{Vídeo 3 - Raios de Fogo☇Documentário FULL HD }}.
	
	Sugestão do Professor: levar ao menos uma dúvida para discussão em sala de aula.
	
	\section{Dúvidas e Questionamentos}
	
	\begin{itemize}
		\item Este documentário é bem mais sério do que os anteriores. Enquanto nos 2 primeiros vídeos havia sensacionalismo, neste a necessidade de causar polêmica é bem reduzida.
		
		\item Um erro, talvez de tradução: elétrons positivos e negativos. O documentário diz que há a formação de elétrons positivos dentro da nuvem e, por serem mais leves, estes sao transportados para a parte de cima da nuvem. Uma outra idéia: por serem em menor numero, o campo elétrico gerado pelas cargas negativas é suficiente para empurrá-las para cima, certo?
		
		\item Uma nuvem por si não é capaz de acumular tanta energia e produzir uma centelha: questionamento interessante e pertinente.
		
		\item primeiro video a falar do \textit{Runaway Breakdown}, inclusive é bem extenso a descrição do fenômeno a partir da detecção de raios \textit{gamma} $\Rightarrow$ talvez a explicação para a aparente falta de capacidade de gerar raios das nuvens.
		
		\item Eventos Luminosos Transientes têm potência o bastante para interferir nas comunicações e equipamentos de navegação $\Rightarrow$ por serem eventos muito rápidos, devem produzir uma grande gama de radiação de alta frequência, interferindo nas comunicações. Realidade linda porém infeliz.
		
		\item Todos os vídeos falam de vítimas $\Rightarrow$ embora seja pertinente, não é de nosso escopo estudar sequelas, portanto, não tecerei nenhum comentário sobre as vítimas de raios.
		
	\end{itemize}
	
\end{document}