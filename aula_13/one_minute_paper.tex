%\documentclass[journal, onecolumn, letterpaper]{IEEEtran}
%\documentclass[journal,onecolumn]{IEEEtran}
% \documentclass[conference]{IEEEtran}
\documentclass[a4paper, 12pt, onecolumn,singlespacing]{article}

% The preceding line is only needed to identify funding in the first footnote. If that is unneeded, please comment it out.
\usepackage[level]{fmtcount} % equivalent to \usepackage{nth}
% \include{util}
\usepackage[portuguese, brazil, english]{babel}
\usepackage{multirow}
\usepackage{array} % for defining a new column type
\usepackage{varwidth} %for the varwidth minipage environment
\usepackage[super]{nth}
\usepackage{authblk}
\usepackage{cite}
\usepackage{amsmath,amssymb,amsfonts}
\usepackage{ulem}
\usepackage{graphicx}
% \usepackage{subfig}
\usepackage{textcomp}
\usepackage{xcolor}
\usepackage{mathptmx}
\usepackage[T1]{fontenc}
\usepackage{textcomp}
\usepackage{titlesec}
\usepackage{helvet}
\usepackage{gensymb}
\usepackage{setspace} % espacamento entre linhas
\usepackage{pgfplots}
\usepackage{tikz}
\usepackage{subcaption}
\usepackage{minted}
\usepackage[left=2cm, right=2cm, bottom=2cm, top=2cm]{geometry} 
\usepackage{makecell}
\usepackage{pdfpages}


\usepackage{hyperref}
\usepackage{fancyhdr}
\renewcommand{\headrulewidth}{1pt}
\renewcommand{\footrulewidth}{0.5pt}
\fancyhf{} % limpa os cabecalhos e rodapés
\fancyhead[C]{\textit{CURSO DE ENGENHARIA DOS RAIOS - TE981} } % define o cabeçalho personalizado
\fancyfoot[C]{\textit{AUGUSTO MATHIAS ADAMS}}
\pagestyle{fancy} % sem definir esse comando, o cabeçalho personalizado não é exibido

\hypersetup{
	colorlinks=true,
	linkcolor=blue,
	filecolor=magenta,      
	urlcolor=blue,
	pdftitle={ENGENHARIA DOS RAIOS - TE981 - ONE MINUTE PAPER}
}
\renewcommand\theadalign{bc}
\renewcommand\theadfont{\bfseries}
\renewcommand\theadgape{\Gape[4pt]}
\renewcommand\cellgape{\Gape[4pt]}

%dashed line
\usepackage{booktabs, makecell}
\renewcommand\theadfont{\bfseries}
\renewcommand\theadgape{}
\usepackage{arydshln}
\setlength\dashlinedash{0.2pt}
\setlength\dashlinegap{1.5pt}
\setlength\arrayrulewidth{0.3pt}

% padrao 1.5 de espacamento entre linhas
\setstretch{1.5}
\makeatletter
\def\@maketitle{%
	\newpage
	\null
	\vskip 2em%
	\begin{center}%
		\let \footnote \thanks
		{\LARGE \@title \par}%
		\vskip 1.5em%
		{\large
			\lineskip .5em%
			\begin{tabular}[t]{c}%
				\@author
			\end{tabular}\par}%
		%\vskip 1em%
		%{\large \@date}%
	\end{center}%
	\par
	\vskip 1.5em}
\makeatother

\title{\normalsize{ENGENHARIA DOS RAIOS - TE981}\\ \huge{\textbf\textit{{AULA 13 - FÍSICA DAS DESCARGAS ATMOSFÉRICAS}}\\}}
\author{\small{AUGUSTO MATHIAS ADAMS}}
\setcounter{Maxaffil}{0}
\renewcommand\Affilfont{\itshape\small}

\begin{document}
	% Seleciona o idioma do documento
	\selectlanguage{brazil}
	
	% título
	\maketitle
	
	\section{Aprendizado da Aula}
	
	\paragraph{Contextualização} A física das descargas atmosféricas é um campo de estudo que se dedica a compreender os processos físicos envolvidos nas descargas elétricas que ocorrem na atmosfera, como os raios.
	
	As descargas atmosféricas são fenômenos naturais de alta energia que envolvem a transferência de cargas elétricas entre nuvens, entre nuvens e a Terra, ou dentro de uma única nuvem. Elas ocorrem devido a uma diferença de potencial elétrico significativa entre duas regiões na atmosfera, que pode ser causada por processos de eletrificação das nuvens, movimento vertical de partículas carregadas, entre outros fatores.
	
	Existem vários tipos de descargas atmosféricas, sendo os raios nuvem-terra (cloud-to-ground) os mais conhecidos e estudados. Esses raios são caracterizados por uma corrente elétrica intensa que percorre um caminho condutor entre uma nuvem carregada eletricamente e a superfície da Terra. Essa corrente é composta por pulsos de alta velocidade chamados de líderes, que são descargas elétricas ascendentes e descendentes que procuram estabelecer um caminho condutor através do ar ionizado.
	
	A formação de um raio envolve uma série complexa de processos físicos, incluindo a ionização do ar, a formação de líderes e o desenvolvimento de um canal condutor para a corrente elétrica. Durante o processo, ocorrem colisões entre partículas eletricamente carregadas, a geração de campos elétricos intensos, a criação de plasma e o aquecimento do ar, resultando em emissões de luz (relâmpagos) e sons (trovões).
	
	A física das descargas atmosféricas é estudada utilizando uma combinação de observações em campo, experimentos laboratoriais e modelagem computacional. Os pesquisadores procuram entender os mecanismos fundamentais que governam a ocorrência, a propagação e os efeitos das descargas atmosféricas, a fim de melhorar a previsão de tempestades e desenvolver medidas de proteção contra raios.
	
	\paragraph{Modelo do Canal de Descargas}
	
	Ao considerar o Modelo de Fonte-Carga para descrever a propagação de um Leader de relâmpago, fazemos a consideração de que a propagação é unidirecional e unipolar. Nesse modelo, a alteração do campo elétrico total no solo, produzido pelo Leader ao longo de um canal (L), pode ser descrita pela seguinte fórmula:
	
	\begin{equation}
		\Delta E = -\frac{\lambda_q}{2 \pi \epsilon} \left[\frac{1}{D} - \frac{1 + H_T^2}{\sqrt{H_T^2+D^2}}\right]
	\end{equation}

	Onde $D = \sqrt{\left(x - x_i\right)^2 + \left(y - y_i\right)^2}$. Se considerarmos o processo do líder escalonado e da descarga de retorno, a carga elétrica final será reduzida a um ponto onde a variação de carga é $ \Delta Q = \lambda qH_T$. Isso ocorre porque a descarga de retorno neutraliza a carga elétrica do líder escalonado (Stepped Leader). Nesse caso, o campo elétrico pode ser definido da seguinte forma:
	
	\begin{equation}
		\Delta E = -\frac{2 \Delta Q H_T}{4 \pi \epsilon \sqrt{\left(H_T^2 + D^2\right)^3}}
	\end{equation}
	
	\paragraph{Modelo Leader Bi-direcional} No modelo do Leader Bi-direcional, consideramos uma propagação bidirecional e bipolar do \textit{Stepped Leade}r, onde são admitidos dois canais de propagação a partir de um ponto inicial localizado em $H_T$, com orientações e polaridades opostas. Nesse caso, a distribuição de carga $\lambda_q$ varia linearmente em relação ao comprimento do canal do líder escalonado.
	
	Por assumir a mesma velocidade para os líderes positivos e negativos, a variação total do campo elétrico em solo, produzido pelo canal do líder, pode ser calculada da seguinte forma:
	
	\begin{equation}
		\Delta E = \frac{K}{2 \pi \epsilon} \left[\frac{H_T}{\sqrt{4H_T^2 + D^2}}\right] + \frac{K}{2 \pi \epsilon} ln \left[\frac{2H_T + \sqrt{4H_T^2 + D^2}}{D}\right]
	\end{equation}

	sendo $\lambda_q = -KH_t$. A variação final do campo elétrico no solo após este
	processo pode ser calculado pela expressão:
	
	\begin{equation}
		\Delta E = \frac{\Delta Q}{4 \pi \epsilon} \left[\frac{-2 H_T}{\sqrt{4H^2 + D^2}} + ln \left(\frac{2H_T + \sqrt{4H_T^2+ D^2}}{D}\right)\right]
	\end{equation}

	onde $\Delta Q = K H_T^2$ e demais variáveis no S.I.
	
	\section{Temas Impactantes, dúvidas e questionamentos}
	
	Nada que os livros e artigos não possam sanar.
	
\end{document}