%\documentclass[journal, onecolumn, letterpaper]{IEEEtran}
%\documentclass[journal,onecolumn]{IEEEtran}
% \documentclass[conference]{IEEEtran}
\documentclass[a4paper, 12pt, onecolumn,singlespacing]{article}

% The preceding line is only needed to identify funding in the first footnote. If that is unneeded, please comment it out.
\usepackage[level]{fmtcount} % equivalent to \usepackage{nth}
% \include{util}
\usepackage[portuguese, brazil, english]{babel}
\usepackage{multirow}
\usepackage{array} % for defining a new column type
\usepackage{varwidth} %for the varwidth minipage environment
\usepackage[super]{nth}
\usepackage{authblk}
\usepackage{cite}
\usepackage{amsmath,amssymb,amsfonts}
\usepackage{ulem}
\usepackage{graphicx}
% \usepackage{subfig}
\usepackage{textcomp}
\usepackage{xcolor}
\usepackage{mathptmx}
\usepackage[T1]{fontenc}
\usepackage{textcomp}
\usepackage{titlesec}
\usepackage{helvet}
\usepackage{gensymb}
\usepackage{setspace} % espacamento entre linhas
\usepackage{pgfplots}
\usepackage{tikz}
\usepackage{subcaption}
\usepackage{minted}
\usepackage[left=2cm, right=2cm, bottom=2cm, top=2cm]{geometry} 
\usepackage{makecell}
\usepackage{pdfpages}

\usepackage{hyperref}
\usepackage{fancyhdr}
\renewcommand{\headrulewidth}{1pt}
\renewcommand{\footrulewidth}{0.5pt}
\fancyhf{} % limpa os cabecalhos e rodapés
\fancyhead[C]{\textit{CURSO DE ENGENHARIA DOS RAIOS - TE981} } % define o cabeçalho personalizado
\fancyfoot[C]{\textit{AUGUSTO MATHIAS ADAMS}}
\pagestyle{fancy} % sem definir esse comando, o cabeçalho personalizado não é exibido

\hypersetup{
	colorlinks=true,
	linkcolor=blue,
	filecolor=magenta,      
	urlcolor=blue,
	pdftitle={ENGENHARIA DOS RAIOS - TE981 - ATIVIDADE REMOTA 5}
}

\renewcommand\theadalign{bc}
\renewcommand\theadfont{\bfseries}
\renewcommand\theadgape{\Gape[4pt]}
\renewcommand\cellgape{\Gape[4pt]}

%dashed line
\usepackage{booktabs, makecell}
\renewcommand\theadfont{\bfseries}
\renewcommand\theadgape{}
\usepackage{arydshln}
\setlength\dashlinedash{0.2pt}
\setlength\dashlinegap{1.5pt}
\setlength\arrayrulewidth{0.3pt}

% padrao 1.5 de espacamento entre linhas
\setstretch{1.5}
\makeatletter
\def\@maketitle{%
	\newpage
	\null
	\vskip 2em%
	\begin{center}%
		\let \footnote \thanks
		{\LARGE \@title \par}%
		\vskip 1.5em%
		{\large
			\lineskip .5em%
			\begin{tabular}[t]{c}%
				\@author
			\end{tabular}\par}%
		%\vskip 1em%
		%{\large \@date}%
	\end{center}%
	\par
	\vskip 1.5em}
\makeatother

\title{\normalsize{ENGENHARIA DOS RAIOS - TE981}\\ \huge{\textbf\textit{{ATIVIDADE REMOTA - VÍDEO 5}}\\}}
\author{\small{AUGUSTO MATHIAS ADAMS}}
\setcounter{Maxaffil}{0}
\renewcommand\Affilfont{\itshape\small}

\begin{document}
	% Seleciona o idioma do documento
	\selectlanguage{brazil}
	
	% título
	\maketitle

	\section{Aprendizado da Aula}
	
	Atividade remota sobre o \textbf{\textit{Vídeo 5 - EM Webinar Descargas atmosféricas - A física dos raios.}}
	
	\textit{Sinopse: } Em 20 anos de pesquisa, com base em estudos feitos com câmeras de alta velocidade e sensores de campo elétrico, o pesquisador do INPE Marcelo Saba tem produzido um conhecimento riquíssimo sobre a física dos raios. Neste \textit{webinar}, o especialista vai falar sobre os tipos de raios ― intranuvem, descendentes, ascendentes, positivos, negativos e bipolares ―, as condições em que acontecem e as principais características de cada um: números de descargas que contêm, duração, pico de corrente, correntes de longa duração e os perigos que oferecem. O \textit{webinar} também contará com a participação do especialista em aterramento e proteção contra raios e surtos João Gilberto Cunha, que trará uma visão sobre a aplicação prática dos conceitos explanados e as implicações sobre a normalização da área.
	
	Palestrantes : 
	\begin{itemize}
		\item 	\textbf{\textit{Marcelo Saba: }}  Graduado em Física pela USP (1985), mestre (1992) e doutor (1997) em Geofísica Espacial pelo INPE - Instituto Nacional de Pesquisas Espaciais, onde é pesquisador titular; coordena projetos de pesquisa no Brasil e participa de projetos no exterior; iniciou seu estudo sistemático de relâmpagos com câmeras de alta velocidade em 2003, o que resultou no	maior banco de vídeos de raios filmados em alta velocidade no mundo e em importantes contribuições publicadas, por exemplo nas revistas “Atmospheric Research”, “Journal of Geophyical Research” e “Geophysical Research Letters”. Sua pesquisa atual é sobre raios e a interação com sistemas de proteção contra descargas atmosféricas, raios ascendentes e altas energias associadas a descargas atmosféricas. É membro da International Commission on Atmospheric Electricity, do Cigrè - International Council on Large Electric Systems, da Sociedade Brasileira de Física e da American Association of Physics Teachers.
		\item \textbf{\textit{João Gilberto Cunha: }} Engenheiro eletricista (UFU), mestre em engenharia (ITA), diretor da empresa Mi Omega, integrante das comissões das normas ABNT NBR 5419 (Proteção contra descargas atmosféricas), NBR 5410 (instalações de baixa tensão), NBR 14039 (instalações de média tensão) e 16690 (requisitos de projeto de arranjos fotovoltaicos), diretor técnico da
		
		\item \textbf{\textit{Abracopel: }} Associação Brasileira de Conscientização para os Perigos da Eletricidade, escreveu diversos livros e inúmeros artigos técnicos, inclusive nas revistas EM e FotoVolt.
	\end{itemize}
	
	
	Sugestão do Professor: levar ao menos uma dúvida para discussão em sala de aula.
	
	\section{Dúvidas e Questionamentos}
	
	\begin{itemize}
		\item A descrição dos raios é praticamente a mesma que vimos durante o semestre
		\item Se os comentários são feitos por engenheiros, me parece que saíram do curso técnico e não da universidade.
		\item Excetuando a palestra do Marcelo Saba, o resto é quase tudo bobagem.
	\end{itemize}
	
\end{document}