%\documentclass[journal, onecolumn, letterpaper]{IEEEtran}
%\documentclass[journal,onecolumn]{IEEEtran}
% \documentclass[conference]{IEEEtran}
\documentclass[a4paper, 12pt, onecolumn,singlespacing]{article}

% The preceding line is only needed to identify funding in the first footnote. If that is unneeded, please comment it out.
\usepackage[level]{fmtcount} % equivalent to \usepackage{nth}
% \include{util}
\usepackage[portuguese, brazil, english]{babel}
\usepackage{multirow}
\usepackage{array} % for defining a new column type
\usepackage{varwidth} %for the varwidth minipage environment
\usepackage[super]{nth}
\usepackage{authblk}
\usepackage{cite}
\usepackage{amsmath,amssymb,amsfonts}
\usepackage{ulem}
\usepackage{graphicx}
% \usepackage{subfig}
\usepackage{textcomp}
\usepackage{xcolor}
\usepackage{mathptmx}
\usepackage[T1]{fontenc}
\usepackage{textcomp}
\usepackage{titlesec}
\usepackage{helvet}
\usepackage{gensymb}
\usepackage{setspace} % espacamento entre linhas
\usepackage{pgfplots}
\usepackage{tikz}
\usepackage{subcaption}
\usepackage{minted}
\usepackage[left=2cm, right=2cm, bottom=2cm, top=2cm]{geometry} 
\usepackage{makecell}
\usepackage{pdfpages}

\usepackage{hyperref}
\usepackage{fancyhdr}
\renewcommand{\headrulewidth}{1pt}
\renewcommand{\footrulewidth}{0.5pt}
\fancyhf{} % limpa os cabecalhos e rodapés
\fancyhead[C]{\textit{CURSO DE ENGENHARIA DOS RAIOS - TE981} } % define o cabeçalho personalizado
\fancyfoot[C]{\textit{AUGUSTO MATHIAS ADAMS}}
\pagestyle{fancy} % sem definir esse comando, o cabeçalho personalizado não é exibido

\hypersetup{
	colorlinks=true,
	linkcolor=blue,
	filecolor=magenta,      
	urlcolor=blue,
	pdftitle={ENGENHARIA DOS RAIOS - TE981 - ONE MINUTE PAPER}
}
\renewcommand\theadalign{bc}
\renewcommand\theadfont{\bfseries}
\renewcommand\theadgape{\Gape[4pt]}
\renewcommand\cellgape{\Gape[4pt]}

%dashed line
\usepackage{booktabs, makecell}
\renewcommand\theadfont{\bfseries}
\renewcommand\theadgape{}
\usepackage{arydshln}
\setlength\dashlinedash{0.2pt}
\setlength\dashlinegap{1.5pt}
\setlength\arrayrulewidth{0.3pt}

% padrao 1.5 de espacamento entre linhas
\setstretch{1.5}
\makeatletter
\def\@maketitle{%
	\newpage
	\null
	\vskip 2em%
	\begin{center}%
		\let \footnote \thanks
		{\LARGE \@title \par}%
		\vskip 1.5em%
		{\large
			\lineskip .5em%
			\begin{tabular}[t]{c}%
				\@author
			\end{tabular}\par}%
		%\vskip 1em%
		%{\large \@date}%
	\end{center}%
	\par
	\vskip 1.5em}
\makeatother

\title{\normalsize{ENGENHARIA DOS RAIOS - TE981}\\ \huge{\textbf\textit{{AULA 1 - ENGENHARIA DAS DESCARGAS ATMOSFÉRICAS}}\\}}
\author{\small{AUGUSTO MATHIAS ADAMS}}
\setcounter{Maxaffil}{0}
\renewcommand\Affilfont{\itshape\small}

\begin{document}
	% Seleciona o idioma do documento
	\selectlanguage{brazil}
	
	% título
	\maketitle
	
	\section{Aprendizado da Aula}
	
	\begin{itemize}
		\item \textbf{\textit{Ementa do Curso $\Rightarrow$ }} aprender de A a Z sobre o fenômeno das Descargas atmosféricas.
		\subitem \textbf{\textit{Roadmap $\Rightarrow$}} Estudar os fenômenos na ionosfera,  estudar os fenômenos que ocorrem dentro da nuvem, estudar os fenômenos e processos que acontecem entre a nuvem e o solo, estudar os fenômenos que acontecem quando a descarga chega no solo.
		\subitem \textbf{\textit{Palavras-Chave $\Rightarrow$ }} Fenômenos da Ionosfera, Processos que dão origem a relâmpagos, Circuito Elétrico Global e Local, Stepped Leader, Dart Leader, Return Stroke, Parâmetros de Relâmpago, Equipamentos de Monitoramento de Tempestades, Impactos no Sistema Elétrico, Normas Regulamentadoras, Gestão de Risco contra raios.
		\item \textbf{\textit{Importância do Estudo $\Rightarrow$ }} Descargas atmosféricas são fenômenos com impactos econômicos, ambientais e sociais. Alguns fatos rápidos apresentados durante a aula:
		\subitem \textbf{\textit{Mortes por Raios $\Rightarrow$}}  Os raios podem ser extremamente perigosos e até mesmo mortais para seres humanos. De acordo com a Organização Mundial da Saúde (OMS), estima-se que cerca de 2.000 pessoas são mortas por raios a cada ano em todo o mundo. No entanto, este número pode ser subestimado, uma vez que muitas mortes causadas por raios são erroneamente atribuídas a outras causas. Nota: a maioria das mortes causadas por raios (Cerca de 80\%) seriam evitadas com conhecimentos básicos sobre o fenômeno das descargas atmosféricas.
		\subitem \textbf{\textit{Incêndios Causados Por Raios $\Rightarrow$}} Os raios são uma das principais causas de incêndios naturais em todo o mundo. Quando um raio atinge a superfície terrestre, ele pode aquecer instantaneamente o ar e causar uma explosão, que pode incendiar a vegetação circundante. Esses incêndios causados por raios são conhecidos como incêndios de origem elétrica. Os incêndios de origem elétrica podem se espalhar rapidamente, especialmente em áreas com clima seco e ventoso, e podem ser muito difíceis de controlar. Eles também podem ser extremamente perigosos, ameaçando vidas humanas, animais e propriedades.
		\subitem \textbf{\textit{Impactos Econômicos $\Rightarrow$}} As descargas atmosféricas podem ter impactos significativos na economia, especialmente nos setores de energia, infraestrutura e agricultura. Aqui estão alguns exemplos:
		\begin{itemize}
			\item \textbf{\textit{Setor de energia:}} As descargas atmosféricas podem causar interrupções no fornecimento de energia elétrica, danificando linhas de transmissão, transformadores e outros equipamentos. Isso pode levar a cortes de energia prolongados e significativos prejuízos financeiros para as empresas de energia e para os consumidores.
			
			\item \textbf{\textit{Infraestrutura:}} Descargas atmosféricas também podem danificar edifícios, pontes, torres de comunicação e outras infraestruturas críticas. Os custos de reparo e substituição dessas estruturas podem ser altos e afetar a economia local.
			
			\item \textbf{\textit{Agricultura:}} Descargas atmosféricas podem danificar plantações e colheitas, afetando a produção agrícola e os preços dos alimentos. Além disso, os incêndios causados por raios podem destruir terras agrícolas e pastagens, causando danos econômicos significativos.
		\end{itemize}

		
		Além disso, os custos de seguro podem aumentar em áreas de alto risco de raios, como resultado do aumento dos custos de reparação e substituição de equipamentos danificados. As empresas também podem ser afetadas pelos custos de paralisação das operações e pela perda de produtividade dos funcionários durante interrupções causadas por descargas atmosféricas. Portanto, é importante tomar medidas preventivas para minimizar os impactos econômicos das descargas atmosféricas.
		\item \textbf{\textit{Curiosidades e Fatos Rápidos $\Rightarrow$}} alguns fatos rápidos discutidos em sala e aula e outros retirados da internet.
		
		\subitem \textbf{\textit{Maneiras pelas quais um raio pode matar $\Rightarrow$ }}	Um raio pode ser letal de várias maneiras. Aqui estão algumas maneiras pelas quais um raio pode matar:
	
		\begin{itemize}
			\item \textbf{\textit{Parada cardíaca:}} A descarga elétrica do raio pode afetar o sistema cardiovascular do corpo, levando a uma parada cardíaca. Isso ocorre quando o coração para de bater e é responsável pela maioria das mortes por raios.
			
			\item \textbf{\textit{Queimaduras:}} O raio pode causar queimaduras na pele e nos órgãos internos, o que pode ser fatal, dependendo da gravidade da queimadura.
			
			\item \textbf{\textit{Lesões cerebrais:}} O raio pode afetar o sistema nervoso central, incluindo o cérebro, causando lesões cerebrais graves que podem ser fatais.
			
			\item \textbf{\textit{Danos aos órgãos internos:}} A descarga elétrica do raio pode afetar os órgãos internos, como o fígado, os rins e os pulmões, causando danos graves que podem ser fatais.
			
			\item \textbf{\textit{ Explosão do corpo:}} Em casos raros, o raio pode causar uma explosão do corpo devido à rápida expansão do ar dentro do corpo, o que pode ser fatal.
		\end{itemize}

		\subitem \textbf{\textit{Mecanismos Decorrentes das Descargas Atmosféricas que levam à morte $\Rightarrow$}} Os mecanismos mais comuns que levam à morte humana e animal são:
		
		\begin{itemize}
			\item \textbf{\textit{``Ground Current'': }}A corrente de solo é um fenômeno elétrico que pode ocorrer durante uma descarga atmosférica. Durante uma tempestade com raios, a descarga elétrica do raio que atinge o solo cria uma corrente elétrica que se propaga em média de forma radial com centro no ponto de impacto. Isso ocorre porque o solo é um condutor elétrico, embora com resistência elétrica significativa. A corrente de solo é uma corrente elétrica de baixa frequência e alta amplitude que se propaga pelo solo a partir do ponto de contato do raio com o solo. A corrente de solo é particularmente perigosa  pois pode causar uma diferença de potencial (tensão de passo) entre dois pontos no solo, resultando em uma corrente elétrica que afeta as pessoas, animais e equipamentos próximos. A corrente de solo causa um risco de choque elétrico, especialmente em áreas onde há muita umidade ou com solo úmido, o que torna o solo um melhor condutor elétrico.
			A corrente de solo é responsável por cerca de 50 a 55\% das mortes causadas por raios.
			\item \textbf{\textit{``Side Flash'': }}Side flash é um fenômeno perigoso que pode ocorrer durante uma descarga atmosférica. Isso acontece quando um raio atinge um objeto alto, como uma árvore, poste, torre ou prédio, e a corrente elétrica do raio segue o caminho do objeto em direção ao solo. Se houver alguém próximo ao objeto, a corrente elétrica pode saltar do objeto para a pessoa, causando uma descarga elétrica perigosa.O side flash pode ser perigoso, pois a corrente elétrica pode atingir uma pessoa sem que ela esteja diretamente no caminho da descarga elétrica do raio. Por exemplo, se uma pessoa estiver embaixo de uma árvore durante uma tempestade com raios e um raio atingir a árvore, a corrente elétrica pode saltar da árvore para a pessoa, mesmo que ela não esteja tocando a árvore diretamente. O Side Flash é responsável por 30 a 35\% das mortes causadas por raios.
			
			\item \textbf{\textit{``Upward Leader'': }} Durante uma tempestade, a carga elétrica se acumula na nuvem e na superfície terrestre, criando um campo elétrico intenso entre elas. Esse campo elétrico pode ionizar o ar e formar uma espécie de caminho condutor para a descarga elétrica. O processo de formação da descarga elétrica começa com o aparecimento de pequenos canais condutores de ar, chamados de líderes ascendentes, que se propagam a partir de objetos pontiagudos no solo, como árvores, antenas, edifícios, folhas de capim e até mesmo as pessoas. Esses líderes ascendentes contêm cargas elétricas que se movem em direção à nuvem, criando um caminho condutor para a descarga elétrica. Os líderes ascendentes são responsáveis por 10 a 15\% das mortes causadas por raios.
			
			\item \textbf{\textit{Incidência Direta: }}As mortes por incidência direta de raios ocorrem quando uma pessoa é atingida diretamente por um raio. Esse tipo de acidente é raro, mas pode ser fatal. De acordo com a Organização Mundial da Saúde (OMS), estima-se que cerca de 3 a 5\% das pessoas atingidas por raios morrem devido à incidência direta. No entanto, é importante lembrar que muitas mortes causadas por raios são atribuídas a outras causas, já que os sintomas podem ser semelhantes aos de outras condições médicas. Os efeitos da incidência direta de raios no corpo humano podem variar, mas geralmente incluem queimaduras graves, lesões musculares e danos aos órgãos internos, como coração e pulmões. O raio pode causar parada cardíaca, parada respiratória e outras condições que podem levar à morte.
			
			\item \textbf{\textit{``Contact Voltage'': }} A tensão de contato é uma forma de risco elétrico que ocorre quando uma pessoa toca diretamente em um objeto eletricamente energizado. Essa forma de tensão elétrica pode ser fatal em determinadas circunstâncias, cais como eletrização por descarga atmosférica.	A tensão de contato pode ser causada por uma variedade de situações, como tocar em um fio elétrico exposto, em um aparelho elétrico com defeito ou em uma superfície metálica eletricamente carregada, entre outras. A intensidade da tensão de contato depende da voltagem da fonte de energia elétrica, da resistência elétrica do corpo humano e da duração do contato. A tensão de contato é responsável por 3 a 5\% das mortes causadas por raios.
			
		\end{itemize}
		
		\subitem \textbf{\textit{Guia Rápido de Sobrevivência $\Rightarrow$}} Existem algumas medidas que podem ser tomadas para aumentar as chances de sobrevivência durante uma tempestade com raios:
		
		\begin{itemize}
			\item \textbf{\textit{Procure abrigo em um local seguro:}} A melhor maneira de evitar ser atingido por um raio é encontrar um local seguro. Evite prédios sem proteção contra descargas atmosféricas e qualquer estrutura alta ou metálica, como torres de energia eólica. Também evite árvores e arbustos.
			\item \textbf{\textit{Evite áreas abertas}}: Se você estiver ao ar livre, evite áreas abertas e expostas, como campos abertos, praias e campos de golfe. Procure abrigo em uma área protegida, como um carro ou uma caverna.
			
			\item \textbf{\textit{Agache-se e cubra a cabeça:}} Se não houver abrigo disponível, agache-se e cubra a cabeça com as mãos. Mantenha os pés juntos e não toque em superfícies metálicas.
			
			\item \textbf{\textit{Mantenha-se afastado de objetos condutores:}} Durante uma tempestade com raios, evite contato com objetos condutores de eletricidade, como cercas, linhas de energia elétrica, encanamentos de metal e equipamentos elétricos.
			
			\item \textbf{\textit{Espere a tempestade passar:}} Espere pelo menos 30 minutos depois de ouvir o último trovão antes de sair do abrigo. Lembre-se de que as tempestades podem se mover rapidamente e é importante esperar até que a tempestade esteja bem distante antes de sair de um local seguro.
		\end{itemize}		
		
		A melhor maneira de sobreviver a uma descarga atmosférica é evitar situações de risco, como áreas abertas e objetos condutores, e procurar abrigo em um prédio seguro ou em um carro. Se você estiver preso ao ar livre, agache-se e cubra a cabeça com as mãos, mantendo os pés juntos e evitando superfícies metálicas.
		
	\end{itemize}

	\section{Temas impactantes, dúvidas e questionamentos}
		\begin{itemize}
			\item Uma constatação triste é a de que muitas mortes por raios seriam evitadas se conhecimentos básicos sobre o fenômenos da eletricidade atmosférica fossem difundidos. Em uma situação ideal, 80\% das mortes seriam evitadas seguindo procedimentos simples de proteção. Mas o que esperar de um povo cujo Q.I médio é 86? No Brasil, por mais que se difunda, por mais procedimentos que se façam, ainda existe o poder discricionário do matuto sabe-tudo que não manja nada. Realidade infeliz.
			\item À primeira vista, os raios são fenômenos habituais e até simples. É tudo corrente elétrica certo? sou engenheiro e sei do que se trata.... Só que não: um raio é um processo complexo que vai desde a ionosfera até o chão e não existe nada que possa ser subentendido ou menosprezado.
			\item Porque (quase) tudo na natureza que é bonito também é perigoso? Um \textit{blue jet} deve ser lindo, um raio é magnífico, porém altamente perigosos.
		\end{itemize}
	\section{Mitos e Verdades Sobre Descargas Atmosféricas}
	
	\begin{itemize}
		\item \textbf{\textit{ Raios nunca atingem o mesmo lugar duas vezes. $\Rightarrow$}}
		\textit{Mito:} Na verdade, raios podem atingir o mesmo lugar várias vezes, especialmente se o local é alto e condutivo, como uma torre de transmissão ou um arranha-céu.
		
		\item 	\textbf{\textit{Áreas montanhosas são mais seguras durante tempestades com raios $\Rightarrow$}}.
		\textit{Mito:}  Áreas montanhosas e picos de montanhas são na verdade mais perigosos durante tempestades com raios, pois a altitude pode aumentar a possibilidade de uma pessoa ser atingida.
		
		\item \textbf{\textit{É seguro ficar dentro de um carro durante uma tempestade com raios. $\Rightarrow$}}
		\textit{Verdade:} Ficar dentro de um carro é uma das opções mais seguras durante uma tempestade com raios, pois o carro age como uma gaiola de Faraday, protegendo as pessoas dentro dele.
		
		\item \textbf{\textit{É seguro se abrigar embaixo de uma árvore durante uma tempestade com raios. $\Rightarrow$}}
		\textit{Mito:} Na verdade, é perigoso se abrigar embaixo de uma árvore durante uma tempestade com raios, pois os raios tendem a atingir as árvores mais altas, e a eletricidade pode passar para uma pessoa que esteja embaixo dela.
		
		\item \textbf{\textit{O raio só pode atingir pessoas que estão fora durante uma tempestade $\Rightarrow$}} \textit{Mito:} Infelizmente, isso não é verdade. Pessoas dentro de casa, prédios e outros locais fechados ainda podem ser atingidas por raios, especialmente se estiverem em contato com objetos condutores de eletricidade, como telefones com fio, encanamentos de metal e equipamentos eletrônicos.
		
		\item \textbf{\textit{É seguro tomar banho ou usar água corrente durante uma tempestade com raios. $\Rightarrow$}}
		\textit{Mito: } Não é seguro tomar banho ou usar água corrente durante uma tempestade com raios, pois a eletricidade pode passar através da água e chegar até uma pessoa.
		
		\item \textbf{\textit{O arco-íris após uma tempestade com raios é um sinal de que a tempestade acabou e é seguro sair $\Rightarrow$}}
		\textit{Mito:} Embora o arco-íris possa ser um belo espetáculo após uma tempestade com raios, isso não significa que a tempestade acabou e que é seguro sair. As descargas atmosféricas ainda podem ocorrer mesmo após o fim da chuva.
		
		\item \textbf{\textit{Um para-raios é 100\% eficaz em evitar que raios atinjam uma estrutura $\Rightarrow$}}
		\textit{Mito:} Um para-raios pode reduzir significativamente a probabilidade de um raio atingir uma estrutura, mas não é uma garantia absoluta. É importante que os para-raios sejam instalados corretamente e mantenham um bom estado de conservação.
		
		\item \textbf{\textit{A distância do trovão pode ser calculada contando o número de segundos entre o relâmpago e o trovão $\Rightarrow$}}	\textit{Verdade: }Contar o número de segundos entre o relâmpago e o trovão pode dar uma estimativa grosseira da distância do trovão, mas não é uma medida precisa. A distância real pode variar dependendo de muitos fatores, como a direção do trovão e a temperatura e umidade do ar.
		
		\item \textbf{\textit{A posição do corpo durante uma descarga atmosférica pode afetar a gravidade da lesão $\Rightarrow$}}
		\textit{Mito: }A posição do corpo não afeta a gravidade da lesão durante uma descarga atmosférica. Qualquer parte do corpo que esteja em contato com um objeto condutivo de eletricidade pode ser afetada pela eletricidade.
		
		\item \textbf{\textit{É seguro filmar ou fotografar tempestades com raios ao ar livre $\Rightarrow$}}
		\textit{Mito: }É altamente desaconselhável filmar ou fotografar tempestades com raios ao ar livre, pois isso aumenta significativamente o risco de ser atingido por um raio. É melhor observar a tempestade de dentro de casa ou de um local seguro.
		
		\item \textbf{\textit{Se você tocar uma vítima de incidência de descarga atmosférica, será eletrocutado $\Rightarrow$}} \textit{Mito:} o corpo humano não armazena eletricidade.
		\item \textbf{\textit{Você pode sentir uma descarga atmosférica antes de ser atingido $\Rightarrow$}} \textit{Mito:} Algumas pessoas relatam sentir uma sensação de formigamento ou choque elétrico antes de serem atingidas por um raio, mas isso não é comum e não é uma garantia de que você será atingido. É importante procurar abrigo imediatamente e evitar ficar em áreas abertas durante tempestades com raios.
	\end{itemize}

	\section{Normas}
		
		\subsection{Norma ABNT NBR 5419}
		
		A ABNT NBR 5419 é uma norma brasileira que estabelece as diretrizes e requisitos para o projeto, instalação e manutenção de sistemas de proteção contra descargas atmosféricas (SPDA). A norma aborda temas como a análise de risco, dimensionamento dos componentes do SPDA e procedimentos de inspeção e manutenção. O objetivo da ABNT NBR 5419 é garantir a segurança de pessoas e patrimônio em caso de descargas atmosféricas, prevenindo danos e acidentes. A norma é atualizada periodicamente para se adequar a novas tecnologias e práticas de segurança.
		A norma ABNT NBR 5419 é dividida em 5 partes:
		
		\begin{itemize}
			\item \textbf{\textit{Parte 1 - Princípios gerais $\Rightarrow$ }}a parte 1 da norma ABNT NBR 5419 estabelece os requisitos gerais para o projeto, instalação e manutenção de sistemas de proteção contra descargas atmosféricas (SPDA). Esta parte também fornece orientações para a avaliação de riscos envolvidos e a seleção de medidas de proteção adequadas, considerando as características do local, a função da estrutura e as consequências de uma descarga atmosférica. Além disso, são abordados temas como a classificação das estruturas e o dimensionamento do sistema de proteção, considerando os parâmetros elétricos, geométricos e ambientais. Esta parte da norma é essencial para garantir a eficácia e segurança dos SPDA instalados.
			\item \textbf{\textit{Parte 2 - Gerenciamento de Risco $\Rightarrow$ }}a Parte 2 da ABNT NBR 5419 fornece informações específicas para a avaliação do risco devido aos efeitos indiretos de raios, como sobretensões transitórias e correntes de terra, em estruturas e sistemas elétricos, eletrônicos e de telecomunicações. A norma apresenta os requisitos para o projeto, instalação e manutenção de medidas de proteção contra sobretensões transitórias e correntes de terra induzidas por raios em estruturas e sistemas elétricos, eletrônicos e de telecomunicações. A Parte 2 da norma inclui também informações sobre a seleção e instalação de dispositivos de proteção contra sobretensões transitórias, a medição da resistência de terra, a avaliação da compatibilidade eletromagnética, a análise de risco, a classificação dos locais em termos de risco e as especificações para o projeto e instalação de sistemas de aterramento.
			\item \textbf{\textit{Parte 3 - Danos físicos a estruturas e perigos à vida animal $\Rightarrow$ }}a parte 3 da ABNT NBR 5419, intitulada "Danos físicos a estruturas e perigos à vida animal", estabelece diretrizes para avaliar os efeitos das descargas atmosféricas em estruturas e na vida animal, incluindo edificações, linhas de transmissão, antenas e torres de telecomunicação, e animais em pastos, estábulos ou cercados. A norma descreve procedimentos para avaliar a probabilidade de danos e as consequências de uma descarga atmosférica, incluindo o risco de incêndios e explosões, e a necessidade de medidas de proteção para minimizar os riscos. A parte 3 da norma também apresenta orientações para a análise do risco de danos a equipamentos eletrônicos e de telecomunicações e para a avaliação dos efeitos da eletricidade de contato e da indução eletromagnética em pessoas e animais próximos a estruturas afetadas por descargas atmosféricas.
			\item \textbf{\textit{Parte 4 - Sistemas elétricos e eletrônicos internos $\Rightarrow$ }}a parte 4 da NBR 5419 é intitulada "Sistemas Elétricos e Eletrônicos Internos na Estrutura", e estabelece as medidas de proteção contra surtos para os sistemas elétricos e eletrônicos internos em uma estrutura, tais como equipamentos de telecomunicações, sistemas de automação, entre outros. A norma define requisitos para o projeto, instalação e manutenção desses sistemas, incluindo a seleção e instalação de dispositivos de proteção contra surtos. Além disso, a parte 4 da NBR 5419 também aborda a necessidade de um sistema de aterramento eficiente para garantir a eficácia da proteção contra surtos. Essa parte da norma é importante para garantir a segurança e a integridade dos sistemas elétricos e eletrônicos internos de uma estrutura, evitando danos e perda de funcionamento de equipamentos sensíveis a sobretensões.
			\item \textbf{\textit{Parte 5 - Outros serviços e estruturas $\Rightarrow$ }}a parte 5 da ABNT NBR 5419 estabelece as diretrizes para o sistema de proteção contra descargas atmosféricas. Essa parte especifica os requisitos para o projeto, instalação e manutenção de sistemas de proteção contra descargas atmosféricas em estruturas que contenham equipamentos eletroeletrônicos sensíveis, como centrais de telecomunicações, sistemas de informática e outros similares.	A parte 5 da NBR 5419 traz recomendações para a proteção de equipamentos eletroeletrônicos sensíveis contra sobretensões transitórias induzidas por descargas atmosféricas em suas linhas de alimentação de energia, de sinal e de dados. Ela aborda aspectos como a seleção e instalação de dispositivos de proteção contra sobretensões, a coordenação desses dispositivos, as características elétricas dos equipamentos a serem protegidos e os procedimentos de inspeção e manutenção desses sistemas de proteção.
		\end{itemize}
		
		\subsection{Norma ABNT NBR 5410}
		
		A Norma Brasileira ABNT NBR 5410 é referente às instalações elétricas de baixa tensão em edificações, e tem como objetivo estabelecer as condições a que devem satisfazer as instalações elétricas de baixa tensão, a fim de garantir a segurança de pessoas e animais, o funcionamento adequado da instalação e a conservação dos bens. Ela define as características dos materiais e equipamentos a serem utilizados, as condições a que devem satisfazer as instalações elétricas, as disposições relativas à proteção e ao funcionamento da instalação, entre outras diretrizes importantes para a segurança e bom funcionamento das instalações elétricas.
		
		A NBR 5410 é dividida em partes e seções, sendo elas:
		
		\begin{itemize}
			\item \textbf{\textit{Parte 1 - Objetivo, campo de aplicação e definições gerais $\Rightarrow$ }}a parte 1 da NBR 5410 estabelece o objetivo, o campo de aplicação e as definições gerais para a norma. Seu objetivo é definir as medidas necessárias para garantir a segurança das pessoas e dos animais, o funcionamento adequado dos equipamentos e a conservação dos bens em instalações elétricas de baixa tensão, abrangendo as fases de projeto, execução, inspeção, manutenção e reforma. O campo de aplicação da norma inclui instalações elétricas de baixa tensão, com tensões de até 1.000V em corrente alternada (CA) ou 1.500V em corrente contínua (CC), em edificações residenciais, comerciais, públicas, industriais e rurais, bem como em áreas externas, como ruas, praças e estradas. As definições gerais estabelecem termos e conceitos que serão utilizados em todas as partes da norma, visando garantir uma linguagem uniforme e adequada. Alguns exemplos de definições abordadas na parte 1 incluem: aparelho de proteção contra surtos (DPS), circuito elétrico, corrente nominal, dispositivo de proteção contra sobrecorrente, entre outros.
			\item \textbf{\textit{Parte 2 - Condições gerais de projeto $\Rightarrow$ }}a parte 2 da NBR 5410 estabelece as condições gerais de projeto que devem ser consideradas para garantir a segurança das pessoas, dos animais e dos bens materiais contra os efeitos térmicos e não térmicos da eletricidade. Essa parte da norma estabelece os princípios básicos a serem seguidos em todas as instalações elétricas de baixa tensão e define os parâmetros de projeto, tais como:
			\begin{itemize}
				\item Seleção dos equipamentos elétricos de acordo com sua adequação ao uso pretendido e às condições ambientais e de instalação;
				\item Critérios para a seleção das características nominais dos componentes elétricos, tais como corrente nominal, tensão nominal, corrente de curto-circuito suportável, seletividade e coordenação entre dispositivos de proteção;
				\item Dimensionamento dos condutores elétricos e cálculo da queda de tensão;
				\item Proteção contra sobrecorrentes, curto-circuito, sobretensão e contato direto;
				\item Disposições para aterramento e equipotencialização;
				\item Isolação e proteção contra choques elétricos;
				\item Proteção contra os efeitos do calor e fogo;
				\item Identificação e sinalização das instalações elétricas.
			\end{itemize}
			
			
			Essa parte da norma também apresenta exemplos de aplicação dos critérios estabelecidos, para orientar o projetista na correta aplicação dos conceitos e parâmetros definidos.
			\item \textbf{\textit{Parte 3 - Proteção contra choques elétricos $\Rightarrow$ }}a parte 3 da NBR 5410 aborda os aspectos das instalações elétricas em relação às proteções contra choques elétricos. Ela fornece os requisitos de proteção para as pessoas, animais e bens, bem como para o próprio sistema elétrico. Alguns dos tópicos abordados nesta parte incluem:
			\begin{itemize}
				\item Proteção contra choques elétricos diretos e indiretos;
				\item Medidas de proteção para reduzir a probabilidade de falhas de isolamento;
				\item Seleção de dispositivos de proteção contra sobrecorrente;
				\item Proteção contra sobretensões;
				\item Proteção contra sobrecargas e curto-circuitos;
				\item Coordenação de isolamento;
				\item Verificação da eficácia da proteção.
			\end{itemize}
			.
			
			Essa parte da norma é essencial para garantir a segurança das pessoas e a proteção dos equipamentos elétricos.
			\item \textbf{\textit{Parte 4 - Proteção contra sobretensões e perturbações eletromagnéticas $\Rightarrow$ }}a parte 4 da NBR 5410 estabelece os requisitos para a seleção e instalação de dispositivos de proteção contra sobretensões transitórias, também conhecidos como DPS (Dispositivos de Proteção contra Surtos).
			
			Essa parte da norma apresenta orientações sobre as características e especificações técnicas que os DPS devem atender, bem como os critérios para sua seleção e instalação, de forma a garantir a proteção adequada dos equipamentos elétricos contra surtos de tensão decorrentes de descargas atmosféricas, manobras de rede e outras causas.
			
			Entre os tópicos abordados na Parte 4 da NBR 5410, destacam-se:
			\begin{itemize}
				\item Definições e classificação dos DPS;
				\item Características elétricas e mecânicas dos DPS;
				\item Critérios para seleção dos DPS, com base nas \item características do sistema elétrico e nos requisitos de proteção;
				\item Requisitos para a instalação dos DPS, incluindo sua localização, conexão e equipotencialização;
				\item Verificação da eficácia dos DPS instalados, por meio de ensaios e testes.
			\end{itemize}
			
			
			O cumprimento dos requisitos da Parte 4 da NBR 5410 é fundamental para garantir a proteção dos equipamentos elétricos e a segurança das pessoas que trabalham com eletricidade.
			
			\item \textbf{\textit{Parte 5 - Instalações elétricas em locais com condições especiais $\Rightarrow$ }}a parte 5 da NBR 5410 estabelece requisitos específicos para instalações elétricas em locais com condições especiais, tais como:
			
			\begin{itemize}
				\item \textbf{\textit{Locais com presença de poeira:}} a norma estabelece requisitos para a instalação de equipamentos e dispositivos elétricos em locais com presença de poeira combustível, de forma a prevenir a ocorrência de explosões ou incêndios.
				
				\item \textbf{\textit{Locais com presença de umidade:}} a norma estabelece requisitos para a instalação de equipamentos e dispositivos elétricos em locais úmidos ou com possibilidade de inundação, de forma a garantir a segurança das pessoas e a integridade dos equipamentos.
				
				\item \textbf{\textit{Locais com presença de gases ou vapores explosivos:}} a norma estabelece requisitos para a instalação de equipamentos e dispositivos elétricos em locais com presença de gases ou vapores inflamáveis, de forma a prevenir a ocorrência de explosões ou incêndios.
				
				\item \textbf{\textit{Locais com presença de líquidos inflamáveis:}} a norma estabelece requisitos para a instalação de equipamentos e dispositivos elétricos em locais com presença de líquidos inflamáveis, de forma a prevenir a ocorrência de incêndios.
				
				\item \textbf{\textit{Locais com presença de animais:}} a norma estabelece requisitos para a instalação de equipamentos e dispositivos elétricos em locais com presença de animais, de forma a prevenir a ocorrência de acidentes elétricos.
				
				\item \textbf{\textit{Locais com presença de pessoas com necessidades especiais:}} a norma estabelece requisitos para a instalação de equipamentos e dispositivos elétricos em locais com presença de pessoas com necessidades especiais, de forma a garantir a acessibilidade e a segurança dessas pessoas.
			\end{itemize}
			
			\item \textbf{\textit{Parte 6 - Verificação $\Rightarrow$ }}a parte 6 da NBR 5410 estabelece os procedimentos de verificação das instalações elétricas de baixa tensão, incluindo as inspeções visuais e os ensaios elétricos e funcionais.
			
			Esta parte da norma apresenta as diretrizes para a realização de inspeções visuais nas instalações elétricas, que devem ser realizadas antes da energização e após a conclusão das obras. Além disso, também são descritos os procedimentos de ensaios elétricos e funcionais que devem ser realizados em cada fase do processo de instalação elétrica.
			
			Os ensaios elétricos incluem medições de resistência de circuito de terra, verificação da continuidade dos condutores de proteção, testes de isolação, entre outros. Já os ensaios funcionais têm como objetivo verificar o funcionamento correto dos dispositivos de proteção, como disjuntores, fusíveis, DRs, entre outros.
			
			A Parte 6 da NBR 5410 também apresenta os critérios para aceitação e rejeição das instalações elétricas, com base nos resultados das inspeções visuais e dos ensaios elétricos e funcionais. Por fim, a norma estabelece as responsabilidades dos diferentes agentes envolvidos na verificação das instalações elétricas, como projetistas, instaladores e usuários.
			
			\item \textbf{\textit{Parte 7 - Disposições para segurança $\Rightarrow$ }}a parte 7 da NBR5410 aborda as disposições para a segurança das pessoas, animais e bens materiais envolvidos em instalações elétricas. Essas disposições incluem medidas de proteção contra choques elétricos, isolamento elétrico, proteção contra sobretensões e sobrecorrentes, proteção contra incêndios e explosões, entre outras.
			
			A Parte 7 estabelece os critérios de segurança a serem observados em todas as etapas de projeto, construção, instalação, operação, manutenção e inspeção de instalações elétricas. Ela também fornece orientações sobre a seleção de dispositivos de proteção, equipamentos de medição e outros dispositivos de segurança, bem como sobre a identificação de riscos e a adoção de medidas preventivas.
			
			A Parte 7 da NBR5410 é fundamental para garantir a segurança das pessoas e dos bens envolvidos em instalações elétricas e deve ser seguida rigorosamente por todos os profissionais envolvidos em atividades relacionadas à eletricidade.
			
			\item \textbf{\textit{Seções complementares $\Rightarrow$ }} Conjunto de normas e documentos técnicos que complementam a NBR 5410 e são referenciados ao longo do texto da norma. 
		\end{itemize}
		
		
		A NBR 5410 cita a NBR 5419, que é a norma que estabelece os requisitos para o sistema de proteção contra descargas atmosféricas (SPDA) em edificações e estruturas. A NBR 5410, por sua vez, estabelece as normas técnicas para instalações elétricas de baixa tensão em edificações. É comum que ambas as normas sejam utilizadas em conjunto para garantir a segurança e o bom funcionamento das instalações elétricas em edificações.	
		
		\subsection{Norma Regulamentadora NR-10}
		
		A NR-10 é uma Norma Regulamentadora do Ministério do Trabalho e Emprego (atualmente, Secretaria Especial de Previdência e Trabalho do Ministério da Economia) que estabelece as condições mínimas para garantir a segurança e saúde dos trabalhadores que atuam em instalações e serviços em eletricidade.
		
		A norma tem como objetivo estabelecer medidas de controle e prevenção de acidentes elétricos, através de requisitos para a segurança em projetos, execução, operação, manutenção, reforma e ampliação das instalações elétricas, além de prever a capacitação e treinamento dos trabalhadores que atuam nessas áreas.
		
		Entre os temas abordados pela NR-10, estão a análise de risco, a utilização de equipamentos de proteção coletiva e individual, a sinalização de segurança, a documentação técnica das instalações elétricas, as medidas de proteção contra incêndios e explosões, a investigação de acidentes, entre outros. A norma também estabelece as responsabilidades dos empregadores, empregados e prestadores de serviços quanto à segurança elétrica.
		
		A NR 10 e a NBR 5419 são duas normas que tratam de assuntos distintos, mas complementares no que se refere à segurança em instalações elétricas.
		
		A NR 10, estabelecida pelo Ministério do Trabalho e Emprego (MTE), trata da segurança em serviços em eletricidade. Ela estabelece as diretrizes e requisitos mínimos para a segurança dos trabalhadores que interagem direta ou indiretamente com instalações elétricas e serviços com eletricidade. A norma tem como objetivo garantir a segurança e a saúde dos trabalhadores que trabalham com eletricidade, prevenindo acidentes e minimizando os riscos associados a essas atividades.
		
		Já a NBR 5419, estabelecida pela Associação Brasileira de Normas Técnicas (ABNT), trata da proteção contra descargas atmosféricas. Ela estabelece os requisitos mínimos para a proteção de pessoas e patrimônios contra os efeitos diretos e indiretos das descargas atmosféricas em edificações e estruturas.
		
		Apesar de terem focos diferentes, as duas normas são complementares, pois uma instalação elétrica pode estar exposta a descargas atmosféricas e, portanto, precisa ser protegida. Por isso, a NBR 5419 é frequentemente mencionada na NR 10 como uma norma complementar que deve ser seguida em instalações elétricas expostas a riscos de descargas atmosféricas.
		
	\section{Nomes aos Bois - Nomenclatura}
	
	\begin{itemize}
		\item \textbf{\textit{Raio ou descarga atmosférica (``Lightning'') $\Rightarrow$ }}refere-se a todos os elementos envolvidos no processo completo de descarga atmosférica, incluindo a formação do canal de descarga, a corrente de retorno, bem como as manifestações elétricas, visuais e sonoras associadas.
		
		\item \textbf{\textit{Relâmpago (``Lightning'') $\Rightarrow$ }} o efeito luminoso produzido pelo aquecimento do canal causado pelo fluxo de corrente de retorno.
		
		\item \textbf{\textit{Trovão (``Thunder'') $\Rightarrow$ }}o efeito sonoro resultante do deslocamento abrupto do ar próximo ao canal de descarga, que se expande devido ao aquecimento gerado pela corrente elétrica que passa pelo canal.
		
		\item \textbf{\textit{Descarga Elétrica (``Electric	discharge'') $\Rightarrow$ }}fenômeno físico que ocorre quando há uma transferência súbita de cargas elétricas entre dois corpos com potenciais elétricos diferentes. Esse processo é acompanhado por uma liberação intensa de energia em forma de luz, calor e ondas eletromagnéticas.
		
		\item \textbf{\textit{Descarga atmosférica plena (``Flash'') $\Rightarrow$ }} Corresponde ao conjunto de descargas de retorno \textit{(``Return Stroke'')} envolvidas após o fechamento do canal. Um \textit{flash} pode ser constituído de	uma ou mais descargas \textit{(``Strokes'')} pelo canal.
		
		\item \textbf{\textit{Canal precursor de descarga (Líder Escalonado ou ``Stepped Leader'') $\Rightarrow$ }}É o canal ionizado que evolui por passos, da ordem de 50 m, decorrente de
		sucessivos saltos no ar, com intervalos de 50 $\mu s$.
		
		\item \textbf{\textit{Canal descendente (``Downward Leader'') $\Rightarrow$ }}é o canal ionizado que se forma a partir da quebra do ar e se propaga em direção ao solo, partindo da nuvem em direção ao solo. Esse canal apresenta uma direção vertical, com ramificações e oscilações.
		
		\item \textbf{\textit{Canal Ascendente (``Upward Leader'') $\Rightarrow$ }}é o canal ionizado que se propaga no sentido ascendente a partir do solo, evoluindo em direção à nuvem. Também tem direção vertical, com tortuosidades e ramificações.
		
		\item \textbf{\textit{Canal de descarga (``Lightning Channel'') $\Rightarrow$ }}é um caminho ionizado através do qual a eletricidade flui durante uma descarga elétrica, como um raio. É criado quando um campo elétrico intenso ioniza o ar e cria um canal condutivo para que a carga elétrica possa se mover. O canal de descarga geralmente se estende do solo até a nuvem de tempestade, embora também possa se formar entre diferentes partes da mesma nuvem. O canal de descarga é rodeado por uma aura brilhante de luz chamada de corona, que é causada pela ionização do ar ao redor do canal. O núcleo do canal mede alguns centímetros e a aura do efeito corona mede alguns metros.
		
		\item \textbf{\textit{Corrente de retorno (``Return Current'') $\Rightarrow$ }}A corrente que flui pelo percurso ionizado após o fechamento do canal de descarga corresponde à corrente de retorno. É a corrente elétrica que retorna da nuvem para o solo, fechando o circuito elétrico da descarga atmosférica. A intensidade dessa corrente pode atingir valores muito elevados, da ordem de dezenas ou até centenas de milhares de amperes. O descarregamento dessa corrente pode gerar diversos efeitos, como luminosos, sonoros, térmicos e magnéticos.
		
		\item \textbf{\textit{Descarga de Retorno (``Stroke'' ou ``Return Stroke'') $\Rightarrow$ }}A descarga de retorno é um evento elétrico que ocorre durante uma descarga atmosférica. Ela está associada ao fluxo de corrente de retorno pelo canal de descarga após o fechamento. Um único flash de descarga atmosférica pode incluir várias descargas de retorno (strokes).
		
		\item \textbf{\textit{Descarga de Retorno Subsequente (``Subsequent Stroke'') $\Rightarrow$ }} descargas subsequentes à descarga de retorno.
		
		\item \textbf{\textit{Corrente de Recarregamento do Canal de Descarga (``Dart Leader Current'') $\Rightarrow$ }} A corrente de recarregamento do canal de descarga é uma corrente de baixo valor que corresponde ao deslocamento de cargas negativas da nuvem para o canal restante após uma descarga de retorno negativa anterior. Logo após o fluxo da corrente de retorno, o canal tende a se dissipar. No entanto, em alguns casos (entre 70 e 80\% das vezes), ocorrem processos disruptivos na parte superior do canal, conectando-o a outro centro de cargas negativas na nuvem.
		
		\item \textbf{\textit{Processo de Descarregamento do Canal de Descarga (``Dart Leader'') $\Rightarrow$ }} é um evento que ocorre durante a dissipação da corrente de retorno de uma descarga anterior. Nesse momento, a corrente de recarga flui pelo canal de descarga, criando um fenômeno semelhante a um "dardo" percorrendo o canal.
		
		\item \textbf{\textit{Processo de Conexão ou Fechamento do Canal de Descarga (``Attachment Process'') $\Rightarrow$ }}após a evolução dos canais ascendente e descendente, ocorre a conexão entre eles, dando origem ao canal de descarga. Nesse canal, ocorrerá o fluxo da corrente de retorno. Esse processo é considerado o estágio final da evolução do fenômeno da descarga elétrica atmosférica.
		
		\item \textbf{\textit{Canal Piloto de Descarga (``Streamer'') $\Rightarrow$ }}canal precursor estabelecido junto a um eletrodo ou na extremidade de um canal ionizado, o qual antecede a formação do salto da descarga elétrica plena.
		
		\item \textbf{\textit{Descarga Direta (``Direct Flash'') $\Rightarrow$ }}O evento associado à incidência direta de uma descarga sobre uma vítima, estrutura ou objeto.
		
		\item \textbf{\textit{Descarga indireta, lateral ou próxima (``Indirect Flash'' ou ``Close Strike'') $\Rightarrow$ }}Descarga indireta é um evento no qual a descarga elétrica não atinge diretamente uma pessoa, objeto ou estrutura, mas causa um efeito indireto. Por exemplo, a descarga pode atingir um objeto próximo, como uma árvore ou poste, e o efeito da corrente elétrica pode se propagar para um objeto ou pessoa próxima através de condução ou radiação. Esse tipo de descarga também é conhecido como descarga lateral ou próxima.
	\end{itemize}
	

	\section{Crenças sobre os Raios}
	
	\subsection{Mitologia Nórdica}
	
	A mitologia nórdica inclui várias histórias e crenças relacionadas a tempestades e descargas atmosféricas. Thor, o deus do trovão e da guerra, é uma das figuras mais conhecidas da mitologia nórdica associada a raios e tempestades.
	
	De acordo com as lendas, Thor empunhava um martelo mágico chamado Mjolnir, que era capaz de criar trovões e raios quando era jogado. Os antigos nórdicos acreditavam que a presença de trovões e relâmpagos era uma manifestação da ira de Thor contra os inimigos dos deuses.
	
	Em algumas histórias, os raios eram vistos como uma forma de comunicação entre os deuses e os mortais. Acredita-se que os raios também possam ser uma forma de purificação, com Thor usando seus poderes para limpar a terra dos males e influências negativas.
	
	Os antigos nórdicos também tinham rituais e práticas para se proteger de tempestades e raios. Por exemplo, eles acreditavam que a queima de fogueiras durante as tempestades poderia ajudar a afastar os raios e evitar danos.
	
	A mitologia nórdica ainda tem um lugar na cultura popular, especialmente em relação a Thor e seu papel como o deus do trovão e dos raios.
	
	\subsection{Mitologia Suméria}
	
	A mitologia suméria é uma das mais antigas do mundo, datando de cerca de 4000 a.C. Embora a mitologia suméria não tenha um deus específico do trovão ou dos raios, há várias referências a tempestades e descargas atmosféricas em suas histórias e lendas.
	
	Uma das figuras mais conhecidas da mitologia suméria é Enlil, o deus da tempestade, que controlava o vento, a chuva e os trovões. Ele era frequentemente retratado segurando uma arma com a qual ele poderia criar trovões e relâmpagos.
	
	Os antigos sumérios também acreditavam que as tempestades eram um sinal de descontentamento dos deuses. Eles acreditavam que, quando os deuses estavam descontentes com a humanidade, eles enviariam tempestades e outras manifestações climáticas para demonstrar sua ira.
	
	Além disso, havia práticas e rituais para se proteger de tempestades e raios na mitologia suméria. Por exemplo, acredita-se que a queima de incenso e oferendas aos deuses poderia ajudar a afastar tempestades e evitar danos causados por descargas atmosféricas.
	
	\subsection{Mitologia Babilônica}
	
	A mitologia babilônica, que se desenvolveu a partir da cultura suméria, também faz referência a tempestades e descargas atmosféricas em suas histórias e lendas.
	
	O deus babilônico Adad, também conhecido como Ishkur, era o deus da tempestade, da chuva e dos trovões. Ele era frequentemente retratado com um raio na mão e acreditava-se que tinha o poder de controlar as tempestades.
	
	Na mitologia babilônica, assim como na suméria, as tempestades eram frequentemente vistas como uma manifestação da ira divina. Acredita-se que, quando os deuses estavam insatisfeitos com a humanidade, eles poderiam enviar tempestades, chuvas torrenciais e raios para demonstrar sua ira.
	
	Os babilônios também desenvolveram rituais para se protegerem contra tempestades e raios. Eles acreditavam que oferecer sacrifícios aos deuses poderia ajudar a afastar as tempestades e evitar danos causados por descargas atmosféricas.
	
	\subsection{Mitologia Egípcia}
	
	A mitologia egípcia também tem várias referências a tempestades e descargas atmosféricas em suas histórias e crenças religiosas.
	
	Um dos deuses mais associados com as tempestades e raios na mitologia egípcia é Seth. Ele era frequentemente retratado segurando uma lança ou um raio, e acredita-se que ele tinha o poder de controlar o vento e as tempestades. Hórus, o deus falcão, também era às vezes associado com tempestades e raios, especialmente quando ele assumia sua forma como o deus da guerra.
	
	Assim como em outras culturas antigas, as tempestades e raios eram frequentemente vistos como um sinal da ira divina na mitologia egípcia. Por exemplo, a história do deus Rá, que era o deus sol, mostra que ele se tornou irado com a humanidade e enviou tempestades para punir aqueles que o haviam ofendido.
	
	Os antigos egípcios também desenvolveram amuletos e outras práticas para se protegerem contra tempestades e raios. Por exemplo, eles acreditavam que usar amuletos em forma de escaravelho podia ajudar a afastar as tempestades e evitar danos causados por descargas atmosféricas.
	
	\subsection{Mitologia Grega}
	
	A mitologia grega também tem várias referências a tempestades e descargas atmosféricas em suas histórias e crenças religiosas.
	
	Um dos deuses mais associados com as tempestades e raios na mitologia grega é Zeus, o deus do trovão e do raio. Ele é frequentemente retratado segurando um raio e é considerado o governante do céu e do clima. As tempestades e descargas atmosféricas eram frequentemente vistas como uma expressão do poder e da ira de Zeus.
	
	Acredita-se que o mito de Prometeu seja uma das histórias mais antigas na mitologia grega sobre como os humanos descobriram o fogo. Na história, Prometeu rouba o fogo dos deuses e o dá aos humanos, mas como punição, Zeus o acorrenta a uma rocha e ordena que uma águia coma seu fígado todos os dias. A imagem de um raio atingindo uma rocha e incendiando-a é frequentemente usada como símbolo desta história.
	
	Os antigos gregos acreditavam que Zeus podia controlar as tempestades e descargas atmosféricas, mas também acreditavam que as tempestades eram um resultado da luta entre os deuses e os titãs, ou uma resposta a comportamentos humanos ofensivos. Além disso, muitas práticas religiosas gregas envolviam oferecer sacrifícios aos deuses para acalmar a ira de Zeus e evitar as tempestades.
	
	\subsection{Mitologia Maia}
	
	A mitologia maia, assim como outras mitologias mesoamericanas, tem uma forte ligação com fenômenos atmosféricos, incluindo as descargas atmosféricas. Os maias acreditavam que os deuses controlavam o clima e o tempo, e as tempestades e descargas atmosféricas eram consideradas uma forma de comunicação divina.
	
	Um dos principais deuses maias associados às tempestades e aos raios era Chaac, o deus da chuva. Ele era frequentemente representado com uma faca de pedra e era responsável por trazer as chuvas necessárias para o cultivo das plantações. Os maias acreditavam que Chaac controlava os raios e as tempestades com sua faca, e que podia enviá-los como uma forma de punição ou bênção.
	
	Outra figura importante na mitologia maia relacionada às descargas atmosféricas é o deus K'awiil, que era frequentemente representado como um homem com um raio na mão. K'awiil era associado ao comércio, à prosperidade e à guerra, e os maias acreditavam que ele podia usar o raio para proteger ou destruir os humanos, dependendo de suas ações.
	
	Os maias também acreditavam que as tempestades e descargas atmosféricas eram uma forma de comunicação divina, e que os deuses podiam enviar mensagens ou avisos através desses eventos. Por exemplo, uma tempestade ou raio durante um ritual religioso era visto como um sinal de aprovação dos deuses.
	
	\subsection{Mitologia Asteca}
	
	Na mitologia asteca, os raios eram associados ao deus Tláloc, que era responsável pela chuva, relâmpagos e trovões. Tláloc era um dos principais deuses da mitologia asteca e era adorado como um deus da fertilidade e da agricultura.
	
	De acordo com a lenda, Tláloc usava seus raios para fertilizar a terra e trazer a chuva necessária para as colheitas. No entanto, os astecas também acreditavam que Tláloc poderia lançar raios como uma forma de punição para aqueles que o desrespeitavam ou desobedeciam suas ordens.
	
	Além disso, os astecas também acreditavam que as pessoas poderiam ser transformadas em raios depois de morrerem. Esses raios se juntavam ao poder de Tláloc e eram capazes de controlar a chuva e os trovões, ajudando a trazer a fertilidade para a terra e proteger as colheitas.
	
	\subsection{Mitologia Inca}
	
	Na mitologia inca, o deus dos raios era conhecido como Illapa, ou às vezes chamado de Catequil. Ele era um dos deuses mais importantes do panteão inca e era considerado responsável por trazer a chuva necessária para a agricultura. Illapa também era associado aos raios e trovões e considerado um deus temido e poderoso.
	
	Os incas acreditavam que Illapa usava seus raios para castigar aqueles que desobedeciam suas ordens ou eram desrespeitosos. Diz-se que, quando os incas foram conquistados pelos espanhóis, Illapa se retirou para as montanhas e ainda é venerado por alguns povos andinos como um espírito protetor.
	
	\subsection{Mitologia Aborígene}
	
	As culturas aborígenes da Austrália têm uma forte conexão com a natureza e acredita-se que os fenômenos atmosféricos, incluindo as descargas atmosféricas, são influenciados por forças espirituais. Embora haja muitas culturas aborígenes na Austrália, cada uma com suas próprias crenças e histórias, há algumas crenças comuns sobre os raios e tempestades.
	
	Para muitas culturas aborígenes, as descargas atmosféricas são consideradas uma manifestação da força espiritual conhecida como o "Caminho do Raio" ou o "Caminho do Trovão". Acredita-se que essa força espiritual seja capaz de se mover através da terra e do céu, e que possa ser convocada por aqueles que possuem um conhecimento especial.
	
	Os aborígenes também acreditam que os raios são uma manifestação da força vital da Terra, e que sua energia pode ser canalizada para propósitos específicos, como cura ou proteção. Os raios são frequentemente vistos como uma forma de comunicação entre os seres humanos e o mundo espiritual.
	
	Além disso, muitas culturas aborígenes acreditam que os raios são uma manifestação da ira dos espíritos ou ancestrais descontentes. Eles acreditam que os raios podem ser evocados como uma forma de punição ou para alertar as pessoas sobre comportamentos inadequados.
	
	Os aborígenes australianos também têm histórias e lendas sobre criaturas mitológicas que estão associadas aos raios e tempestades. Por exemplo, o povo Yolngu do norte da Austrália acredita na existência do Guirrimbirra, um espírito do raio que é conhecido por suas habilidades de cura e que é muitas vezes convocado durante cerimônias religiosas.
	
	A lenda aborígene dos "irmãos raios" é uma história que fala sobre a origem dos raios. Segundo essa lenda, havia dois irmãos que eram muito próximos e sempre estavam juntos. Um dia, eles se apaixonaram pela mesma mulher e começaram a brigar por ela.
	
	A luta entre os irmãos foi tão intensa que suas armas de pedra começaram a faiscar e a produzir sons estrondosos. Quando um dos irmãos atingiu o outro com sua arma, uma enorme descarga elétrica surgiu no céu, seguida de um trovão ensurdecedor.
	
	Os aborígenes acreditavam que essa luta entre os irmãos gerou os primeiros raios e trovões. Desde então, os irmãos raios continuaram a lutar, gerando novas descargas elétricas a cada golpe de suas armas.
	
	Para os aborígenes, os raios eram vistos como um sinal de poder e força, e a história dos irmãos raios simbolizava a luta pela vida e pelo amor. Essa lenda é um exemplo de como diferentes culturas e povos criaram mitos e histórias para explicar a origem dos fenômenos naturais que os rodeavam.
	
	\subsection{Mitologia Chinesa}
	
	Na mitologia chinesa, os raios são associados ao deus do trovão, Lei Gong. Ele é frequentemente retratado com um martelo ou um machado de batalha, que ele usa para criar trovões e raios.
	
	Lei Gong é considerado um deus protetor que mantém o equilíbrio entre o céu e a terra, e acredita-se que ele é responsável por trazer chuvas para fertilizar as colheitas. Ele também é um deus guerreiro que é chamado em tempos de guerra para proteger os soldados chineses.
	
	Na cultura chinesa, acredita-se que os raios têm poderes curativos e protetores. Os chineses costumam pendurar imagens do deus do trovão em suas casas e negócios para se protegerem de raios e trovões. Além disso, também é comum fazer ofertas e orações a Lei Gong em templos dedicados a ele.
	
	\subsection{Mitologia Eslava}
	
	Na mitologia eslava, o deus do trovão é Perun. Ele é retratado como um homem forte com cabelos longos e barba. Perun é considerado o deus mais poderoso do panteão eslavo e seu símbolo é um machado de batalha ou martelo.
	
	Perun é frequentemente retratado lançando raios e trovões do céu com seu martelo, e é creditado por trazer chuva e fertilidade à terra. Ele é também considerado um deus guerreiro, e acredita-se que tenha ajudado os eslavos a vencerem várias batalhas contra seus inimigos.
	
	Na mitologia eslava, acreditava-se que as tempestades com raios eram causadas por Perun lutando contra os espíritos malignos e demônios que habitavam o céu. As pessoas frequentemente faziam oferendas a Perun para apaziguar sua ira e garantir sua proteção contra os perigos das tempestades.
	
	\subsection{Mitologia Gaulesa}
	
	A mitologia dos gauleses, um povo celta que habitava a região que é hoje a França, é pouco conhecida devido à falta de registros escritos. No entanto, alguns historiadores acreditam que a deusa Taranis era a divindade associada aos raios e trovões pelos gauleses.
	
	De acordo com a crença, Taranis era uma deusa guerreira com um martelo ou machado na mão, que controlava o tempo, a chuva e as tempestades. Ela era frequentemente retrataao com um raio, simbolizando seu poder sobre as descargas atmosféricas. Taranis também era vista como uma deusa protetora dos guerreiros gauleses e era frequentemente invocado em batalhas.
	
	Embora não haja muitas informações sobre a mitologia dos gauleses, acredita-se que sua crença em Taranis e sua associação com raios e trovões tenham sido significativas para a cultura e a religião do povo celta.
	
	\subsection{Mitologias Africanas}
	
	A África é um continente com grande diversidade cultural e, consequentemente, com uma rica variedade de mitologias a respeito dos trovões e raios. Em muitas culturas africanas, esses fenômenos naturais são considerados sagrados e associados a divindades ou espíritos.
	
	Na mitologia iorubá, por exemplo, Shango é o deus do trovão e do fogo, e é frequentemente representado segurando um machado de duas lâminas e vestindo roupas vermelhas e brancas. Ele é considerado um dos orixás mais importantes e é associado a várias virtudes, como a justiça, a honestidade e a coragem.
	
	Já na mitologia bantu, que engloba várias culturas de países como Angola, Moçambique e África do Sul, o trovão é frequentemente associado ao espírito do ancestral. Diz-se que, quando um ancestral está irritado ou precisa enviar uma mensagem, ele faz isso através dos trovões e raios.
	
	Outras culturas africanas também possuem suas próprias mitologias a respeito dos trovões e raios, como os ashanti de Gana, os himba da Namíbia e os zulu da África do Sul. Em geral, essas mitologias destacam a importância desses fenômenos naturais na vida das pessoas e como eles podem ser interpretados de diferentes maneiras, dependendo da cultura e da crença de cada um.
	
	\subsection{Mitologia Indígena do Brasil}
	
	Diferentes povos indígenas brasileiros possuem mitos e lendas que explicam a origem dos raios. Na mitologia Tupi-Guarani, por exemplo, o deus trovão Tupã é o responsável pelos raios e trovões, que são seus instrumentos de poder. Acredita-se que Tupã lançava raios para punir aqueles que quebravam suas leis ou desrespeitavam a natureza.
	
	Já para os índios Xavante, os raios são resultado da disputa entre dois irmãos: Omo e Iri. Omo, que representa o raio, é o irmão mais novo e sempre tenta superar o irmão mais velho, Iri, que representa o trovão. Durante a luta entre os dois, Omo lança raios para atingir o irmão, que responde com trovões.
	
	Os índios Karajá acreditam que os raios são causados por uma cobra gigante, chamada de Bojigi, que vive no fundo dos rios. Quando a cobra sai da água, ela cospe raios que atingem as pessoas e as árvores. Para se proteger, os Karajá costumam fazer oferendas à cobra e evitar entrar nos rios durante tempestades.
	
	Essas são apenas algumas das muitas histórias e mitos indígenas que explicam a origem dos raios e trovões. Essas narrativas refletem a forte relação que os povos indígenas têm com a natureza e como eles buscam entender e respeitar os fenômenos naturais que os cercam.
	
	\subsection{Mitologia dos povos indígenas norte-americanos}
	
	Diferentes povos nativos da América do Norte têm mitologias que explicam a origem e o significado dos raios. Abaixo estão algumas dessas mitologias:
	\begin{itemize}
		\item Para os Navajos, os raios são os arcos das divindades do trovão, que disparam flechas para afastar os espíritos malignos e purificar o ar. O trovão é o som das flechas quebrando o ar, e a chuva que cai após a tempestade é a água que as flechas arrastam consigo.
		
		\item Para os índios da Tribo dos Pés Pretos, os raios são causados por uma serpente gigante que vive nas nuvens e protege as pessoas da fome e da doença. Quando a serpente se move, seus raios iluminam a terra e trazem a chuva.
		
		\item Na mitologia dos Índios Hopi, os raios são os braços do Deus da Chuva, que usa sua arma mágica para combater as forças do mal. A eletricidade dos raios é a manifestação do poder divino que protege a terra e seus habitantes.
		
		\item Para os índios da Tribo dos Sioux, os raios são causados por um pássaro gigante que vive no céu e bate suas asas para produzir trovões. O pássaro é visto como um mensageiro divino que traz bênçãos para o povo.
		
	\end{itemize}
	
	Essas são apenas algumas das muitas mitologias indígenas que existem sobre os raios na América do Norte. Cada tribo tem sua própria interpretação e suas próprias histórias para explicar esses fenômenos naturais.
	
	\subsection{Mitologia esquimó}
	
	Na mitologia da Groenlândia, os raios são associados a uma figura chamada Qasavaraq, uma entidade sobrenatural responsável pela criação de tempestades e pelo lançamento de raios. Acredita-se que ele habite as montanhas e os fiordes da Groenlândia e que seja capaz de lançar raios em pessoas e animais que o desrespeitem ou o irritem. Alguns contos populares descrevem Qasavaraq como um ser perigoso e vingativo que pode causar a morte de quem o enfrentar. Por essa razão, muitos groenlandeses acreditam que é preciso ter muito respeito e reverência por essa entidade e que é necessário realizar oferendas e rituais para acalmá-lo e evitar a sua ira.
	
	\subsection{Mitologia árabe}
	
	Na mitologia árabe, os raios eram considerados uma manifestação da vontade divina. Eles eram associados a várias divindades, como Shango, deus da tempestade e do trovão, que era adorado pelos iorubás da Nigéria e do Benin. Alguns árabes acreditavam que os raios eram disparados por um arco celestial, que era uma espécie de arco-íris feito de fogo e que se estendia do céu à terra. Outros acreditavam que os raios eram disparados por um pássaro mítico chamado Anqa, que era o símbolo da divindade suprema do Islã, Alá. Ainda há outras histórias e lendas sobre raios na mitologia árabe, variando de acordo com as regiões e as crenças locais.
	
	\subsection{Mitologia Indo-européia}
	
	A mitologia indo-europeia, que se originou em uma área que se estendia da Índia ao leste até a Europa Ocidental, tinha várias divindades relacionadas aos raios. Na religião hindu, por exemplo, o deus Indra era considerado o senhor dos raios e trovões, e era frequentemente retratado segurando um raio em sua mão. Na mitologia grega, o deus do trovão era Zeus, que usava raios como arma. Na mitologia romana, o equivalente a Zeus era Júpiter, também relacionado aos raios e trovões.
	
	A mitologia indo-europeia também tinha uma figura conhecida como Perkūnas, que era o deus do trovão nas religiões bálticas, incluindo a lituana e a letã. Ele era geralmente retratado como um guerreiro com um martelo, e era frequentemente associado à proteção contra doenças e desastres naturais.
	
	Em outras culturas indo-europeias, como a celta, os raios eram frequentemente associados a divindades femininas, como a deusa irlandesa Brigid, que era a deusa da poesia, da cura e do fogo, entre outras coisas. O deus nórdico Thor também era relacionado com os raios, sendo frequentemente representado segurando um martelo e viajando em uma carruagem puxada por cabras.
	
	\subsection{Mitologia anglo-saxâ}
	
	A mitologia saxã, também conhecida como mitologia anglo-saxã, é a coleção de crenças, lendas e tradições dos antigos povos germânicos que habitavam a Inglaterra na Idade Média. Embora não haja muitas referências específicas sobre raios na mitologia saxã, sabe-se que os povos germânicos em geral atribuíam grande importância a Thor, deus do trovão e dos raios.
	
	Na mitologia saxã, Thor era conhecido como Thunor ou Thunraz, e era visto como um deus poderoso e protetor, capaz de controlar os elementos e combater inimigos com seu martelo mágico, o Mjolnir. De acordo com algumas lendas, Thor era capaz de invocar raios e trovões com seu martelo, e era frequentemente associado a tempestades violentas e poderosas.
	
	Além de Thor, outros deuses e deusas da mitologia saxã também tinham poderes relacionados aos elementos e fenômenos naturais, incluindo a chuva, o vento e o mar. No entanto, não há muitas histórias ou lendas específicas sobre raios na mitologia saxã, e as referências a Thor e aos trovões e raios geralmente aparecem de forma mais genérica e simbólica.
	
	\subsection{Mitologia Ugarítica}
	
	A mitologia ugarítica é uma religião antiga que foi praticada no que é hoje a Síria e o Líbano durante o segundo milênio antes de Cristo. Embora a mitologia ugarítica não seja especificamente sobre raios, havia uma deusa chamada Shapash, que era a deusa do sol, justiça e verdade. Ela era frequentemente retratada com um raio em sua mão e era considerada uma deusa poderosa que podia proteger e iluminar a humanidade. Além disso, Baal, o deus da tempestade, também era frequentemente associado a raios e trovões, já que ele controlava o clima e era considerado responsável pelas tempestades.
	
	\subsection{Tradição Judaica}
	
	Na tradição judaica, há diversas referências a trovões e raios, que geralmente são vistos como manifestações da presença divina. Por exemplo, na Bíblia hebraica, no livro de Êxodo, há o relato da descida de Deus no Monte Sinai em meio a trovões, relâmpagos e uma espessa nuvem. Em outro trecho da Bíblia, no livro de Jó, o personagem central é descrito como ouvindo a voz de Deus nos trovões.
	
	De maneira geral, os trovões e raios são vistos na tradição judaica como um sinal da força e poder de Deus, mas também podem ser vistos como um castigo divino ou como um alerta para a necessidade de se arrepender dos pecados. Além disso, a crença em amuletos protetores contra raios é comum em algumas comunidades judaicas tradicionais.
	
	Na tradição judaica, raios são frequentemente associados à ira divina e ao poder de Deus. Em Êxodo 9:23-24, Deus envia um raio e granizo como um sinal de Sua ira contra o faraó do Egito, enquanto que em Jó 28:26, trovões e relâmpagos são descritos como parte da obra de Deus e um reflexo da Sua sabedoria. Na literatura rabínica, também há menção de raios como parte da punição divina para aqueles que desobedecem a Deus. Por exemplo, é dito que a cidade de Sodoma foi destruída por um raio como punição por sua maldade.
	
	O Talmud é uma compilação de escritos judaicos que inclui ensinamentos rabínicos e comentários sobre a Torá e outras tradições judaicas. Embora o Talmud faça menção a raios e trovões em alguns trechos, ele não aborda especificamente mitologia ou crenças religiosas relacionadas a esses fenômenos naturais. Em geral, o Talmud enfatiza a importância de temer a Deus e seguir seus mandamentos, mas não fornece informações detalhadas sobre interpretações mitológicas ou simbólicas de fenômenos naturais.
	
	\subsection{Raios na Bíblia}
	
	A Bíblia menciona raios em várias passagens, principalmente como um sinal da ira divina. Um exemplo é o livro de Êxodo, capítulo 19, onde descreve a aproximação de Deus no Monte Sinai com trovões e raios. Outra passagem conhecida é a de Jó 28:26, que diz: "Quando deu à chuva um peso e estabeleceu um caminho para o relâmpago dos trovões, então viu a sabedoria e a declarou; firmou-a e a esquadrinhou". Também há menções a raios em outras passagens, como no livro de Ezequiel e no Salmo 77. Em geral, a Bíblia usa a imagem de raios para demonstrar o poder divino e a manifestação da sua presença de maneira majestosa e impressionante.
	
	No Novo Testamento, o livro de Apocalipse descreve trovões e raios como parte do juízo final de Deus sobre o mundo (Apocalipse 8:5, 11:19, 16:18). O Evangelho de Mateus também menciona trovões e raios em conexão com a segunda vinda de Cristo (Mateus 24:27). Em geral, trovões e raios são usados na Bíblia como símbolos da presença, poder e julgamento de Deus.
	
	Os trovões são mencionados em vários salmos da Bíblia, geralmente como um sinal da presença poderosa e majestosa de Deus. Aqui estão algumas referências:
	\begin{itemize}
		\item Salmo 18:13: "O Senhor trovejou dos céus; o Altíssimo levantou a sua voz com granizo e carvões ardentes."
		\item Salmo 29:3: "A voz do Senhor ouve-se sobre as águas; o Deus da glória troveja; o Senhor está sobre as muitas águas."
		\item Salmo 77:18: "A voz do teu trovão estava no céu; os relâmpagos iluminaram o mundo; a terra tremeu e abalou-se."
		\item Salmo 81:7: "Na angústia me invocaste, e te livrei; respondi-te no lugar oculto dos trovões; provei-te nas águas de Meribá."
	\end{itemize}

	
	Em todos esses casos, os trovões são usados para enfatizar a grandeza, a soberania e a autoridade de Deus.
	
	No livro de Ezequiel, na Bíblia, há diversas menções a trovões, principalmente no capítulo 1, que descreve a visão da glória de Deus. No versículo 13, por exemplo, é dito que "no meio do fogo havia uma semelhança de quatro seres viventes, e esta era a sua aparência: tinham a semelhança de homem". No versículo 24, é descrito que "ouvindo eu o ruído das suas asas, como o ruído de muitas águas, como a voz do Onipotente, quando fala, como o ruído de um arraial; ouvi também o ruído das rodas junto a eles, como o ruído de grandes águas". Em outras partes do livro, como no capítulo 10, os trovões também são mencionados como sinais da presença divina e do julgamento divino sobre os pecados do povo.
	
	No Evangelho de Mateus, há uma referência a trovões e raios no momento da crucificação de Jesus. Segundo o relato, no momento em que Jesus morreu, houve um grande terremoto e o véu do templo rasgou-se em dois, do alto até em baixo. Além disso, o texto afirma que "o centurião e os que com ele guardavam Jesus, vendo o terremoto e as coisas que haviam sucedido, tiveram grande temor e disseram: Verdadeiramente este era Filho de Deus" (Mateus 27:54).	Embora o texto não mencione diretamente trovões e raios, a descrição do terremoto e da rasgação do véu do templo sugere um evento de grande magnitude e poder, que pode ser interpretado como um sinal divino. É possível que a imagem de trovões e raios esteja implícita nessa descrição, como uma metáfora para o impacto da morte de Jesus sobre seus seguidores e a sociedade em geral.
	
	No Apocalipse, livro bíblico que relata a visão profética de João sobre o fim dos tempos, os trovões são mencionados como uma das sete pragas que caem sobre a Terra. No capítulo 16, versículo 18, é dito: "E houve vozes, e trovões, e relâmpagos, e um grande terremoto, como nunca houve desde que há homens sobre a Terra, um terremoto tão grande, tão forte". Nesse contexto, os trovões representam a ira divina e a destruição que acompanham o fim dos tempos.
	
\end{document}