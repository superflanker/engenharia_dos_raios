%\documentclass[journal, onecolumn, letterpaper]{IEEEtran}
%\documentclass[journal,onecolumn]{IEEEtran}
% \documentclass[conference]{IEEEtran}
\documentclass[a4paper, 12pt, onecolumn,singlespacing]{article}

% The preceding line is only needed to identify funding in the first footnote. If that is unneeded, please comment it out.
\usepackage[level]{fmtcount} % equivalent to \usepackage{nth}
% \include{util}
\usepackage[portuguese, brazil, english]{babel}
\usepackage{multirow}
\usepackage{array} % for defining a new column type
\usepackage{varwidth} %for the varwidth minipage environment
\usepackage[super]{nth}
\usepackage{authblk}
\usepackage{cite}
\usepackage{amsmath,amssymb,amsfonts}
\usepackage{ulem}
\usepackage{graphicx}
% \usepackage{subfig}
\usepackage{textcomp}
\usepackage{xcolor}
\usepackage{mathptmx}
\usepackage[T1]{fontenc}
\usepackage{textcomp}
\usepackage{titlesec}
\usepackage{helvet}
\usepackage{gensymb}
\usepackage{setspace} % espacamento entre linhas
\usepackage{pgfplots}
\usepackage{tikz}
\usepackage{subcaption}
\usepackage{minted}
\usepackage[left=2cm, right=2cm, bottom=2cm, top=2cm]{geometry} 
\usepackage{makecell}
\usepackage{pdfpages}

\usepackage{hyperref}
\usepackage{fancyhdr}
\renewcommand{\headrulewidth}{1pt}
\renewcommand{\footrulewidth}{0.5pt}
\fancyhf{} % limpa os cabecalhos e rodapés
\fancyhead[C]{\textit{CURSO DE ENGENHARIA DOS RAIOS - TE981} } % define o cabeçalho personalizado
\fancyfoot[C]{\textit{AUGUSTO MATHIAS ADAMS}}
\pagestyle{fancy} % sem definir esse comando, o cabeçalho personalizado não é exibido

\hypersetup{
	colorlinks=true,
	linkcolor=blue,
	filecolor=magenta,      
	urlcolor=blue,
	pdftitle={ENGENHARIA DOS RAIOS - TE981 - ONE MINUTE PAPER}
}
\renewcommand\theadalign{bc}
\renewcommand\theadfont{\bfseries}
\renewcommand\theadgape{\Gape[4pt]}
\renewcommand\cellgape{\Gape[4pt]}

%dashed line
\usepackage{booktabs, makecell}
\renewcommand\theadfont{\bfseries}
\renewcommand\theadgape{}
\usepackage{arydshln}
\setlength\dashlinedash{0.2pt}
\setlength\dashlinegap{1.5pt}
\setlength\arrayrulewidth{0.3pt}

% padrao 1.5 de espacamento entre linhas
\setstretch{1.5}
\makeatletter
\def\@maketitle{%
	\newpage
	\null
	\vskip 2em%
	\begin{center}%
		\let \footnote \thanks
		{\LARGE \@title \par}%
		\vskip 1.5em%
		{\large
			\lineskip .5em%
			\begin{tabular}[t]{c}%
				\@author
			\end{tabular}\par}%
		%\vskip 1em%
		%{\large \@date}%
	\end{center}%
	\par
	\vskip 1.5em}
\makeatother

\title{\normalsize{ENGENHARIA DOS RAIOS - TE981}\\ \huge{\textbf\textit{{AULA 15 - ENERGIA DAS DESCARGAS ATMOSFÉRICAS}}\\}}
\author{\small{AUGUSTO MATHIAS ADAMS}}
\setcounter{Maxaffil}{0}
\renewcommand\Affilfont{\itshape\small}

\begin{document}
	% Seleciona o idioma do documento
	\selectlanguage{brazil}
	
	% título
	\maketitle
	
	\section{Aprendizado da Aula}
	
	\paragraph{Contextualização}
	
	A energia das descargas atmosféricas, como raios e raios, é um fenômeno natural de grande magnitude que ocorre durante tempestades elétricas. Essas descargas resultam em uma rápida liberação de energia elétrica na atmosfera, gerando fortes campos elétricos e magnéticos.
	
	A energia de uma descarga atmosférica é impressionante. Um único raio pode liberar energia na faixa de vários bilhões de joules. Essa energia é gerada através do movimento rápido de cargas elétricas dentro das nuvens e entre as nuvens e a superfície da Terra.
	
	A energia das descargas atmosféricas é capaz de aquecer o ar ao seu redor a temperaturas extremamente altas, resultando em um rápido aumento de temperatura e uma expansão explosiva do ar, o que gera o som característico do trovão. Além disso, a energia elétrica liberada pode causar danos significativos a estruturas, árvores, equipamentos eletrônicos e até mesmo representar riscos para seres humanos e animais.
	
	Devido à sua alta energia e capacidade de causar danos, a compreensão e o estudo da energia das descargas atmosféricas são importantes para a segurança e a proteção contra raios. Pesquisas científicas e tecnológicas têm sido realizadas para desenvolver sistemas de detecção e prevenção de raios, bem como para entender melhor os mecanismos envolvidos nas descargas atmosféricas e sua interação com o ambiente.
	
	\paragraph{Energia das Descargas Atmosféricas}
	
	Um raio consiste em vários raios que ocorrem em sequência através de canais ionizados. Cada raio é separado por 40$\mu s$. Às vezes, há pausas entre os raios que podem ser observadas pelo olho humano, resultando em um efeito de piscar do raio. A duração média de um flash completo varia de 70$ms$ a 250$ms$ e geralmente é composto por cerca de quatro raios individuais separados por intervalos de microssegundos.
	
	A eletrificação predominante na atmosfera é conhecida como eletrificação "de tempo bom". Além disso, apenas uma pequena parte do planeta é coberta por tempestades a qualquer momento. No entanto, cálculos usando modelos globais de Circuitos Elétricos Atmosféricos sugerem que há uma diferença de potencial de 200.000 a 500.000 volts entre a ionosfera e a superfície da Terra, sendo a Terra o polo negativo. Isso corresponde a uma média de 100 a 140 volts por metro, resultando em uma corrente contínua de aproximadamente mil amperes fluindo da atmosfera para o solo durante condições de tempo bom. Uma maneira alternativa de medir o campo elétrico de um raio a uma determinada distância D é fornecida pela seguinte equação:
	
	\begin{equation}
		E = \frac{M}{D^3} + \frac{1}{cD^2}\frac{dM}{dt} + \frac{1}{c^2D}\frac{d^2M}{dt^2}
	\end{equation}
	
	Onde $E$ é o campo elétrico a uma distância $D$ de uma nuvem de tempestade, sendo significativamente maior em comparação com a dimensão da distribuição de carga dentro da nuvem. $M$ representa o momento total das cargas elétricas na nuvem no tempo $t - D/c$, sendo $c$ a velocidade da luz.
	
	As medidas do pico de corrente de um raio podem variar de 5.000A a 20.000A, e correntes ainda maiores, acima de 200.000A, já foram registradas. Um raio pode transferir $10^20$ elétrons em uma fração de segundo e alcançar um pico de corrente de até 10$kA$. Estima-se que um raio possua energia suficiente para acender 150 milhões de lâmpadas por segundo ou alimentar uma lâmpada de 100 watts por três meses.
	
	A temperatura do raio na atmosfera circundante pode atingir valores entre 8.000°C e 33.000°C, cerca de cinco vezes mais quente do que a superfície do Sol.
	
	De fato, a energia dos raios é o que causa o som do trovão. Quando ocorre uma descarga elétrica intensa e repentina durante uma tempestade, a alta temperatura resultante aquece o ar ao seu redor rapidamente. Esse rápido aumento de temperatura causa uma expansão explosiva do ar, gerando uma onda de choque que se propaga em todas as direções. Essa onda de choque é o que percebemos como o trovão.
	
	O som do trovão pode ser extremamente alto e estrondoso, podendo ser ouvido a grandes distâncias. A intensidade e a duração do som podem variar dependendo da distância entre o observador e o local onde o raio ocorreu. Quanto mais próximo o observador estiver do raio, mais alto e imediato será o som do trovão. À medida que a distância aumenta, o som pode se tornar mais abafado e prolongado.
	
	Assim, a energia liberada pelos raios durante as tempestades elétricas é o que gera o som característico do trovão, proporcionando uma experiência auditiva poderosa e muitas vezes impressionante durante os eventos atmosféricos.
		
	A duração da corrente de retorno determina o raio inicial do canal de um raio. Estudos indicam que a intensidade total do espectro de um raio está relacionada com o raio inicial do canal. A ionização e a energia térmica têm uma relação linear, e a energia de dissociação do canal está correlacionada com a energia térmica e de ionização por unidade de comprimento.
	
	A energia por unidade de comprimento é diretamente proporcional ao quadrado do raio inicial para diferentes tipos de descargas Nuvem-Solo. Isso significa que, à medida que o raio inicial do canal aumenta, a energia por unidade de comprimento também aumenta de forma proporcional ao quadrado desse raio. Isso ressalta a importância do tamanho inicial do canal de raio na determinação da energia envolvida na descarga.
	
	No modelo eletrodinâmico, os Dart Leaders e Return Strokes são descritos como Ondas Eletromagnéticas (OEM) que se propagam ao longo do canal de um raio. Essas ondas eletromagnéticas são geradas pela rápida descarga de eletricidade eletroestática durante um raio.
	
	O Dart Leader é uma onda eletromagnética que se propaga rapidamente da nuvem em direção ao solo. Ele cria um caminho ionizado pelo qual a descarga principal do raio (Return Stroke) seguirá. O Return Stroke é uma onda eletromagnética subsequente que se propaga ao longo do canal ionizado criado pelo Dart Leader, em direção à nuvem.
	
	
	A Energia dissipada por unidade de comprimento de um raio é igual a:
	
	\begin{equation}
		\frac{\epsilon}{L} = \lambda_q^2\left[\frac{1}{2} + log\left(\frac{E_{break}}{E_{cloud}}\right)\right]
	\end{equation}

	A energia do canal do raio pode ser dividida em três partes, quando a energia $\epsilon$ é rapidamente depositada dentro do canal pelo Return Stroke, antes da expansão radial do canal do raio. Essas partes são:

	\begin{itemize}
		\item \textbf{\textit{Energia térmica:}} A energia térmica é a energia associada ao aquecimento do canal do raio. Ela resulta do rápido aumento da temperatura devido à dissipação de energia durante a descarga. Essa energia é responsável por elevar a temperatura do canal, podendo atingir valores extremamente altos $\Rightarrow$ $\epsilon_{thermal} = \pi r_{init}^2 L(1+f) \frac{3}{2} n_{atomic} \kappa T$.
		
		\item \textbf{\textit{Energia de dissociação molecular:}} A energia de dissociação molecular está relacionada à quebra de ligações moleculares ao longo do canal do raio. Durante a descarga, a energia depositada é capaz de separar as moléculas presentes no canal, levando à formação de íons e radicais livres $\Rightarrow$ $\epsilon_{disso} = \pi r_{init}^2 L n_{molec} \tau_{disso}$.
		
		\item \textbf{\textit{Energia de ionização:}} A energia de ionização é a energia necessária para remover elétrons de átomos ou moléculas, resultando na formação de íons. No canal do raio, a alta energia depositada pelo Return Stroke é capaz de ionizar as partículas presentes, aumentando significativamente a concentração de íons ao longo do canal $\Rightarrow$ $\epsilon_{ioniz} = \pi r_{init}^2 L n_{atomic} f \tau_{ioniz}$.
		
	\end{itemize}

	Onde $r_{init}$ é o raio inicial em $m$, $L$ é o comprimento do canal em $m$, $T$ é a temperatura em $K$, $f$ é a média de ionização ($f = 0,97$) e $\kappa = 1,38 x 10^{-16}$ $erg/K$ é a constante de Boltzmann. $n_{atomic}$ é o número esperado de densidade atômica no canal do raio, $n_{molec}$ é o número médio esperado de moléculas no canal, $\tau_{disso}$ = $9,8$ $eV$ é a energia de dissociação de uma molécula de $N_2$, e $\tau_{ioniz}$ = $14,5$ $eV$ é a energia para a ionização atômica do Nitrogênio. As moléculas de $N_2$ no ar são facilmente dissociadas no canal do raio, então $n_{molec}$ = $0.5$ $n_{atomic}$.

	A densidade linear de carga no canal do raio, denotada por $\lambda_q$, está diretamente relacionada a essas diferentes formas de energia. Ela representa a quantidade de carga elétrica por unidade de comprimento ao longo do canal do raio. O campo elétrico de Breakdown ($E_{break}$) e o campo elétrico dentro da tempestade ($E_{cloud}$) são parâmetros que influenciam a dissipação de energia no canal do raio.
	
	Incluindo as 3 equações de energia na equação da Energia dissipada por unidade de comprimento de um raio, temos:
	
	\begin{equation}
		r_{init} = \lambda_q \left[\frac{1}{2} + log\left(\frac{E_{break}}{E_{cloud}}\right)\right]^{\frac{1}{2}} \times \left(\pi n_{atomic}\right)^{-\frac{1}{2}} \times \left[\left(1+f\right) \frac{3}{2} \kappa T + \frac{1}{2} \tau_{disso} + f\tau_{ioniz}\right]^{-\frac{1}{2}}
	\end{equation}

	Assim, o raio final do canal, após a expansão por aquecimento pode ser dada como:
	
	\begin{equation}
		r_{final} = r_{init} \left[\frac{2}{5} + \frac{6}{5} (1 + f) \frac{T}{T_{atmosfera}}\right]^{\frac{1}{2}}
	\end{equation}

	Onde $T_{atmosfera}$ é a a temperatura da atmosfera fora do canal de descarga.
	
	
	\section{Temas Impactantes, dúvidas e questionamentos}

	Nada que os livros e artigos não possam sanar.	
	
\end{document}