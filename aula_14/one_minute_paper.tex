%\documentclass[journal, onecolumn, letterpaper]{IEEEtran}
%\documentclass[journal,onecolumn]{IEEEtran}
% \documentclass[conference]{IEEEtran}
\documentclass[a4paper, 12pt, onecolumn,singlespacing]{article}

% The preceding line is only needed to identify funding in the first footnote. If that is unneeded, please comment it out.
\usepackage[level]{fmtcount} % equivalent to \usepackage{nth}
% \include{util}
\usepackage[portuguese, brazil, english]{babel}
\usepackage{multirow}
\usepackage{array} % for defining a new column type
\usepackage{varwidth} %for the varwidth minipage environment
\usepackage[super]{nth}
\usepackage{authblk}
\usepackage{cite}
\usepackage{amsmath,amssymb,amsfonts}
\usepackage{ulem}
\usepackage{graphicx}
% \usepackage{subfig}
\usepackage{textcomp}
\usepackage{xcolor}
\usepackage{mathptmx}
\usepackage[T1]{fontenc}
\usepackage{textcomp}
\usepackage{titlesec}
\usepackage{helvet}
\usepackage{gensymb}
\usepackage{setspace} % espacamento entre linhas
\usepackage{pgfplots}
\usepackage{tikz}
\usepackage{subcaption}
\usepackage{minted}
\usepackage[left=2cm, right=2cm, bottom=2cm, top=2cm]{geometry} 
\usepackage{makecell}
\usepackage{pdfpages}

\renewcommand\theadalign{bc}
\renewcommand\theadfont{\bfseries}
\renewcommand\theadgape{\Gape[4pt]}
\renewcommand\cellgape{\Gape[4pt]}

%dashed line
\usepackage{booktabs, makecell}
\renewcommand\theadfont{\bfseries}
\renewcommand\theadgape{}
\usepackage{arydshln}
\setlength\dashlinedash{0.2pt}
\setlength\dashlinegap{1.5pt}
\setlength\arrayrulewidth{0.3pt}

% padrao 1.5 de espacamento entre linhas
\setstretch{1.5}

\title{Aula 14 - Revisão do Conteúdo}

\author[1]{Augusto Mathias Adams}
\affil[1]{augusto.adams@ufpr.br}
\setcounter{Maxaffil}{0}
\renewcommand\Affilfont{\itshape\small}

\begin{document}
	% Seleciona o idioma do documento
	\selectlanguage{brazil}
	
	% título
	\maketitle
	
	\section{Aprendizado da Aula}
	
	\begin{itemize}
		\item \textbf{\textit{Como Se formam os Raios $\Rightarrow$ }} Os raios são formados por cargas elétricas que se acumulam em nuvens de tempestade. Quando essas cargas atingem um nível crítico, a resistência do ar é rompida e ocorre uma descarga elétrica em direção ao solo ou entre nuvens. Essa descarga é vista como um flash brilhante de luz, e é acompanhada por um som alto conhecido como trovão. Nesta aula, estuda-se o que acontece nas camadas superiores da atmosfera.
		\item \textbf{\textit{Magnetosfera $\Rightarrow$ }}A magnetosfera é a camada mais externa da Terra, que se estende até a magnetopausa, situada a uma distância variável entre 10 e 15 vezes o raio da Terra. Ela é delimitada na parte inferior pela ionosfera, que se estende em torno de 1000 km acima da superfície terrestre. A magnetosfera é uma região dominada pelo campo magnético da Terra, que influencia a dinâmica do plasma presente na região, composto principalmente de prótons e elétrons. Na parte voltada para o Sol, as linhas de fluxo magnético são achatadas devido à ação do vento solar, enquanto na extremidade oposta, a magnetosfera se alonga.
		\item \textbf{\textit{Plasmasfera $\Rightarrow$ }} A Plasmasfera é a região da alta atmosfera terrestre onde ocorrem os primeiros fenômenos associados às tempestades na Terra e está acoplada à região F da ionosfera. Descargas atmosféricas podem causar a propagação de ondas eletromagnéticas naturais de baixa frequência que se dirigem para as camadas mais altas da atmosfera e entram na Plasmasfera, propagando-se em formato espiralado entre os polos Norte e Sul. Essas ondas, geralmente na ordem de 5-10 kHz no espectro de frequência, são detectadas como assobios atmosféricos Whistlers. O estudo dos Whistlers está relacionado ao processo de compreensão da dinâmica da magnetosfera e das propriedades magnetoionicas dos elétrons e íons, que podem afetar a propagação de sinais eletromagnéticos neste meio.
		\item \textbf{\textit{Ionosfera $\Rightarrow$ }}A ionosfera é uma região da atmosfera terrestre que se estende entre cerca de 60 km e 1000 km de altitude, e é composta por íons e elétrons livres. É uma região altamente eletricamente condutiva e é influenciada pela radiação solar e pelas correntes elétricas que fluem na magnetosfera da Terra.
		
		A ionização da atmosfera superior é produzida pela radiação ultravioleta e pelos raios cósmicos que bombardeiam a atmosfera terrestre, o que resulta na ionização dos átomos e moléculas presentes. Os íons e elétrons livres na ionosfera interagem com as ondas eletromagnéticas, permitindo a propagação de sinais de rádio e televisão por longas distâncias, além de influenciar a transmissão de dados de sistemas de navegação por satélite.
		
		A ionosfera é dividida em várias camadas que são determinadas pela sua altura e densidade. As camadas da ionosfera são:
		\begin{itemize}
			\item \textbf{\textit{Camada D:}} A camada D é a mais próxima da superfície da Terra e está localizada entre 60 e 90 km de altitude. Esta camada é a mais densa de todas as camadas da ionosfera e é composta principalmente por íons N+ e O2+. A camada D é responsável pela reflexão das ondas de rádio de alta frequência (HF) e é por isso que é usada para comunicações de longa distância.
			
			\item \textbf{\textit{Camada E:}} A camada E está localizada acima da camada D, entre 90 e 120 km de altitude. Esta camada é menos densa do que a camada D e é composta principalmente de íons O+ e NO+. A camada E é importante para comunicações de rádio devido à sua capacidade de refletir ondas de rádio de frequência muito alta (VHF).
			
			\item \textbf{\textit{Camada F1:}} A camada F1 está localizada acima da camada E, entre 150 e 200 km de altitude. Esta camada é menos densa do que as camadas D e E e é composta principalmente por íons O+ e H+. A camada F1 é importante para comunicações de rádio devido à sua capacidade de refletir ondas de rádio de frequência ultra-alta (UHF).
			
			\item \textbf{\textit{Camada F2:}} A camada F2 é a camada mais alta da ionosfera e está localizada entre 200 e 500 km de altitude. Esta camada é composta principalmente de íons O+ e H+ e é menos densa do que as camadas D, E e F1. A camada F2 é importante para comunicações de rádio devido à sua capacidade de refletir ondas de rádio de frequência muito alta (VHF) e é a camada mais utilizada para comunicações de longa distância.
		\end{itemize}
	
		Distúrbios na ionosfera polar podem ocasionar interrupções na comunicação por ondas curtas, conhecidos como “apagões de rádio”. Além disso, correntes elétricas induzidas na ionosfera podem influenciar o fornecimento de energia e causar corrosão em oleodutos, entre outros problemas. Os raios cósmicos são divididos em partículas primárias, que permeiam o espaço interplanetário, e partículas secundárias, que surgem a partir de interações com outras partículas presentes na atmosfera. Quando hádrons incidem na alta atmosfera, geralmente sofrem interações de natureza forte ao colidir com núcleos atmosféricos como nitrogênio e oxigênio. Quando a energia das partículas de hádrons atinge cerca de $10^9$ eV, ocorre uma cascata de partículas, conhecida como Chuveiro de Partículas ou Avalanche de Partículas Cósmicas. Essas interações sucessivas geram uma série de partículas secundárias, incluindo mésons.
		
		A avalanche de partículas cósmicas, por sua vez, é utilizada como base para a teoria de eletrização da nuvem denominada \textbf{\textit{Runaway Breakdown}}.

	\end{itemize}
	
\end{document}