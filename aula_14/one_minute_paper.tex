%\documentclass[journal, onecolumn, letterpaper]{IEEEtran}
%\documentclass[journal,onecolumn]{IEEEtran}
% \documentclass[conference]{IEEEtran}
\documentclass[a4paper, 12pt, onecolumn,singlespacing]{article}

% The preceding line is only needed to identify funding in the first footnote. If that is unneeded, please comment it out.
\usepackage[level]{fmtcount} % equivalent to \usepackage{nth}
% \include{util}
\usepackage[portuguese, brazil, english]{babel}
\usepackage{multirow}
\usepackage{array} % for defining a new column type
\usepackage{varwidth} %for the varwidth minipage environment
\usepackage[super]{nth}
\usepackage{authblk}
\usepackage{cite}
\usepackage{amsmath,amssymb,amsfonts}
\usepackage{ulem}
\usepackage{graphicx}
% \usepackage{subfig}
\usepackage{textcomp}
\usepackage{xcolor}
\usepackage{mathptmx}
\usepackage[T1]{fontenc}
\usepackage{textcomp}
\usepackage{titlesec}
\usepackage{helvet}
\usepackage{gensymb}
\usepackage{setspace} % espacamento entre linhas
\usepackage{pgfplots}
\usepackage{tikz}
\usepackage{subcaption}
\usepackage{minted}
\usepackage[left=2cm, right=2cm, bottom=2cm, top=2cm]{geometry}
\usepackage{makecell}
\usepackage{pdfpages}
\usepackage{bm}


\usepackage{hyperref}
\usepackage{fancyhdr}
\renewcommand{\headrulewidth}{1pt}
\renewcommand{\footrulewidth}{0.5pt}
\fancyhf{} % limpa os cabecalhos e rodapés
\fancyhead[C]{\textit{CURSO DE ENGENHARIA DOS RAIOS - TE981} } % define o cabeçalho personalizado
\fancyfoot[C]{\textit{AUGUSTO MATHIAS ADAMS}}
\pagestyle{fancy} % sem definir esse comando, o cabeçalho personalizado não é exibido

\hypersetup{
	colorlinks=true,
	linkcolor=blue,
	filecolor=magenta,      
	urlcolor=blue,
	pdftitle={ENGENHARIA DOS RAIOS - TE981 - ONE MINUTE PAPER}
}
\renewcommand\theadalign{bc}
\renewcommand\theadfont{\bfseries}
\renewcommand\theadgape{\Gape[4pt]}
\renewcommand\cellgape{\Gape[4pt]}

%dashed line
\usepackage{booktabs, makecell}
\renewcommand\theadfont{\bfseries}
\renewcommand\theadgape{}
\usepackage{arydshln}
\setlength\dashlinedash{0.2pt}
\setlength\dashlinegap{1.5pt}
\setlength\arrayrulewidth{0.3pt}

% padrao 1.5 de espacamento entre linhas
\setstretch{1.5}
\makeatletter
\def\@maketitle{%
	\newpage
	\null
	\vskip 2em%
	\begin{center}%
		\let \footnote \thanks
		{\LARGE \@title \par}%
		\vskip 1.5em%
		{\large
			\lineskip .5em%
			\begin{tabular}[t]{c}%
				\@author
			\end{tabular}\par}%
		%\vskip 1em%
		%{\large \@date}%
	\end{center}%
	\par
	\vskip 1.5em}
\makeatother

\title{\normalsize{ENGENHARIA DOS RAIOS - TE981}\\ \huge{\textbf\textit{{AULA 14 - REVISÃO DO CONTEÚDO}}\\}}
\author{\small{AUGUSTO MATHIAS ADAMS}}
\setcounter{Maxaffil}{0}
\renewcommand\Affilfont{\itshape\small}

\begin{document}
	% Seleciona o idioma do documento
	\selectlanguage{brazil}
	
	% título
	\maketitle
	
	\section{Roadmap}
	
	\subsection{Iniciação de um raio e tipos de raio}
	
	A iniciação de um raio ocorre em diferentes etapas. Primeiro, há o desenvolvimento de um forte campo elétrico dentro de uma nuvem de tempestade. Isso resulta em uma descarga elétrica chamada de líder escalonado, que se move em direção ao solo em saltos. Quando o líder escalonado se aproxima do solo, ocorre a liberação de um líder ascendente a partir de objetos elevados próximos ao solo, completando o caminho para o raio.
	
	Existem diferentes tipos de raios que podem ocorrer durante uma tempestade. Os mais comuns são os raios intra-nuvem (IC) e os raios nuvem-solo (CG). Os raios IC ocorrem dentro da nuvem, envolvendo o movimento de cargas elétricas entre diferentes regiões da mesma nuvem. Os raios CG são aqueles que se estendem do topo da nuvem até o solo, conectando a nuvem e o solo com uma descarga elétrica intensa.
	
	Além desses, também existem outros tipos de raios menos comuns, como os raios nuvem-nuvem, que ocorrem entre diferentes nuvens, e os raios ascendentes, que se originam no solo e se propagam em direção à nuvem. Esses diferentes tipos de raios podem apresentar características distintas em relação à sua aparência, intensidade e duração.
	
	O estudo dos diferentes tipos de raios é importante para entender os mecanismos envolvidos nas descargas elétricas atmosféricas e os riscos associados a elas, como incêndios, danos em estruturas e riscos para a segurança humana.
	
	\subsection{Energia dissipada durante um raio}
	
	Durante um raio, ocorre a dissipação de uma grande quantidade de energia. A descarga do raio consiste em diferentes correntes elétricas, sendo a mais intensa a corrente de retorno, que dura cerca de 100-200 microssegundos e tem uma intensidade entre 10.000 e 100.000 amperes. O pico de corrente estimado para a corrente de retorno pode chegar a 300.000 amperes em regiões temperadas e 450.000-500.000 amperes em regiões tropicais.
	
	A dissipação de energia durante um raio ocorre principalmente devido à parte resistiva do canal de ar percorrido pela corrente elétrica. Essa dissipação se manifesta como calor ou energia térmica, aumentando a temperatura do canal do raio. A temperatura no núcleo do canal do raio pode atingir valores extremamente altos, da ordem de 30.000 Kelvin.
	
	A energia total dissipada durante um raio é significativa e pode ter diferentes efeitos observáveis, como incêndios florestais, danos em estruturas, sistemas de comunicação, linhas de energia e sistemas elétricos. A energia térmica dissipada durante um raio pode chegar a um valor de pico em torno de 10 bilhões de watts ($10^{10}$ W).
	
	É importante compreender a dissipação de energia durante os raios para avaliar os riscos associados a eles e implementar medidas de proteção adequadas para prevenir danos e minimizar os perigos potenciais.
	
	\subsection{Tempestades Como Geradores de Energia}	
	
	As tempestades são fenômenos atmosféricos que podem funcionar como poderosos geradores elétricos. Durante uma tempestade, ocorrem processos que levam à separação de cargas elétricas dentro das nuvens e entre as nuvens e a superfície da Terra. Isso cria um campo elétrico significativo, resultando em descargas elétricas, como raios.
	
	A formação de nuvens de tempestade envolve a ascensão de ar quente e úmido, que, ao subir, esfria e condensa, formando gotículas de água e cristais de gelo. Durante esse processo, as colisões entre as partículas resultam na separação de cargas elétricas, com cargas positivas concentradas nas partes superiores das nuvens e cargas negativas nas partes inferiores.
	
	Essa separação de cargas cria um campo elétrico verticalmente orientado dentro da nuvem. À medida que o campo elétrico se intensifica, ocorre a formação de líderes elétricos, que são descargas elétricas invisíveis que se movem em direção ao solo ou entre as nuvens.
	
	Quando um líder elétrico alcança o solo ou entra em contato com outro líder elétrico, ocorre uma descarga elétrica visível conhecida como raio. O raio é uma descarga elétrica intensa e de curta duração que equilibra as cargas elétricas entre a nuvem e o solo ou entre as nuvens.
	
	Durante a ocorrência de raios, correntes elétricas extremamente altas são geradas, atingindo valores de dezenas a centenas de milhares de amperes. Essas correntes elétricas resultam em uma dissipação significativa de energia, aquecendo o ar ao redor do canal do raio a temperaturas extremamente altas.
	
	As tempestades, portanto, funcionam como geradores elétricos naturais, convertendo a energia potencial elétrica armazenada na separação de cargas em energia cinética durante as descargas elétricas dos raios. O estudo e compreensão desses processos são essenciais para a previsão e mitigação de riscos relacionados a raios e tempestades, bem como para o desenvolvimento de tecnologias de proteção contra descargas elétricas.
	
	\subsection{Circuito Elétrico Atmosférico Global e Local}
	
	\paragraph{Circuito Elétrico Atmosférico Global} O Circuito Elétrico Atmosférico Global é um sistema complexo de correntes elétricas que ocorrem na atmosfera da Terra. Envolve a interação entre a ionosfera, a atmosfera inferior e a superfície terrestre, formando um circuito elétrico completo.
	
	Na ionosfera, localizada a uma altitude de cerca de 60 a 1.000 km, ocorrem processos de ionização, nos quais partículas carregadas são formadas devido à interação com a radiação solar. Essa região é influenciada principalmente pela luz ultravioleta do Sol.
	
	A atmosfera inferior, incluindo a troposfera e a estratosfera, é onde ocorrem os fenômenos meteorológicos, como tempestades e nuvens. Durante as tempestades, a separação de cargas elétricas cria campos elétricos intensos. Descargas elétricas, como raios, ocorrem para equilibrar essas cargas e transferir energia entre a atmosfera e a superfície terrestre.
	
	A superfície da Terra, por sua vez, atua como uma parte condutora do circuito, permitindo a transferência de cargas elétricas entre a atmosfera e a terra. A topografia, a cobertura vegetal e a presença de corpos d'água afetam a distribuição e a intensidade das correntes elétricas.
	
	Esse circuito elétrico global é influenciado por vários fatores, como a atividade solar, a composição da atmosfera, as características geográficas e as condições meteorológicas locais. Além disso, fenômenos como as correntes de jato, as frentes atmosféricas e as variações na condutividade atmosférica também desempenham um papel importante na dinâmica desse circuito.
	
	O estudo do Circuito Elétrico Atmosférico Global é fundamental para entender os processos elétricos na atmosfera, a ocorrência de fenômenos meteorológicos e a influência das correntes elétricas na ionosfera. Também é relevante para o desenvolvimento de tecnologias de monitoramento e previsão de tempestades, bem como para a proteção de infraestruturas sensíveis a descargas elétricas, como sistemas de energia e comunicações.
	
	\paragraph{Circuito Elétrico Atmosférico Local}
	
	O Circuito Elétrico Atmosférico Local refere-se ao sistema de correntes elétricas que ocorrem em uma área específica da atmosfera durante a ocorrência de tempestades e descargas elétricas. É uma parte do Circuito Elétrico Atmosférico Global e está intimamente relacionado às condições meteorológicas locais.
	
	Durante as tempestades, ocorre a separação de cargas elétricas dentro das nuvens, resultando na formação de campos elétricos intensos. Esses campos elétricos podem criar uma diferença de potencial significativa entre a nuvem e a superfície terrestre, estabelecendo assim um circuito elétrico local.
	
	Nesse circuito, as nuvens atuam como a fonte de energia, fornecendo a carga elétrica e gerando campos elétricos intensos. As descargas elétricas, como os raios, ocorrem para equalizar essa diferença de potencial, permitindo a transferência de carga elétrica entre a nuvem e a superfície terrestre.
	
	A superfície terrestre desempenha um papel importante como uma parte condutora do circuito. Através de objetos elevados, como edifícios, árvores ou montanhas, ocorre a ionização do ar e a formação de líderes ascendentes, que são canais condutivos que se estendem em direção à nuvem. Esses líderes ascendentes são responsáveis pela formação dos canais de retorno, nos quais ocorre a descarga de retorno, conectando a nuvem à superfície terrestre.
	
	O Circuito Elétrico Atmosférico Local é influenciado por uma variedade de fatores, incluindo a intensidade das tempestades, a presença de partículas carregadas na atmosfera, a condutividade atmosférica e as características da superfície terrestre. A topografia, a presença de corpos d'água e a vegetação também podem afetar a dinâmica desse circuito.
	
	O estudo do Circuito Elétrico Atmosférico Local é essencial para compreender a geração de descargas elétricas, a formação de raios e os efeitos associados, como trovões e fenômenos luminosos transientes, como os sprites. Além disso, é importante para a proteção de infraestruturas e para a segurança humana durante tempestades, permitindo o desenvolvimento de tecnologias de monitoramento e previsão de descargas elétricas.
	
	\subsection{Física das Descargas Atmosféricas}
	
	A física das descargas atmosféricas é um campo de estudo que se dedica a compreender os processos físicos envolvidos nas descargas elétricas que ocorrem na atmosfera, como osraios.
	
	As descargas atmosféricas são fenômenos naturais de alta energia que envolvem a transferência de cargas elétricas entre nuvens, entre nuvens e a Terra, ou dentro de uma única nuvem. Elas ocorrem	devido a uma diferença de potencial elétrico significativa entre duas regiões na atmosfera, que pode ser causada por processos de eletrificação das nuvens, movimento vertical de partículas carregadas,	entre outros fatores.
	
	Existem vários tipos de descargas atmosféricas, sendo os raios nuvem-terra (cloud-to-ground) os mais conhecidos e estudados. Esses raios são caracterizados por uma corrente elétrica intensa que percorre um caminho condutor entre uma nuvem carregada eletricamente e a superfície da Terra. Essa corrente é composta por pulsos de alta velocidade chamados de líderes, que são descargas elétricas ascendentes e descendentes que procuram estabelecer um caminho condutor através do ar ionizado.
	
	A formação de um raio envolve uma série complexa de processos físicos, incluindo a ionização do	ar, a formação de líderes e o desenvolvimento de um canal condutor para a corrente elétrica. Durante o processo, ocorrem colisões entre partículas eletricamente carregadas, a geração de campos elétricos intensos, a criação de plasma e o aquecimento do ar, resultando em emissões de luz (relâmpagos) e sons (trovões).
	
	A física das descargas atmosféricas é estudada utilizando uma combinação de observações em campo, experimentos laboratoriais e modelagem computacional. Os pesquisadores procuram entender os mecanismos fundamentais que governam a ocorrência, a propagação e os efeitos das descargas atmosféricas, a fim de melhorar a previsão de tempestades e desenvolver medidas de proteção contra	raios.
	
\end{document}