%\documentclass[journal, onecolumn, letterpaper]{IEEEtran}
%\documentclass[journal,onecolumn]{IEEEtran}
% \documentclass[conference]{IEEEtran}
\documentclass[a4paper, 12pt, onecolumn,singlespacing]{article}

% The preceding line is only needed to identify funding in the first footnote. If that is unneeded, please comment it out.
\usepackage[level]{fmtcount} % equivalent to \usepackage{nth}
% \include{util}
\usepackage[portuguese, brazil, english]{babel}
\usepackage{multirow}
\usepackage{array} % for defining a new column type
\usepackage{varwidth} %for the varwidth minipage environment
\usepackage[super]{nth}
\usepackage{authblk}
\usepackage{cite}
\usepackage{amsmath,amssymb,amsfonts}
\usepackage{ulem}
\usepackage{graphicx}
% \usepackage{subfig}
\usepackage{textcomp}
\usepackage{xcolor}
\usepackage{mathptmx}
\usepackage[T1]{fontenc}
\usepackage{textcomp}
\usepackage{titlesec}
\usepackage{helvet}
\usepackage{gensymb}
\usepackage{setspace} % espacamento entre linhas
\usepackage{pgfplots}
\usepackage{tikz}
\usepackage{subcaption}
\usepackage{minted}
\usepackage[left=2cm, right=2cm, bottom=2cm, top=2cm]{geometry} 
\usepackage{makecell}
\usepackage{pdfpages}

\usepackage{hyperref}
\usepackage{fancyhdr}
\renewcommand{\headrulewidth}{1pt}
\renewcommand{\footrulewidth}{0.5pt}
\fancyhf{} % limpa os cabecalhos e rodapés
\fancyhead[C]{\textit{CURSO DE ENGENHARIA DOS RAIOS - TE981} } % define o cabeçalho personalizado
\fancyfoot[C]{\textit{AUGUSTO MATHIAS ADAMS}}
\pagestyle{fancy} % sem definir esse comando, o cabeçalho personalizado não é exibido

\hypersetup{
	colorlinks=true,
	linkcolor=blue,
	filecolor=magenta,      
	urlcolor=blue,
	pdftitle={ENGENHARIA DOS RAIOS - TE981 - ATIVIDADE REMOTA 4}
}
\renewcommand\theadalign{bc}
\renewcommand\theadfont{\bfseries}
\renewcommand\theadgape{\Gape[4pt]}
\renewcommand\cellgape{\Gape[4pt]}

%dashed line
\usepackage{booktabs, makecell}
\renewcommand\theadfont{\bfseries}
\renewcommand\theadgape{}
\usepackage{arydshln}
\setlength\dashlinedash{0.2pt}
\setlength\dashlinegap{1.5pt}
\setlength\arrayrulewidth{0.3pt}

% padrao 1.5 de espacamento entre linhas
\setstretch{1.5}
\makeatletter
\def\@maketitle{%
	\newpage
	\null
	\vskip 2em%
	\begin{center}%
		\let \footnote \thanks
		{\LARGE \@title \par}%
		\vskip 1.5em%
		{\large
			\lineskip .5em%
			\begin{tabular}[t]{c}%
				\@author
			\end{tabular}\par}%
		%\vskip 1em%
		%{\large \@date}%
	\end{center}%
	\par
	\vskip 1.5em}
\makeatother

\title{\normalsize{ENGENHARIA DOS RAIOS - TE981}\\ \huge{\textbf\textit{{ATIVIDADE REMOTA - VÍDEO 4}}\\}}
\author{\small{AUGUSTO MATHIAS ADAMS}}
\setcounter{Maxaffil}{0}
\renewcommand\Affilfont{\itshape\small}

\begin{document}
	% Seleciona o idioma do documento
	\selectlanguage{brazil}
	
	% título
	\maketitle
	
	\section{Aprendizado da Aula}
	
	Atividade remota sobre o \textbf{\textit{Vídeo 4 - DESVENDANDO OS RELÂMPAGOS E RAIOS - Documentário (2010)}}.
	
	\textit{Sinopse: } Veja as novas pesquisas pelo mundo sobre tempestades de raios e trovões usando a mais moderna tecnologia nos estudos desses fenômenos e também a história dessa ciência.
	
	Sugestão do Professor: levar ao menos uma dúvida para discussão em sala de aula.
	
	\section{Dúvidas e Questionamentos}
	
	\begin{itemize}
		\item \textit{Cada raio é um raio: } quanto mais se estuda raios, mais diversos eles parecem. Concordo com isto
		\item \textit{Câmeras rápidas: } eu gostaria de umas câmera de 54 mil quadros por segundo pra fazer a mesma coisa que os americanos fazem. Muito bonito, muito profissional e científico.
		\item Estudar raios induzidos por foguetes é como estudar ratos em uma gaiola: não é o mesmo comportamento.
		\item Torres aumentam tanto a probabilidade dos raios caírem em uma área? No SIPDA me lembro do Rachidi reclamando que as hastes de Franklin de fato diminuem a probabilidade de um raio atingir a estrutura.
		\item \textit{Mapa de Raios: } aquele mapa deve estar mudando
		\item \textit{Sprites: } se uma câmera consegue ver estrelas, consegue ver \textit{sprites} $\Rightarrow$ bom saber disto.....
	\end{itemize}
	
\end{document}