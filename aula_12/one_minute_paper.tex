%\documentclass[journal, onecolumn, letterpaper]{IEEEtran}
%\documentclass[journal,onecolumn]{IEEEtran}
% \documentclass[conference]{IEEEtran}
\documentclass[a4paper, 12pt, onecolumn,singlespacing]{article}

% The preceding line is only needed to identify funding in the first footnote. If that is unneeded, please comment it out.
\usepackage[level]{fmtcount} % equivalent to \usepackage{nth}
% \include{util}
\usepackage[portuguese, brazil, english]{babel}
\usepackage{multirow}
\usepackage{array} % for defining a new column type
\usepackage{varwidth} %for the varwidth minipage environment
\usepackage[super]{nth}
\usepackage{authblk}
\usepackage{cite}
\usepackage{amsmath,amssymb,amsfonts}
\usepackage{ulem}
\usepackage{graphicx}
% \usepackage{subfig}
\usepackage{textcomp}
\usepackage{xcolor}
\usepackage{mathptmx}
\usepackage[T1]{fontenc}
\usepackage{textcomp}
\usepackage{titlesec}
\usepackage{helvet}
\usepackage{gensymb}
\usepackage{setspace} % espacamento entre linhas
\usepackage{pgfplots}
\usepackage{tikz}
\usepackage{subcaption}
\usepackage{minted}
\usepackage[left=2cm, right=2cm, bottom=2cm, top=2cm]{geometry} 
\usepackage{makecell}
\usepackage{pdfpages}

\usepackage{hyperref}
\usepackage{fancyhdr}
\renewcommand{\headrulewidth}{1pt}
\renewcommand{\footrulewidth}{0.5pt}
\fancyhf{} % limpa os cabecalhos e rodapés
\fancyhead[C]{\textit{CURSO DE ENGENHARIA DOS RAIOS - TE981} } % define o cabeçalho personalizado
\fancyfoot[C]{\textit{AUGUSTO MATHIAS ADAMS}}
\pagestyle{fancy} % sem definir esse comando, o cabeçalho personalizado não é exibido

\hypersetup{
	colorlinks=true,
	linkcolor=blue,
	filecolor=magenta,      
	urlcolor=blue,
	pdftitle={ENGENHARIA DOS RAIOS - TE981 - ONE MINUTE PAPER}
}
\renewcommand\theadalign{bc}
\renewcommand\theadfont{\bfseries}
\renewcommand\theadgape{\Gape[4pt]}
\renewcommand\cellgape{\Gape[4pt]}

%dashed line
\usepackage{booktabs, makecell}
\renewcommand\theadfont{\bfseries}
\renewcommand\theadgape{}
\usepackage{arydshln}
\setlength\dashlinedash{0.2pt}
\setlength\dashlinegap{1.5pt}
\setlength\arrayrulewidth{0.3pt}

% padrao 1.5 de espacamento entre linhas
\setstretch{1.5}
\makeatletter
\def\@maketitle{%
	\newpage
	\null
	\vskip 2em%
	\begin{center}%
		\let \footnote \thanks
		{\LARGE \@title \par}%
		\vskip 1.5em%
		{\large
			\lineskip .5em%
			\begin{tabular}[t]{c}%
				\@author
			\end{tabular}\par}%
		%\vskip 1em%
		%{\large \@date}%
	\end{center}%
	\par
	\vskip 1.5em}
\makeatother

\title{\normalsize{ENGENHARIA DOS RAIOS - TE981}\\ \huge{\textbf\textit{{AULA 12 - CIRCUITO ATMOSFÉRICO GLOBAL}}\\}}
\author{\small{AUGUSTO MATHIAS ADAMS}}
\setcounter{Maxaffil}{0}
\renewcommand\Affilfont{\itshape\small}

\begin{document}
	% Seleciona o idioma do documento
	\selectlanguage{brazil}
	
	% título
	\maketitle
	
	\section{Aprendizado da Aula}
	
	\begin{itemize}
		
		\item \textbf{\textit{Circuito Elétrico Atmosférico Global $\Rightarrow$ }}é possível modelar a atividade elétrica da Terra e sua atmosfera com um circuito elétrico global. Este modelo é composto por diferentes elementos elétricos, tais como fontes de energia, resistores e capacitores, que representam os processos elétricos que ocorrem na Terra e na sua atmosfera.
		
		Por exemplo, a fonte de energia primária no circuito elétrico global é a radiação solar, que é absorvida pela atmosfera e pela superfície terrestre, gerando diferenças de potencial elétrico. Estas diferenças de potencial elétrico são transportadas por correntes elétricas atmosféricas e terrestres, através de resistores que representam a resistência dos diferentes materiais que compõem a Terra e sua atmosfera.
		
		Além disso, existem capacitores que representam as diferentes camadas da atmosfera e da ionosfera, que armazenam cargas elétricas e geram campos elétricos. Estes campos elétricos interagem com as partículas carregadas presentes na atmosfera e na ionosfera, gerando diferentes fenômenos, tais como raios, auroras e perturbações geomagnéticas.
		
		Assim, o modelo do circuito elétrico global pode ser útil para entender e descrever os processos elétricos que ocorrem na Terra e na sua atmosfera, permitindo prever e detectar eventos como tempestades geomagnéticas e perturbações atmosféricas que possam afetar a tecnologia e as comunicações.
		
		\subitem \textbf{\textit{Circuito Elétrico Atmosférico Local $\Rightarrow$ }}entender a distribuição de carga e os fenômenos de acoplamento na atmosfera através de modelos de circuitos elétricos atmosféricos locais pode fornecer informações importantes sobre o comportamento eletrostático e eletrodinâmico da baixa atmosfera da Terra. Além disso, permite a análise da geração e propagação de enormes fluxos de cargas elétricas das nuvens para o solo ou linhas de transmissão, o que pode levar a perturbações nos sistemas condutores de corrente elétrica. A compreensão desses fenômenos é essencial para garantir a segurança e confiabilidade das redes elétricas e de telecomunicações.
		
		Um circuito elétrico atmosférico local é um modelo que representa a distribuição de cargas elétricas e correntes na atmosfera em uma região específica. Ele pode ser usado para estudar os efeitos da atividade elétrica na atmosfera, como relâmpagos, tempestades e auroras, bem como seus efeitos em sistemas elétricos, como linhas de transmissão de energia. O modelo inclui elementos como capacitores, indutores e resistores, que representam as diferentes características da atmosfera, como sua condutividade elétrica, sua capacidade de armazenamento de carga e a presença de campos magnéticos. O circuito elétrico atmosférico local pode ser usado para simular e prever o comportamento da atmosfera em uma determinada região, o que pode ser útil para diversos fins, como a prevenção de danos causados por descargas elétricas em equipamentos elétricos e a proteção de infraestrutura crítica, como torres de transmissão e antenas.
	
	\end{itemize}
	\section{Temas Impactantes, dúvidas e questionamentos}
	Mesmo da aula anterior.
\end{document}