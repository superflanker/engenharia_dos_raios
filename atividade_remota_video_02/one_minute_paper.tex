%\documentclass[journal, onecolumn, letterpaper]{IEEEtran}
%\documentclass[journal,onecolumn]{IEEEtran}
% \documentclass[conference]{IEEEtran}
\documentclass[a4paper, 12pt, onecolumn,singlespacing]{article}

% The preceding line is only needed to identify funding in the first footnote. If that is unneeded, please comment it out.
\usepackage[level]{fmtcount} % equivalent to \usepackage{nth}
% \include{util}
\usepackage[portuguese, brazil, english]{babel}
\usepackage{multirow}
\usepackage{array} % for defining a new column type
\usepackage{varwidth} %for the varwidth minipage environment
\usepackage[super]{nth}
\usepackage{authblk}
\usepackage{cite}
\usepackage{amsmath,amssymb,amsfonts}
\usepackage{ulem}
\usepackage{graphicx}
% \usepackage{subfig}
\usepackage{textcomp}
\usepackage{xcolor}
\usepackage{mathptmx}
\usepackage[T1]{fontenc}
\usepackage{textcomp}
\usepackage{titlesec}
\usepackage{helvet}
\usepackage{gensymb}
\usepackage{setspace} % espacamento entre linhas
\usepackage{pgfplots}
\usepackage{tikz}
\usepackage{subcaption}
\usepackage{minted}
\usepackage[left=2cm, right=2cm, bottom=2cm, top=2cm]{geometry} 
\usepackage{makecell}
\usepackage{pdfpages}
\usepackage{hyperref}

\renewcommand\theadalign{bc}
\renewcommand\theadfont{\bfseries}
\renewcommand\theadgape{\Gape[4pt]}
\renewcommand\cellgape{\Gape[4pt]}

%dashed line
\usepackage{booktabs, makecell}
\renewcommand\theadfont{\bfseries}
\renewcommand\theadgape{}
\usepackage{arydshln}
\setlength\dashlinedash{0.2pt}
\setlength\dashlinegap{1.5pt}
\setlength\arrayrulewidth{0.3pt}

% padrao 1.5 de espacamento entre linhas
\setstretch{1.5}

\title{Atividade Remota (Video 02)}

\author[1]{Augusto Mathias Adams}
\affil[1]{augusto.adams@ufpr.br}
\setcounter{Maxaffil}{0}
\renewcommand\Affilfont{\itshape\small}

\begin{document}
	% Seleciona o idioma do documento
	\selectlanguage{brazil}
	
	% título
	\maketitle
	
	\section{Aprendizado da Aula}
	
	Atividade remota sobre o \textbf{\textit{Vídeo 2 - Planeta Feroz Raio Dublado Documentário Discovery }}.
	
	Sinopse: Raio é uma descarga elétrica de grande intensidade que ocorre na atmosfera, entre regiões eletricamente carregadas, e pode dar-se tanto no interior de uma nuvem, como entre nuvens ou entre nuvem e terra. O raio vem sempre acompanhado do relâmpago (intensa emissão de radiação eletromagnética também visível), e do trovão (som estrondoso), além de outros fenômenos associados. Embora sejam mais frequentes descargas dentro das nuvens (as intra-nuvens) e entre duas nuvens (as inter-nuvens), descargas entre nuvens e a terra são de maior interesse prático para o homem. A maior parte ocorre na zona tropical do planeta e principalmente sobre as terras emersas, associados a fenômenos convectivos, dos quais, quando é intensa a atividade elétrica, resultam as trovoadas.
	
	Sugestão do Professor: levar ao menos uma dúvida para discussão em sala de aula.
	
	\section{Dúvidas e Questionamentos}
	
	\begin{itemize}
		\item Cada raio tem centímetros de espessura e quilômetros de extensão: como um evento geometricamente pequeno pode causar tanto estrago?
		\item Nuvens de Bigorna $\Rightarrow$ Cumulo nimbus (Coerente)
		\item A cada minuto tem 2000 tempestades ocorrendo no mundo: verdade ou mito? Se cada tempestade tem uma média de 2 a 4 raios por minuto, seriam 4000 ou 8000 raios por minuto, por isto a pergunta - não tenho esta estatística.
		\item \textit{``Um raio mata 1 em cada 6 pessoas que atinge''} $\Rightarrow$ esta estatística é bem mais medonha do que a taxa de sobrevivência do Video 01 (90\%) e, não argumentando sobre a precisão da estatística, é bem mais impactante e realista.
		\item Um raio não cai exatamente do nada, certo? o video causa esta impressão.....
		\item O video diz que a nuvem fica eletrificada com cargas negativas embaixo e positivas em cima. O modelo de tripolo é o que descreve de forma mais precisa o que acontece dentro da nuvem?
		\item a descrição da descarga atmosférica negativa descendente está coerente com o que vimos em sala de aula.
		\item O vídeo traz uma descrição do \textit{dart leader}, mas não o trata por este nome.
		\item \textit{``Vários cientistas que tentaram replicar o experimento da pipa de Franklin morreram''} - na verdade, o que eles não entenderam é como foi feito o experimento.
		\item \textit{``Sequelas em Sobreviventes''} $\Rightarrow$ fato impactante do vídeo, realmente não compreendemos muito bem as implicações de ser vítima não fatal de raio. Sobrevivo, mas como? eis a questão.
		\item O vídeo fala de pára raios, interessante saber que haste de Franklin ainda faz sucesso.
		\item Mais tecnologia significa maior vulnerabilidade aos raios $\Rightarrow$ tendo a concordar com este ponto de vista.
		\item Eventos Luminosos Transientes $\Rightarrow$ uma pincelada é dada no video
	\end{itemize}
	
\end{document}