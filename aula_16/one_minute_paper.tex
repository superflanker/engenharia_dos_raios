%\documentclass[journal, onecolumn, letterpaper]{IEEEtran}
%\documentclass[journal,onecolumn]{IEEEtran}
% \documentclass[conference]{IEEEtran}
\documentclass[a4paper, 12pt, onecolumn,singlespacing]{article}

% The preceding line is only needed to identify funding in the first footnote. If that is unneeded, please comment it out.
\usepackage[level]{fmtcount} % equivalent to \usepackage{nth}
% \include{util}
\usepackage[portuguese, brazil, english]{babel}
\usepackage{multirow}
\usepackage{array} % for defining a new column type
\usepackage{varwidth} %for the varwidth minipage environment
\usepackage[super]{nth}
\usepackage{authblk}
\usepackage{cite}
\usepackage{amsmath,amssymb,amsfonts}
\usepackage{ulem}
\usepackage{graphicx}
% \usepackage{subfig}
\usepackage{textcomp}
\usepackage{xcolor}
\usepackage{mathptmx}
\usepackage[T1]{fontenc}
\usepackage{textcomp}
\usepackage{titlesec}
\usepackage{helvet}
\usepackage{gensymb}
\usepackage{setspace} % espacamento entre linhas
\usepackage{pgfplots}
\usepackage{tikz}
\usepackage{subcaption}
\usepackage{minted}
\usepackage[left=2cm, right=2cm, bottom=2cm, top=2cm]{geometry} 
\usepackage{makecell}
\usepackage{pdfpages}

\usepackage{hyperref}
\usepackage{fancyhdr}
\renewcommand{\headrulewidth}{1pt}
\renewcommand{\footrulewidth}{0.5pt}
\fancyhf{} % limpa os cabecalhos e rodapés
\fancyhead[C]{\textit{CURSO DE ENGENHARIA DOS RAIOS - TE981} } % define o cabeçalho personalizado
\fancyfoot[C]{\textit{AUGUSTO MATHIAS ADAMS}}
\pagestyle{fancy} % sem definir esse comando, o cabeçalho personalizado não é exibido

\hypersetup{
	colorlinks=true,
	linkcolor=blue,
	filecolor=magenta,      
	urlcolor=blue,
	pdftitle={ENGENHARIA DOS RAIOS - TE981 - ONE MINUTE PAPER}
}
\renewcommand\theadalign{bc}
\renewcommand\theadfont{\bfseries}
\renewcommand\theadgape{\Gape[4pt]}
\renewcommand\cellgape{\Gape[4pt]}

%dashed line
\usepackage{booktabs, makecell}
\renewcommand\theadfont{\bfseries}
\renewcommand\theadgape{}
\usepackage{arydshln}
\setlength\dashlinedash{0.2pt}
\setlength\dashlinegap{1.5pt}
\setlength\arrayrulewidth{0.3pt}

% padrao 1.5 de espacamento entre linhas
\setstretch{1.5}
\makeatletter
\def\@maketitle{%
	\newpage
	\null
	\vskip 2em%
	\begin{center}%
		\let \footnote \thanks
		{\LARGE \@title \par}%
		\vskip 1.5em%
		{\large
			\lineskip .5em%
			\begin{tabular}[t]{c}%
				\@author
			\end{tabular}\par}%
		%\vskip 1em%
		%{\large \@date}%
	\end{center}%
	\par
	\vskip 1.5em}
\makeatother

\title{\normalsize{ENGENHARIA DOS RAIOS - TE981}\\ \huge{\textbf\textit{{AULA 16 - NBEs E CAMPO ELÉTRICO ABAIXO DA TEMPESTADE}}\\}}
\author{\small{AUGUSTO MATHIAS ADAMS}}
\setcounter{Maxaffil}{0}
\renewcommand\Affilfont{\itshape\small}

\begin{document}
	% Seleciona o idioma do documento
	\selectlanguage{brazil}
	
	% título
	\maketitle
	
	\section{Aprendizado da Aula}
	
	\paragraph{Pulsos Bipolares Estreitos} Os Pulsos Bipolares Estreitos (\textit{Narrow Bipolar Pulses - NBPs}) são um tipo de descarga atmosférica intranuvem que ocorre durante tempestades elétricas. Esses pulsos são caracterizados por sua curta duração e forte emissão de radiofrequência (\textit{RF}) no espectro de frequência muito alta (\textit{VHF}).
	
	Os \textit{NBPs} são observados comumente em regiões tropicais e subtropicais, mais do que em regiões temperadas. Eles estão associados à iniciação de descargas intranuvem sem a formação de líderes escalonados, o que significa que eles não requerem o processo de formação gradual de canais elétricos como outros tipos de descargas.
	
	Existem dois tipos principais de \textit{NBPs}: os Pulsos Bipolares Estreitos Positivos (\textit{Narrow Positive Bipolar Pulses - NPBPs}) e os Pulsos Bipolares Estreitos Negativos (\textit{Narrow Negative Bipolar Pulses - NNBPs}). Os \textit{NPBPs} ocorrem em altitudes de aproximadamente 6 km a 15 km, enquanto os \textit{NNBPs} ocorrem em altitudes entre 15 km e 21 km.
	
	Os campos elétricos medidos durante os \textit{NBPs} são relativamente baixos, com valores em torno de 22.7 $V/m$ para \textit{NPBPs} e 17.6 $V/m$ para \textit{NNBPs}. Essas descargas representam áreas de estudo importantes para entender melhor os processos elétricos na atmosfera durante as tempestades.
	
	Atualmente, os sistemas \textit{wireless} operam em frequências que variam de 2.4 $GHz$ a 5.8 $GHz$. Essas faixas de frequência são escolhidas porque são livres e abertas para uso por qualquer pessoa. Observações recentes da radiação de micro-ondas proveniente de relâmpagos despertaram interesse no estudo de seus efeitos em redes de comunicação sem fio.
	
	Cientistas observaram uma forte radiação de micro-ondas em 1.63 $GHz$ associada a eventos de relâmpagos do tipo nuvem-solo, como \textit{NBPs}, \textit{Steep Leader}, \textit{Dart Leader} e retorno do raio. Medições detectaram radiação de micro-ondas em ondas milimétricas a 37.5 $GHz$ a partir de um retorno do raio a uma distância inferior a 5 $km$. O sinal teve uma duração de 20 a 60 $\mu s$. A intensidade máxima de radiação espectral foi superior a 10-19 $W/(m^2·Hz)$, ou aproximadamente -180 $dB$.
	
	Essas descobertas têm aumentado a conscientização sobre o impacto potencial da radiação de micro-ondas induzida por relâmpagos em sistemas de comunicação sem fio. Pesquisas adicionais são necessárias para entender melhor e mitigar esses efeitos, a fim de garantir uma comunicação sem fio confiável e ininterrupta na presença de atividade de relâmpagos.
	
	
	\paragraph{Campo Elétrico Abaixo da Tempestade}
	
	Durante condições de tempo bom, o campo elétrico vertical oscila em torno de valores negativos, pois o potencial elétrico aumenta com a altura. Isso significa que uma antena vertical terá um potencial positivo em relação à sua base fixada no solo.
	
	No entanto, durante uma tempestade, o campo elétrico vertical pode se tornar muito grande, da ordem de 4 kV/m, e geralmente muda de sinal (polaridade) a partir dos valores negativos. Essas oscilações no campo elétrico vertical são causadas pelo movimento das nuvens, que se deslocam de forma relativamente lenta, influenciando as variações do campo elétrico também no solo.
	
	As nuvens carregadas de uma tempestade podem induzir uma distribuição desigual de cargas no solo, criando diferenças de potencial significativas entre diferentes pontos. À medida que as nuvens se movem, o campo elétrico pode variar devido a alterações na distribuição de carga atmosférica.
	
	Essas variações no campo elétrico são um dos fatores que contribuem para a ocorrência de descargas atmosféricas, como raios. Quando o campo elétrico atinge um valor crítico, pode ocorrer a quebra dielétrica do ar, resultando em uma descarga elétrica intensa entre as nuvens e o solo, equilibrando momentaneamente as diferenças de potencial.
	
	Consideremos então, uma superfície plana e condutora. O campo elétrico vertical sobre esta superfície pode ser escrita	como:
	
	\begin{equation}
		E_z = -\frac{1}{2 \pi \epsilon_0} \left[\int \frac{M(t')dt}{r^3} + \frac{M(t')}{ct^2} + \frac{\frac{dM(t)}{dt}}{rc^2}\right] 
	\end{equation}
	
	Ainda:
	
	\begin{equation}
		H_\phi = \frac{1}{2 \pi} \left[ \frac{M(t')}{r^2} + \frac{\frac{dM(t)}{dt}}{rc}\right]
	\end{equation}
	
	Onde $E_z$ é o campo elétrico vertical em $V/m$. $H_\phi$ é o campo magnético tangencial em $A/m$. $\epsilon_0$ é a permissividade no espaço livre ($(36 \pi \times 10^9)^{-1}$) em $F/m$. $M(t)$ define o momento elétrico vertical em $A.m$. $I(t)$ representa o dipolo de corrente da antena em $A$. $l$ é o comprimento do dipolo da antena em metros. $r$ é a distância da fonte até o observador em metros. $c$ é a velocidade da luz em $m/s$ e $t'$ é por definição igual a $(t - d/c)$.
	
	Se consideramos $I(t) = I cos(\omega t)$, reescrevenmos as equações anteriores como:
	
	\begin{equation}
		\begin{split}
			E_z = -\frac{Il}{2 \pi \epsilon_0} \left[\frac{sen(\omega t')}{\omega r^3} + \frac{cos (\omega t')}{cr^2} - \frac{w sen(\omega t')}{rc^2}\right] \\
			H_\phi = \frac{Il}{2 \pi \epsilon_0} \left[ \frac{cos (\omega t')}{r^2} - \frac{\omega sen(\omega t')}{rc}\right]
		\end{split}
	\end{equation}
	
	Das equações anteriores podemos obter o campo elétrico das componentes de radiação, indução e eletrostática, como segue:
	
	\begin{equation}
		\begin{split}
			E_r = -\frac{2 \frac{dM(t')}{dt}}{10^7 r}\\
			E_i = -\frac{60 M(t')}{r^2}\\
			E_e = - \frac{1.8 \times 10^10 \int M(t')dt}{r^3}
		\end{split}
	\end{equation}
	
	Estas equações representam muito bem o comportamento do campo elétrico em suas componentes de radiação, indução e eletrostática, do que se esperaria para medidas feitas abaixo de 2 $km$ de altitude, a partir do nível do solo.
	
	
	O pico da componente de radiação do campo elétrico médio para descargas atmosféricas (\textit{strokes}) pode ser estimado fazendo:
	
	\begin{equation}
		E_r \approxeq \frac{300}{r}
	\end{equation}
	
	Já o pico da componente de radiação do campo elétrico médio para uma descarga de retorno (return stroke), é estimado	pela relação:
	
	\begin{equation}
		E_r \approxeq \frac{260}{r}
	\end{equation}
	
	Quando o termo eletrostático é calculado com $\int \frac{dM(t')}{dt} = 1.5 \times 10^3 \frac{Am}{s}$, tem-se:
	
	\begin{equation}
		E_e \approxeq \frac{3 \times 10^4}{r^3}
	\end{equation}
	
	Para considerar uma estimativa mais próxima da realidade, podemos assumir uma concentração de cargas de 20 $C$ numa tempestade, a uma altura média de 2,5 $km$, e obter uma estimativa de campo elétrico, para os raios, como:
	
	\begin{equation}
		E_e \approxeq \frac{10^6}{r^3}
	\end{equation}
	Que é um campo elétrico muito próximo dos valores hoje mensurados por equipamentos específicos de monitoramento do campo elétrico atmosférico local. O campo elétrico de tempo bom corresponde a um campo medido em dia de céu	claro, desprovido de nuvens. Este campo é vertical com orientação para baixo, cujo valor próximo à superfície terrestre varia entre 100 - 300 $V/m$
	.
	\section{Temas Impactantes, dúvidas e questionamentos}
	
	\paragraph{Pode haver alguma interferência em altas frequências numa comunicação Wireless entre uma Estação - satélite - Estação?}
	
	Sim, descargas atmosféricas, como relâmpagos, podem causar interferência em comunicações wireless entre uma estação terrestre e um satélite. Isso ocorre devido aos seguintes fatores:
	
	\begin{itemize}
		\item \textbf{\textit{Emissão de ruído eletromagnético:}} Uma descarga atmosférica, especialmente os raios, produz uma intensa emissão de energia eletromagnética em um amplo espectro de frequências, incluindo altas frequências. Essa emissão pode interferir nos sinais de comunicação wireless, causando degradação do sinal ou perda de conexão.
		
		\item \textbf{\textit{Atenuação do sinal:}} A descarga atmosférica, dependendo da sua proximidade e intensidade, pode causar atenuação significativa nos sinais de comunicação wireless. O caminho percorrido pelo sinal pode ser afetado pela ionização eletrostática causada pela descarga, resultando em perda de potência do sinal e redução da qualidade da comunicação.
		
		\item \textbf{\textit{Sobrecarga dos circuitos:}} Descargas atmosféricas próximas podem gerar pulsos de alta corrente e tensão que podem ser induzidos nos cabos e equipamentos de comunicação. Esses pulsos de energia podem causar danos aos circuitos eletrônicos, resultando em falhas na comunicação.
	\end{itemize}

	
	Para mitigar os efeitos da interferência causada por descargas atmosféricas, são adotadas medidas como a utilização de sistemas de proteção contra surtos, aterramento adequado dos equipamentos e a implementação de protocolos de detecção e correção de erros nas transmissões. Além disso, a monitorização do clima e a suspensão temporária das comunicações durante tempestades severas podem ser realizadas para evitar problemas devido às descargas atmosféricas.
	
	
	
\end{document}