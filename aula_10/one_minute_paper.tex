%\documentclass[journal, onecolumn, letterpaper]{IEEEtran}
%\documentclass[journal,onecolumn]{IEEEtran}
% \documentclass[conference]{IEEEtran}
\documentclass[a4paper, 12pt, onecolumn,singlespacing]{article}

% The preceding line is only needed to identify funding in the first footnote. If that is unneeded, please comment it out.
\usepackage[level]{fmtcount} % equivalent to \usepackage{nth}
% \include{util}
\usepackage[portuguese, brazil, english]{babel}
\usepackage{multirow}
\usepackage{array} % for defining a new column type
\usepackage{varwidth} %for the varwidth minipage environment
\usepackage[super]{nth}
\usepackage{authblk}
\usepackage{cite}
\usepackage{amsmath,amssymb,amsfonts}
\usepackage{ulem}
\usepackage{graphicx}
% \usepackage{subfig}
\usepackage{textcomp}
\usepackage{xcolor}
\usepackage{mathptmx}
\usepackage[T1]{fontenc}
\usepackage{textcomp}
\usepackage{titlesec}
\usepackage{helvet}
\usepackage{gensymb}
\usepackage{setspace} % espacamento entre linhas
\usepackage{pgfplots}
\usepackage{tikz}
\usepackage{subcaption}
\usepackage{minted}
\usepackage[left=2cm, right=2cm, bottom=2cm, top=2cm]{geometry} 
\usepackage{makecell}
\usepackage{pdfpages}

\renewcommand\theadalign{bc}
\renewcommand\theadfont{\bfseries}
\renewcommand\theadgape{\Gape[4pt]}
\renewcommand\cellgape{\Gape[4pt]}

%dashed line
\usepackage{booktabs, makecell}
\renewcommand\theadfont{\bfseries}
\renewcommand\theadgape{}
\usepackage{arydshln}
\setlength\dashlinedash{0.2pt}
\setlength\dashlinegap{1.5pt}
\setlength\arrayrulewidth{0.3pt}

% padrao 1.5 de espacamento entre linhas
\setstretch{1.5}

\title{Aula 10 - Energia dissipada durante um raio}

\author[1]{Augusto Mathias Adams}
\affil[1]{augusto.adams@ufpr.br}
\setcounter{Maxaffil}{0}
\renewcommand\Affilfont{\itshape\small}

\begin{document}
	% Seleciona o idioma do documento
	\selectlanguage{brazil}
	
	% título
	\maketitle
	
	\section{Aprendizado da Aula}
	
	\begin{itemize}
		\item \textbf{\textit{Energia dissipada durante o processo de um raio $\Rightarrow$ }} A energia dissipada por um raio varia muito e depende de vários fatores, como a carga elétrica do raio, a distância percorrida pelo raio, o tipo de solo e as condições atmosféricas locais. Em média, estima-se que um raio típico carregue uma carga elétrica de cerca de 30 Coulombs e tenha uma corrente elétrica de cerca de 30.000 Ampères. Essa corrente elétrica pode gerar uma energia de cerca de 1 bilhão de joules, o que é equivalente à energia necessária para acender uma lâmpada de 100 watts por mais de 3 anos. No entanto, é importante lembrar que esses números são apenas médias e que a energia dissipada por um raio pode ser muito maior em certos casos, como quando atinge um objeto ou estrutura específica.
		
		\item \textbf{\textit{Indo mais a fundo $\Rightarrow$ }}As descargas elétricas atmosféricas apresentam dois tipos diferentes de correntes. A primeira e mais intensa é a corrente de retorno, que tem uma intensidade entre 10 e 100 kA e duração de 100 a 200 $\mu s$. O pico de corrente estimado para uma corrente de retorno é de cerca de 300 kA em regiões temperadas e 450 a 500 kA em regiões tropicais. O outro tipo de corrente é a corrente lateral corona, que corresponde ao movimento radial de íons e elétrons em direção ao solo. A potência térmica dissipada é de cerca de $2,2 × 10^10 W$, com um valor de corrente inicial de Io = 22 kA. As descargas atmosféricas contêm uma enorme quantidade de energia, o que pode resultar em incêndios florestais, mortes e ferimentos em animais, danos em edifícios, sistemas de comunicação, linhas de energia e sistemas elétricos. Aviões e ônibus espaciais também não estão totalmente seguros contra raios. A maior parte da energia é dissipada pela parte resistiva da coluna de ar, que aparece como calor ou energia térmica que eleva a temperatura do canal. A temperatura da coluna é tão elevada que produz perturbações acústicas conhecidas como "trovões". A energia total radiada considerando a corrente de retorno e a corrente lateral corona como uma só corrente (return stroke-lateral corona) é da ordem de $3,23 × 10^3 J$. A energia térmica alcança um valor de pico da ordem de $10^{10} W$.
	\end{itemize}
	
	\section{Temas Impactantes, dúvidas e questionamentos}
	
	Imagino como seria a equação desta energia: um mesmo evento dissipa potência através de 3 rotas distintas (Acústica, Luminância e Efeito Joule) e o que sobra ainda é capaz de fazer um estrago enorme aqui embaixo. Ainda bem que vamos ver o assunto em detalhes mais à frente do curso, estou interessado nestas equações.
	
\end{document}