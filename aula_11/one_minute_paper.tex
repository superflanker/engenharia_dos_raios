%\documentclass[journal, onecolumn, letterpaper]{IEEEtran}
%\documentclass[journal,onecolumn]{IEEEtran}
% \documentclass[conference]{IEEEtran}
\documentclass[a4paper, 12pt, onecolumn,singlespacing]{article}

% The preceding line is only needed to identify funding in the first footnote. If that is unneeded, please comment it out.
\usepackage[level]{fmtcount} % equivalent to \usepackage{nth}
% \include{util}
\usepackage[portuguese, brazil, english]{babel}
\usepackage{multirow}
\usepackage{array} % for defining a new column type
\usepackage{varwidth} %for the varwidth minipage environment
\usepackage[super]{nth}
\usepackage{authblk}
\usepackage{cite}
\usepackage{amsmath,amssymb,amsfonts}
\usepackage{ulem}
\usepackage{graphicx}
% \usepackage{subfig}
\usepackage{textcomp}
\usepackage{xcolor}
\usepackage{mathptmx}
\usepackage[T1]{fontenc}
\usepackage{textcomp}
\usepackage{titlesec}
\usepackage{helvet}
\usepackage{gensymb}
\usepackage{setspace} % espacamento entre linhas
\usepackage{pgfplots}
\usepackage{tikz}
\usepackage{subcaption}
\usepackage{minted}
\usepackage[left=2cm, right=2cm, bottom=2cm, top=2cm]{geometry} 
\usepackage{makecell}
\usepackage{pdfpages}

\renewcommand\theadalign{bc}
\renewcommand\theadfont{\bfseries}
\renewcommand\theadgape{\Gape[4pt]}
\renewcommand\cellgape{\Gape[4pt]}

%dashed line
\usepackage{booktabs, makecell}
\renewcommand\theadfont{\bfseries}
\renewcommand\theadgape{}
\usepackage{arydshln}
\setlength\dashlinedash{0.2pt}
\setlength\dashlinegap{1.5pt}
\setlength\arrayrulewidth{0.3pt}

% padrao 1.5 de espacamento entre linhas
\setstretch{1.5}

\title{Aula 11 - Tempestades como geradores de energia}

\author[1]{Augusto Mathias Adams}
\affil[1]{augusto.adams@ufpr.br}
\setcounter{Maxaffil}{0}
\renewcommand\Affilfont{\itshape\small}

\begin{document}
	% Seleciona o idioma do documento
	\selectlanguage{brazil}
	
	% título
	\maketitle
	
	\section{Aprendizado da Aula}
	
	\begin{itemize}
		
		\item \textbf{\textit{Tempestades Como Geradores Elétricos $\Rightarrow$ }}As tempestades podem ser consideradas verdadeiros geradores elétricos naturais, capazes de produzir tensões e correntes elétricas de grande magnitude. O processo de formação de cargas elétricas nas nuvens se dá através de diversos mecanismos, tais como a colisão entre as partículas de gelo e água presentes na nuvem, a separação de cargas na interface entre o gelo e a água, o processo de coalescência das gotículas de água, entre outros.
		
		Esses processos acabam por gerar um gradiente de potencial elétrico dentro da nuvem, podendo atingir valores da ordem de milhões de volts. Quando esse gradiente é suficientemente grande, ocorre uma descarga elétrica, que pode ser observada na forma de relâmpagos.
		
		Durante a descarga elétrica, ocorre uma intensa corrente elétrica que se propaga através da atmosfera e é capaz de aquecer o ar a temperaturas da ordem de 30.000 graus Celsius. Esse aquecimento acaba por expandir rapidamente o ar ao redor do canal da descarga, gerando ondas sonoras que são percebidas como trovões.
		
		Além dos relâmpagos, as tempestades também podem produzir outros fenômenos elétricos, tais como as correntes de retorno, que ocorrem quando a descarga elétrica é atraída pelo solo ou por objetos próximos à superfície terrestre, e os raios globulares, que são descargas elétricas esféricas que ocorrem em condições especiais e podem atingir até vários metros de diâmetro.
	
	\end{itemize}
	
	\section{Temas Impactantes, dúvidas e questionamentos}
	
	É um tema recorrente na minha cabeça não ter a visão macro da dinâmica - seria mais fácil simplesmente modelar a tempestade como um grande gerador de Van Der Graaf e ser feliz, mas não é assim que funciona. Modelar a atmosfera como um grande gerador é útil, mas não passa de um modelo que representa parcialmente a realidade. Substituir um fenômeno esparso por parâmetros concentrados é o que a gente mais fez no curso, porém, o que o modelamento não mostra são os detalhes: o que acontece na atmosfera é, em muitos aspectos, diferente do que o modelo representa.
\end{document}