%\documentclass[journal, onecolumn, letterpaper]{IEEEtran}
%\documentclass[journal,onecolumn]{IEEEtran}
% \documentclass[conference]{IEEEtran}
\documentclass[a4paper, 12pt, onecolumn,singlespacing]{article}

% The preceding line is only needed to identify funding in the first footnote. If that is unneeded, please comment it out.
\usepackage[level]{fmtcount} % equivalent to \usepackage{nth}
% \include{util}
\usepackage[portuguese, brazil, english]{babel}
\usepackage{multirow}
\usepackage{array} % for defining a new column type
\usepackage{varwidth} %for the varwidth minipage environment
\usepackage[super]{nth}
\usepackage{authblk}
\usepackage{cite}
\usepackage{amsmath,amssymb,amsfonts}
\usepackage{ulem}
\usepackage{graphicx}
% \usepackage{subfig}
\usepackage{textcomp}
\usepackage{xcolor}
\usepackage{mathptmx}
\usepackage[T1]{fontenc}
\usepackage{textcomp}
\usepackage{titlesec}
\usepackage{helvet}
\usepackage{gensymb}
\usepackage{setspace} % espacamento entre linhas
\usepackage{pgfplots}
\usepackage{tikz}
\usepackage{subcaption}
\usepackage{minted}
\usepackage[left=2cm, right=2cm, bottom=2cm, top=2cm]{geometry} 
\usepackage{makecell}
\usepackage{pdfpages}

\renewcommand\theadalign{bc}
\renewcommand\theadfont{\bfseries}
\renewcommand\theadgape{\Gape[4pt]}
\renewcommand\cellgape{\Gape[4pt]}

%dashed line
\usepackage{booktabs, makecell}
\renewcommand\theadfont{\bfseries}
\renewcommand\theadgape{}
\usepackage{arydshln}
\setlength\dashlinedash{0.2pt}
\setlength\dashlinegap{1.5pt}
\setlength\arrayrulewidth{0.3pt}

% padrao 1.5 de espacamento entre linhas
\setstretch{1.5}

\title{Segunda Aula - Formação de Nuvens de Tempestade}

\author[1]{Augusto Mathias Adams}
\affil[1]{augusto.adams@ufpr.br}
\setcounter{Maxaffil}{0}
\renewcommand\Affilfont{\itshape\small}

\begin{document}
	% Seleciona o idioma do documento
	\selectlanguage{brazil}
	
	% título
	\maketitle
	
	\section{Aprendizado da Aula}
	
	\begin{itemize}
		\item \textbf{\textit{Formação de nuvens de tempestade $\Rightarrow$ }} As nuvens de tempestade se formam quando uma parcela de ar quente e úmido sobe na atmosfera e se resfria, causando a condensação do vapor de água presente no ar. Existem vários fatores que podem causar essa elevação do ar, como os efeitos orográficos (quando o ar úmido é forçado a subir ao encontrar uma montanha, por exemplo), as correntes de advecção (quando o ar quente é transportado horizontalmente e se encontra com uma massa de ar frio), a convergência de ventos e a convecção (quando o Sol aquece a superfície terrestre e gera correntes ascendentes de ar quente). Quando essas parcelas de ar quente e úmido se elevam, formam as nuvens de tempestade, que podem resultar em raios, trovões, chuvas fortes e outros fenômenos meteorológicos intensos.
		\item \textbf{\textit{Conceitos base $\Rightarrow$ Processo de Formação das Nuvens}}
		\begin{itemize}
			\item \textbf{\textit{Processo: }}Formação da Nuvem
			\item \textbf{\textit{Combustível: }}parcela de ar quente e umido
			\item \textbf{\textit{Catalizador: }}advecção, efeitos orográficos, convergência de ventos, convecção.
		\end{itemize}
		\item \textbf{\textit{Tipos de Nuvem:}}
		\begin{itemize}
			\item  \textbf{\textit{Cirrus:}} são nuvens finas, brancas e com aparência fibrosa. Geralmente são encontradas em altitudes elevadas e indicam tempo bom, mas podem indicar mudanças no clima em algumas situações.
			
			\item \textbf{\textit{Cumulus:}} são nuvens brancas e fofas, com uma aparência de "algodão". Elas podem se formar em altitudes diferentes, mas geralmente indicam tempo bom.
			
			\item \textbf{\textit{Stratus:}} são nuvens cinzentas e uniformes, com uma aparência plana e baixa. Elas geralmente se formam em altitudes baixas e indicam tempo nublado ou chuvoso.
			
			\item \textbf{\textit{Nimbostratus:}} são nuvens densas e escuras que geralmente se formam em altitudes baixas e médias. Elas indicam chuva ou neve.
			
			\item \textbf{\textit{Cumulonimbus:}} são nuvens grandes e volumosas, com uma aparência em forma de bigorna. Elas se formam em altitudes elevadas e são frequentemente associadas a tempestades, raios e ventos fortes.
			
			\item \textbf{\textit{Stratocumulus:}} são nuvens baixas e espessas, com uma aparência de "rolos" ou "bolas". Elas geralmente indicam tempo nublado, mas podem se dissipar rapidamente.
			
			\item \textbf{\textit{Altocumulus:}} são nuvens brancas ou cinzentas que aparecem em camadas. Elas geralmente indicam tempo bom, mas também podem indicar mudanças no clima.
			
			\item \textbf{\textit{Cirrostratus:}} são nuvens finas e transparentes que parecem um véu branco. Elas geralmente indicam tempo bom, mas também podem indicar a chegada de uma frente fria.
			
			\item \textbf{\textit{Cirrocumulus:}} são nuvens pequenas e redondas que parecem bolinhas brancas. Elas geralmente indicam tempo bom, mas também podem indicar a chegada de uma frente fria.
		\end{itemize}
		\item \textbf{\textit{Estágios de Evolução de uma Nuvem de Tempestade $\Rightarrow$ }} A nuvem Cumulonimbus (Cb) é uma nuvem de tempestade que pode se formar a partir de um Cumulus quando há calor e umidade suficientes na atmosfera para alimentar sua formação. A evolução de uma nuvem Cb pode ser dividida em quatro estágios principais:
		\begin{itemize}
			\item \textbf{\textit{Estágio de Desenvolvimento:}} Neste estágio, a nuvem Cb é caracterizada por um grande volume de ar ascendente, que pode ser visto como uma torre em forma de cogumelo. A base da nuvem está a uma altitude relativamente baixa, e o topo da nuvem pode se estender a grandes altitudes. Neste estágio, a nuvem está ganhando energia e crescendo rapidamente.
			
			\item \textbf{\textit{Estágio de Maturidade}}: Neste estágio, a nuvem Cb atinge seu tamanho máximo e é caracterizada por uma grande área de precipitação. O ar ascendente continua a alimentar a nuvem, mas a área de precipitação começa a se espalhar para fora da nuvem. Neste estágio, a nuvem pode produzir trovões, relâmpagos e ventos fortes.
			
			\item \textbf{\textit{Estágio de Dissipação:}} Neste estágio, a nuvem Cb começa a perder sua energia e a se dissipar. A precipitação se torna menos intensa e a base da nuvem começa a se elevar. Neste estágio, ainda podem ocorrer trovões e ventos fortes, mas a intensidade geral da tempestade está diminuindo.
			
			\item \textbf{\textit{Estágio de Dissipação Completa:}} Neste estágio, a nuvem Cb se dissipou completamente e não há mais energia disponível para sustentá-la. A tempestade está completamente acabada, e a área afetada começa a se recuperar.
		\end{itemize}
		\item \textbf{\textit{Carregamento Elétrico de Uma nuvem de Tempestade $\Rightarrow$ }} Hipóteses que tentam explicar o carregamento de uma Nuvem de Tempestade:
		
		\begin{itemize}
			\item \textbf{\textit{ Hipótese de Carregamento Por Convecção:}} A hipótese de carregamento por convecção é uma das formas de carregamento elétrico das nuvens. Esse processo ocorre devido à movimentação vertical do ar, que é causada pela diferença de temperatura entre a superfície terrestre e a atmosfera. Durante esse processo, o ar quente e úmido sobe, enquanto o ar frio e seco desce.
			
			À medida que o ar quente sobe, ele encontra uma região mais fria e entra em processo de resfriamento. Com isso, o vapor de água presente no ar condensa, formando gotículas de água que se agrupam e formam as nuvens. Durante esse processo de condensação, a liberação de energia pode gerar uma diferença de potencial elétrico entre as diferentes regiões da nuvem.
			
			Essa diferença de potencial elétrico pode gerar um campo elétrico intenso na atmosfera, que pode resultar em descargas elétricas, como raios. A hipótese de carregamento por convecção é uma das teorias que tentam explicar a origem dos raios nas nuvens.
			
			\item \textbf{\textit{Hipótese de Carregamento por Precipitação: }}A hipótese de carregamento por precipitação é uma das teorias que explicam a formação de cargas elétricas nas nuvens. Essa teoria sugere que a colisão e a fragmentação de gotículas de água e cristais de gelo dentro de nuvens de tempestade podem gerar cargas elétricas separadas.
			
			Nas nuvens, existem regiões com grande concentração de gotículas de água e cristais de gelo em suspensão. Essas partículas são transportadas pelas correntes de ar dentro da nuvem e colidem entre si, gerando eletricidade estática. À medida que a nuvem cresce e se desenvolve, a força de convecção pode separar as cargas elétricas e gerar uma diferença de potencial elétrico entre as partes superior e inferior da nuvem.
			
			A hipótese de carregamento por precipitação é uma das hipóteses que explicam a formação de cargas elétricas em nuvens de tempestade, juntamente com a hipótese de carregamento por convecção e a hipótese de carregamento por atrito. Acredita-se que esses três processos podem trabalhar em conjunto para produzir as cargas elétricas observadas nas nuvens de tempestade.
			
			\item \textbf{\textit{Processo de Avalanche de Raios Cósmicos ou Runaway Breakdown: }} 
			Os raios cósmicos são partículas altamente energéticas que se originam no espaço sideral, incluindo em explosões de supernovas e buracos negros. Quando essas partículas chegam à Terra, elas podem causar uma cascata de partículas secundárias na atmosfera superior, que eventualmente chegam à superfície terrestre como raios cósmicos secundários.
			
			O processo de avalanche de raios cósmicos ocorre quando uma partícula de alta energia atinge a atmosfera superior e colide com um átomo. Esse choque produz partículas secundárias, que por sua vez colidem com outros átomos e produzem ainda mais partículas secundárias. Esse processo continua até que milhões de partículas tenham sido produzidas, formando uma cascata de partículas que se move em direção à Terra.
			
			À medida que as partículas secundárias se movem em direção à superfície terrestre, elas podem ser detectadas por equipamentos de detecção de raios cósmicos, como câmeras e detectores de partículas. Esses equipamentos são usados para estudar a composição e a origem dos raios cósmicos e para entender melhor como essas partículas interagem com a atmosfera da Terra.
			
			Runaway breakdown é um fenômeno atmosférico que pode ocorrer em nuvens de tempestade e que pode contribuir para a geração de descargas elétricas, incluindo raios. Esse processo envolve a ionização do ar em altas altitudes, o que pode ocorrer devido à interação entre partículas carregadas, como elétrons, e moléculas de ar.
			
			Quando uma nuvem de tempestade é carregada eletricamente, pode ocorrer a formação de um campo elétrico muito forte no seu interior. Esse campo elétrico pode acelerar elétrons a altas velocidades, que colidem com moléculas de ar e ionizam o gás. Quando isso acontece, é possível que ocorra um processo de cascata, em que as partículas carregadas geradas pelas colisões aceleram ainda mais elétrons, gerando mais ionização e assim por diante. Esse processo de cascata é chamado de runaway breakdown.
			
			O resultado do runaway breakdown é a formação de uma região altamente ionizada na nuvem de tempestade, que pode contribuir para a geração de descargas elétricas, incluindo raios. O processo de avalanche de elétrons gerado pelo runaway breakdown é capaz de produzir elétrons de alta energia, que podem colidir com moléculas de ar e gerar novas ionizações, aumentando ainda mais a corrente elétrica na nuvem e a probabilidade de descargas elétricas.
		\end{itemize}
	\end{itemize}
	

\end{document}