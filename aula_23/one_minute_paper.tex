%\documentclass[journal, onecolumn, letterpaper]{IEEEtran}
%\documentclass[journal,onecolumn]{IEEEtran}
% \documentclass[conference]{IEEEtran}
\documentclass[a4paper, 12pt, onecolumn,singlespacing]{article}

% The preceding line is only needed to identify funding in the first footnote. If that is unneeded, please comment it out.
\usepackage[level]{fmtcount} % equivalent to \usepackage{nth}
% \include{util}
\usepackage[portuguese, brazil, english]{babel}
\usepackage{multirow}
\usepackage{array} % for defining a new column type
\usepackage{varwidth} %for the varwidth minipage environment
\usepackage[super]{nth}
\usepackage{authblk}
\usepackage{cite}
\usepackage{amsmath,amssymb,amsfonts}
\usepackage{ulem}
\usepackage{graphicx}
% \usepackage{subfig}
\usepackage{textcomp}
\usepackage{xcolor}
\usepackage{mathptmx}
\usepackage[T1]{fontenc}
\usepackage{textcomp}
\usepackage{titlesec}
\usepackage{helvet}
\usepackage{gensymb}
\usepackage{setspace} % espacamento entre linhas
\usepackage{pgfplots}
\usepackage{tikz}
\usepackage{subcaption}
\usepackage{minted}
\usepackage[left=2cm, right=2cm, bottom=2cm, top=2cm]{geometry}
\usepackage{makecell}
\usepackage{pdfpages}
\usepackage{bm}

\usepackage{hyperref}
\usepackage{fancyhdr}
\renewcommand{\headrulewidth}{1pt}
\renewcommand{\footrulewidth}{0.5pt}
\fancyhf{} % limpa os cabecalhos e rodapés
\fancyhead[C]{\textit{CURSO DE ENGENHARIA DOS RAIOS - TE981} } % define o cabeçalho personalizado
\fancyfoot[C]{\textit{AUGUSTO MATHIAS ADAMS}}
\pagestyle{fancy} % sem definir esse comando, o cabeçalho personalizado não é exibido

\hypersetup{
	colorlinks=true,
	linkcolor=blue,
	filecolor=magenta,      
	urlcolor=blue,
	pdftitle={ENGENHARIA DOS RAIOS - TE981 - ONE MINUTE PAPER}
}
\renewcommand\theadalign{bc}
\renewcommand\theadfont{\bfseries}
\renewcommand\theadgape{\Gape[4pt]}
\renewcommand\cellgape{\Gape[4pt]}

%dashed line
\usepackage{booktabs, makecell}
\renewcommand\theadfont{\bfseries}
\renewcommand\theadgape{}
\usepackage{arydshln}
\setlength\dashlinedash{0.2pt}
\setlength\dashlinegap{1.5pt}
\setlength\arrayrulewidth{0.3pt}

% padrao 1.5 de espacamento entre linhas
\setstretch{1.5}
\makeatletter
\def\@maketitle{%
	\newpage
	\null
	\vskip 2em%
	\begin{center}%
		\let \footnote \thanks
		{\LARGE \@title \par}%
		\vskip 1.5em%
		{\large
			\lineskip .5em%
			\begin{tabular}[t]{c}%
				\@author
			\end{tabular}\par}%
		%\vskip 1em%
		%{\large \@date}%
	\end{center}%
	\par
	\vskip 1.5em}
\makeatother

\title{\normalsize{ENGENHARIA DOS RAIOS - TE981}\\ \huge{\textbf\textit{{AULA 23 - PRINCIPIOS DE SISTEMAS DE PROTEÇÃO CONTRA DESCARGAS ATMOSFÉRICAS}}\\}}
\author{\small{AUGUSTO MATHIAS ADAMS}}
\setcounter{Maxaffil}{0}
\renewcommand\Affilfont{\itshape\small}

\begin{document}
	% Seleciona o idioma do documento
	\selectlanguage{brazil}
	
	% título
	\maketitle
	
	\section{Aprendizado da Aula}
	
		\paragraph{Raio Crítico}
		O termo "raio crítico" refere-se ao conceito de raio de proteção efetivo de um Sistema de Proteção contra Descargas Atmosféricas (\textit{SPDA}), também conhecido como para-raios.
		
		O raio crítico representa a distância máxima a partir da qual o \textit{SPDA} é capaz de proteger efetivamente uma estrutura contra os efeitos diretos de um raio. Em outras palavras, é a distância a partir da qual o para-raios é capaz de atrair e capturar um raio antes que ele atinja a estrutura protegida.
		
		\paragraph{Raio de Atração}
		
		O termo "raio de atração" é utilizado para descrever a distância máxima a partir da qual um Sistema de Proteção contra Descargas Atmosféricas (\textit{SPDA}), como um para-raios, é capaz de atrair um raio para si, em vez de permitir que ele atinja diretamente uma estrutura ou equipamento a ser protegido.
		
		O raio de atração é determinado pela geometria e altura do sistema de proteção, bem como pelas características do solo e das condições atmosféricas. Em geral, quanto maior for a altura e o dimensionamento adequado do \textit{SPDA}, maior será o raio de atração.
		
		É importante ressaltar que o raio de atração não garante a completa prevenção de raios, mas visa a redirecionar as descargas atmosféricas para pontos seguros, como hastes de aterramento ou sistemas de captação adequados. O objetivo principal do raio de atração é minimizar os riscos de danos estruturais, incêndios e lesões causadas por raios, oferecendo uma rota preferencial para a corrente do raio seguir em direção ao solo.
		
		\paragraph{Método Eletrogeométrico}
		
		O Método Eletrogeométrico é uma abordagem utilizada para estimar o raio de atração de um Sistema de Proteção contra Descargas Atmosféricas (\textit{SPDA}) ou de uma estrutura exposta a raios. Esse método leva em consideração a geometria da estrutura e do sistema de proteção, bem como as características do solo e as condições atmosféricas.
		
		O Método Eletrogeométrico utiliza equações matemáticas e fórmulas empíricas para calcular o raio de atração com base nas dimensões da estrutura, como altura, comprimento e largura, bem como na altura e configuração do sistema de proteção. Estas consideram, principalmente, a altura do objeto e o pico da corrente de retorno.
		
		\paragraph{Modelos de Corrente de Retorno}
		
		\subparagraph{Modelos Físicos} Muito laborados, baseados em equações dinâmicas de gás, que representam as conservações de massa, momento e energia no canal do raio. Aplicado geralmente para avaliar o comportamento de variáveis físicas, como temperatura, pressão e densidade de massa do canal, como função do tempo e da coordenada radial em relação ao canal do raio. Sua formulação não é dirigida à	determinação dos parâmetros de interesse direto da engenharia de proteção.
		
		\subparagraph{Modelos De Engenharia}
		
		Prevalece a simplicidade da formulação e buscam observar a concordância entre os campos
		eletromagnéticos gerados pela distribuição de corrente, que definem ao longo do canal, com aqueles observados através	de medições, sem maiores preocupações com os aspectos físicos envolvidos.
		
		\subparagraph{Modelos Eletromagnéticos}
		
		Representam o canal do raio como uma antena com perdas, constituída por um conjunto de
		dipolos. Através da solução numérica das equações de Maxwell, aplicadas a partir do início de fluxo de corrente de retorno, os modelos calculam a distribuição dessa corrente no canal da descarga atmosférica.
		
		\subparagraph{Modelo a Parâmetros Distribuídos}
		
		Descrevem o canal de descarga como uma linha de transmissão vertical, uniforme. São atribuídos parâmetros por unidade de comprimento ($R$), ($L$), ($G$) e ($C$). A resistência por unidade de comprimento possui perdas ao longo do canal, os demais parâmetros são determinados em função da geometria assumida. Estes modelos calculam a distribuição de corrente no canal em função do tempo e altura. Não consideram as ramificações nem a tortuosidade do canal.
		
		\paragraph{Modelos de Acoplamento}
		
		\subparagraph{Modelo de Acoplamento de Rusck} Utilizado até hoje, é baseado na formulação analítica de que o solo comporta-se	como um condutor perfeito. Expressa o valor máximo de tensão induzida em função do valor de pico da corrente na base do canal.
		
		\paragraph{Tensões Induzidas Em linhas de Média Tensão}
		
		Em linhas de média tensão, o desligamento da rede pode ocorrer devido à atuação dos dispositivos de proteção, como relés e disjuntores. Para evitar desligamentos frequentes em áreas com alta incidência de descargas, especialmente em redes rurais, algumas concessionárias de energia aumentam o nível básico de isolamento da linha (NBI) de 95 kV para 170 kV. O NBI é responsável por determinar a capacidade dos dispositivos de suportar sobretensões externas, como descargas atmosféricas.
		
		\paragraph{Métodos de Simulação de Carga em Linhas de Transmissão}
		
		Falhas de blindagem (\textit{flashover}) podem ser reduzidas mesmo se utilizados cabos de proteção mal instalados. Para determinar o Número de Falhas de Blindagem em linhas de transmissão, é necessário estudar as relações entre os parâmetros estruturais e elétricos do projeto da linha. O cálculo do Número de Falhas de Blindagem (\textit{NFB}) envolve o uso dos seguintes métodos e modelos:
		
		\begin{enumerate}
			\item Método de Simulação de Carga (MSC)
			\item Modelo de Progressão do Líder (MPL)
			\item Modelo de Comprimento Crítico Equivalente de Streamers (CCES)
		\end{enumerate}
		
		O modelo CCES descreve a condição estável de início de um Líder Ascendente positivo, que posteriormente se conectará com um Líder Descendente negativo.
		
	\section{Temas Impactantes, dúvidas e questionamentos}
	
	Nada que os artigos do SIPDA não consigam resolver (Tem, inclusive, um compilado das equações de coordenação de isoladores lá.....) 
	
\end{document}